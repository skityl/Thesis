% Appendix A

\chapter{Classical Harmonic Analysis} % Main appendix title

\label{ClassicalHarmonicAnalysis} % For referencing this appendix elsewhere, use \ref{AppendixA}

\lhead{Appendix \ref{ClassicalHarmonicAnalysis}. \emph{Classical Harmonic Analysis}} % This is for the header on each page - perhaps a shortened title


\section{Introduction}
Given a function $f:\Circ\rightarrow\Cplx$, we have an associated Fourier series, 
\begin{equation}
\label{series}
    f \sim \sum_{n \in \Itgr} \hat{f}(n)z^n
\end{equation}
where $z:\Circ\rightarrow\Circ$ is the identity function, and
\begin{equation}
\label{coefficient}
    \hat{f}(n) = \int_\Circ z^{-n} f\;d\ha
\end{equation}
where $\ha$ is the normalised Haar (or arc length) measure on $\Circ$. Implicitly
$f$ is sufficiently regular so that these Fourier coefficients exist.

I have used the symbol ``$\sim$" rather than an equals sign, since in general
we do not have equality. In general, for $f \in L^1(\Circ,\ha)$, the Fourier
series might diverge almost everywhere if it is interpreted as a sum.

However, if one turns to alternative methods of summation, it is possible to 
interpret $\sim$ as an equality. 

Here we consider Abel summation.

\section{Abel summation}
Abel summation is inspired by Abel's theorem, which we prove now.
\begin{proposition}
    Suppose that
    \begin{equation*}
        f(z) = \sum_{n=0}^\infty a_n z^n
    \end{equation*}
    is a power series converging in $\{z \in \Cplx\;:\;|z| < 1\}$
    such that the coefficients come from a Banach space $(X,\|\cdot\|)$. Suppose further that the sum
    \begin{equation*}
        \sum_{n=0}^\infty a_n
    \end{equation*}
    converges. Then,
    \begin{equation*}
        \lim_{z\rightarrow 1^-} f(z) = \sum_{n=0}^\infty a_n.
    \end{equation*}
    where by $z\rightarrow 1^-$, we mean that $z$ is restricted
    to the subset of the unit disc where $|1-z| \leq M(1-|z|)$
    for some constant $M$.
\end{proposition}
\begin{proof}
    Assume without loss of generality that 
    \begin{equation*}
        \sum_{n=0}^\infty a_n = 0.
    \end{equation*}
    Now define
    \begin{equation*}
        s_k = \sum_{n=0}^k a_n
    \end{equation*}
    and $s_{-1} = 0$.
    Then we have
    \begin{equation*}
        f(z) = \sum_{n=0}^\infty (s_n-s_{n-1})z^n.
    \end{equation*}
    so
    \begin{equation*}
        f(z) = (1-z)\sum_{k=0}^\infty s_k z^k.
    \end{equation*}
    
    Let $\varepsilon > 0$, and choose $n$ large enough such that $\|s_k\| < \varepsilon$
    for $k > n$. Then we have
    \begin{equation*}
        \left\|(1-z)\sum_{k=n}^\infty s_k z^k\right\| \leq \varepsilon|1-z|\sum_{k=n}^\infty |z|^k = \varepsilon|1-z| \frac{|z|^n}{1-|z|} \leq M \varepsilon.
    \end{equation*}
    
    When $z$ is sufficiently close to $1$, we have
    \begin{equation*}
        \left\|(1-z)\sum_{k=0}^{n-1} s_k z^k\right\| < \varepsilon.
    \end{equation*}
    Hence, for $z$ sufficiently close to $1$, we have
    \begin{equation*}
        \|f(z)\| < (M+1)\varepsilon.
    \end{equation*}
\end{proof}
With this in mind, we define the Abel summation method.
\begin{definition}
    Let $\{a_k\}_{k=0}^\infty \subset X$ be a sequence in a Banach space $(X,\|\cdot\|)$. Suppose that for 
    all $r \in (1-\varepsilon,1)$, for some $\varepsilon > 0$ we have
    \begin{equation*}
        \sum_{k=0}^\infty a_k r^k
    \end{equation*}
    exists.
    Then we define the Abel sum, 
    denoted by,
    \begin{equation*}
        A-\sum_{k=0}^\infty a_k
    \end{equation*}
    as
    \begin{equation*}
        A-\sum_{k=0}^\infty a_k := \lim_{r\rightarrow 1^-} \sum_{k=0}^\infty a_k r^k.
    \end{equation*}
    if this limit exists.
\end{definition}
Abel's theorem automatically implies that the Abel sum of a series
agrees with the usual sum, however there are series which are summable
in the Abel sense but not the classical sense. It is easy to see that,
\begin{equation*}
    A-\sum_{n=0}^\infty (-1)^n = \lim_{r\rightarrow 1^-} \sum_{n=0}^\infty (-r)^n = \lim_{r\rightarrow 1^-} \frac{1}{1+r} = \frac{1}{2}.
\end{equation*}

\section{The Poisson Kernel}
To sum a Fourier series in the Abel sense, an important technical
tool is the Poisson Kernel, which we introduce in this section. 

Given $f \in L^1(\Circ,\ha)$, define
\begin{equation*}
    A_r f := \sum_{n \in \Itgr} r^{|n|} \hat{f}(n) z^n.
\end{equation*}
with $r \in (0,1)$.
$A_r f$ exists since the Fourier coefficients of $f$ are bounded. By definition,
\begin{equation*}
    A-\sum_{n\in \Itgr} \hat{f}(n) z^n = \lim_{r\rightarrow 1^-} A_r f.
\end{equation*}
where the limit is taken in an appropriate Banach space.

Like with classical and Ces\`aro sums, Abel sums can be constructed with a convolution.
\begin{proposition}
    We can write,
    \begin{equation*}
        A_r f = P_r * f.
    \end{equation*}
    where
    \begin{equation*}
        P_r = 1+\frac{rz}{1-rz} + \frac{r}{z-r}
    \end{equation*}
\end{proposition}
\begin{proof}
    The Fourier coefficients of $A_r f$ are the Fourier coefficients of
    $f$ multiplied by the coefficients of
    \begin{equation*}
        \sum_{n \in \Itgr} r^{|n|} z^n.
    \end{equation*}
    Hence define
    \begin{equation*}
        P_r = \sum_{n \in \Itgr} r^{|n|} z^n.
    \end{equation*}
    The result follows from summing this geometric series.
\end{proof}

The reason for the superiority of Abel summation
over classical summation is that the Poisson kernels form an approximate identity.
\begin{proposition}
    The Poisson kernels $\{P_r\}_{r \in (0,1)}$ form an approximate
    identity.
\end{proposition}
\begin{proof}
    By the formula,
    \begin{equation*}
        P_r = \sum_{n \in \Itgr} r^{|n|} z^n,
    \end{equation*}
    we have
    \begin{equation*}
        \int_\Circ P_r\;d\ha = 1.
    \end{equation*}
    
    To complete the proof, we convert to coordinates. Let $z = \exp(2\pi i\theta)$,
    for $\theta \in (-1/2,1/2)$, and we can regard $P_r$ as a function of $\theta$.
    
    Then we have
    \begin{align*}
        P_r(\theta) &= 1 + \frac{re^{2\pi i \theta}}{1-re^{2\pi i \theta}}+\frac{re^{-2\pi i \theta}}{1-re^{-2\pi i \theta}}\\
        &= 1 + \frac{re^{2\pi i \theta}(1-re^{-2\pi i \theta})+re^{-2\pi i\theta}(1-re^{2\pi i \theta})}{(1-re^{2\pi i \theta})(1-re^{-2\pi i \theta})}\\
        &= 1 + \frac{2r\cos(2\pi \theta)-2r^2}{1+r^2-2r\cos(2\pi \theta)}\\
        &= \frac{1-r^2}{1-2r\cos(2\pi\theta)+r^2}
    \end{align*}
    Let $\delta > 0$. 
    Now we estimate,
    \begin{equation*}
        \int_{\delta}^{1/2} P_r(\theta)\; d\theta \leq \int_\delta^{1/2} \frac{1-r^2}{1-2r\cos(2\pi \delta)+r^2}\;d\theta \leq \frac{1-r^2}{1-2r\cos(2\pi \delta)+r^2}
    \end{equation*}
    Hence this integral vanishes as $r\rightarrow\infty$.
    
    Moreover, we have
    \begin{equation*}
        1-2r\cos(2\pi \theta) + r^2 \geq (1-r)^2.
    \end{equation*}
    Hence $P_r \geq 0$. Therefore, $\|P_r\|_1 = 1$.
    
    Thus the Poisson kernels form an approximate identity.
\end{proof}
As a consequence of this, we have
\begin{enumerate}
    \item{} If $f \in C(\Circ)$, then $A_r f\rightarrow f$ uniformly.
    \item{} If $f \in L^p(\Circ,\ha)$, for $1\leq p < \infty$, we have $A_rf\rightarrow f$
    in the $L^p$ sense.
\end{enumerate}
Hence, if $f \in L^1(\Circ,\ha)$, we have
\begin{equation*}
    f(\zeta) = A-\sum_{n \in \Itgr} \hat{f}(n)\zeta^n
\end{equation*}
for almost all $\zeta \in \Circ$.

\section{Harmonic functions on $\Disc$}
The reason that Abel summation is so important in harmonic analysis
 is that there is a close connection between the Abel sums $A_r f$
for a function $f$ on $\Circ$, and functions on $\Disc := \{ z \in \Cplx\;:\; |z| < 1\}$
which are \emph{holomorphic}.

Given $f \in L^1(\Circ)$, define $\tilde{f}:\Disc\rightarrow\Cplx$
by
\begin{equation*}
    \tilde{f}(r\zeta) = A_rf(\zeta) = (P_r*f)(\zeta)
\end{equation*}
for $r \in [0,1]$ and $\zeta \in \Circ$.

Note that, by Young's convolution inequality,
\begin{equation*}
    \left(\int_\Circ |\tilde{f}(r\zeta)|^p\;d\ha(\zeta)\right)^{1/p} \leq \|P_r\|_1 \|f\|_{p}.
\end{equation*}
Hence,
\begin{equation*}
    \sup_{r \in [0,1)} \left(\int_\Circ |\tilde{f}(r\zeta)|^p\;d\ha(\zeta)\right)^{1/p} \leq \|f\|_p.
\end{equation*} 

This motivates the definition of the \emph{Hardy spaces}:
\begin{definition}[Hardy Spaces]
    Let $p \in (0,\infty]$. For $p < \infty$, let $H^p(\Circ)$ denote the space
    of complex valued functions $f$ which are complex differentiable in the open unit disc
    such that
    \begin{equation}
        \|f\|_{H^p} := \sup_{r \in [0,1)} \left(\int_\Circ |f(r\zeta)|^p \;d\ha(\zeta)\right)^{1/p} < \infty.
    \end{equation}
    For $p = \infty$, instead we require
    \begin{equation*}
        \|f\|_{H^\infty} := \sup_{\zeta \in \Disc} |f(\zeta)| < \infty.
    \end{equation*}
\end{definition}

The link between Abel summation and Hardy spaces is provided by the following theorems,
proved in \cite{katznelson}:
\begin{theorem}
\label{hardyProjection1}
    Let $f \in L^p(\Circ)$, for $p \in [1,\infty]$. Then $\tilde{f} \in H^p(\Circ)$
    if and only if $\hat{f}(n) = 0$ for $n < 0$.
\end{theorem}
\begin{theorem}
\label{hardyProjection2}
    Let $f \in H^p(\Circ)$, for $p \in [1,\infty]$. Then, for almost all $\zeta \in \Circ$,
    the limit
    \begin{equation*}
        \lim_{r\to 1^-} f(r\zeta)
    \end{equation*}
    exists, and defines a function in $L^p(\Circ)$.
\end{theorem}

Combining theorems \ref{hardyProjection1} and \ref{hardyProjection2}, we 
have the following result, also proved in \cite{katznelson}
\begin{theorem}
    Let $p \in [1,\infty]$. The space $H^p(\Circ)$ embeds
    isometrically into the subspace of $L^p(\Circ)$
    consisting of those $f$ with $\hat{f}(n) = 0$ for $n < 0$. The embedding $\iota$
    is given by, for $f \in H^p(\Circ)$
    \begin{equation}
        \iota(f)(\zeta) = \lim_{r\to 1^-} f(r\zeta)
    \end{equation}
    for almost all $\zeta \in \Circ$.
    
    The map $f\mapsto \tilde{f}$, when restricted to the subspace
    of $f$ with $\hat{f}(n) = 0$ for $n < 0$, is the inverse of $\iota$.
\end{theorem}

