% Appendix A

\chapter{Interpolation} % Main appendix title

\label{Interpolation} % For referencing this appendix elsewhere, use \ref{AppendixA}

\lhead{Appendix \ref{Interpolation}. \emph{Interpolation}} % This is for the header on each page - perhaps a shortened title



\section{Introduction}
The method of interpolation is a very powerful one in analysis,
and it allows many results to be obtained from ``edge" cases.

Historically, interpolation is motivated by the Riesz-Thorin theorem,
which we state here.
\begin{theorem}
    Let $p_0,q_0 \in [1,\infty]$ and $p_1,q_1 \in [1,\infty]$
    with $p_0 \neq p_1$ and $q_0 \neq q_1$. Suppose that $(U,\mu)$
    and $(V,\nu)$ are measure spaces. Let $T$ be an operator such that
    \begin{equation*}
        T:L^{p_0}(U)\rightarrow L^{q_0}(V)
    \end{equation*}
    with norm $M_0$ and
    \begin{equation*}
        T:L^{p_1}(U) \rightarrow L^{q_1}(V)
    \end{equation*}
    with norm $M_1$.
    
    Let $\theta \in (0,1)$, and $p_\theta^{-1} = \theta p_0^{-1}+(1-\theta)p_1^{-1}$,
    and $q_\theta^{-1} = \theta q_0^{-1}+(1-\theta)q_1^{-1}$. 
    
    Then,
    \begin{equation*}
        T:L^{p_\theta}(U)\rightarrow L^{q_\theta}(V)
    \end{equation*}
    with norm $M \leq M_0^{\theta} M_1^{1-\theta}$.
\end{theorem}
This theorem can be proved directly, however it is more insightful to prove it using
abstract interpolation theory. The purpose of these notes is to introduce this theory.


\section{Abstract Interpolation Theory}
Let $\NLS$ be the category of normed linear spaces with morphisms
given by bounded linear maps.

\begin{definition}
    A pair $X,Y \in \NLS$ is called a compatible pair if $X$ and $Y$
    are both subspaces of a topological vector space $U$.
\end{definition}

\begin{definition}
    The category $\NLS_1$ is the category of compatible pairs $(X,Y)$ of
    normed spaces, where a morphism $T:(X_1,Y_1)\rightarrow (X_2,Y_2)$
    is a linear map $T:X_1+Y_1\rightarrow X_2+Y_2$ such that $T:X_1\rightarrow X_2$
    and $T:Y_1\rightarrow Y_2$ is bounded.
\end{definition}

\begin{proposition}
    Let $\Delta:\NLS_1\rightarrow\NLS$ be the function that maps $(X,Y)$
    to $X\cap Y$, where $X \cap Y$ is given the norm,
    \begin{equation*}
        \|x\|_{X \cap Y} = \max\{\|x\|_X,\|x\|_Y\}.
    \end{equation*}
        
    Let $\Sigma:\NLS_1\rightarrow\NLS$ be given by $\Sigma((X,Y)) = X+Y$,
    where $X+Y$
    is given the norm,
    \begin{equation*}
        \|x\|_{X+Y} = \inf\{ \|x_1\|_X+\|x_2\|_Y \;:\;x = x_1+x_2,x_1 \in X,x_2 \in Y\}.
    \end{equation*}
    Then $\Delta$ and $\Sigma$ are functors. 
\end{proposition}

\begin{definition}
    An interpolation functor is a functor $\mathcal{F}:\NLS_1\rightarrow\NLS$
    such that for $(X,Y) \in \NLS_1$, we have 
    \begin{equation*}
        \Delta((X,Y)) \subseteq \mathcal{F}((X,Y)) \subseteq \Sigma((X,Y))
    \end{equation*}


We say that an interpolation functor $\mathcal{F}$
is \emph{uniform} if for any morphism $T:(X_1,Y_1)\rightarrow (X_2,Y_2)$ in $\NLS_1$,
we have
\begin{equation*}
    \|\mathcal{F}(T)\| \leq C\max\{\|T\|_{X_1\rightarrow X_2},\|T\|_{Y_1\rightarrow Y_2}\}.
\end{equation*}
for some constant $C > 0$. If $C = 1$, we say that $\mathcal{F}$ is \emph{exact}.

We say that an interpolation functor $\mathcal{F}$ is of exponent $\theta \in (0,1)$
if
\begin{equation*}
    \|\mathcal{F}(T)\| \leq C\|T\|_{X_1\rightarrow X_2}^\theta \|T\|_{Y_1\rightarrow Y_2}^{1-\theta}.
\end{equation*}
for some constant $C > 0$. If $C = 1$, we say that $\mathcal{F}$ is \emph{exact
of exponent $\theta$}.

\end{definition}

\section{Real Interpolation: The $K$ Method}
\begin{definition}
    Let $(X_0,X_1) \in \NLS_1$. For $x \in X_0+X_1$ and $t > 0$, define
    \begin{equation*}
        K(x,t;X_0,X_1) = \inf\{ \|x_0\|+t\|x_1\|\;:\;x_0 \in X_0,x_1 \in X_1,x = x_0+x_1\}.
    \end{equation*}
    Then for $\theta \in (0,1)$ and $q \in [1,\infty)$, define
    \begin{equation*}
        \|x\|_{\theta,q;K} := \left(\int_0^\infty (t^{-\theta}K(x,t;X_0,X_1))^q\;\frac{dt}{t}\right)^{1/q}
    \end{equation*}
    and for $\theta \in [0,1]$,
    \begin{equation*}
        \|x\|_{\theta,\infty;K} := \sup_{t > 0} t^{-\theta} K(x,t;X_0,X_1).
    \end{equation*}
    
    The space $K_{\theta,q}(X_0,X_1)$ is the set of $x \in X_0+X_1$
    such that $\|x\|_{\theta,q;K} < \infty$.
    
\end{definition}

\begin{proposition}
    The function $(X,Y) \mapsto K_{\theta,q}(X,Y)$ is an exact interpolation functor
    of order $\theta$.
\end{proposition}

\section{Complex Interpolation}
\begin{definition}
    Let $(X_0,X_1)$ be a compatible pair of Banach spaces. Let $\mathcal{S} := \{z \in \Cplx\;:\;\Re(z) \in (0,1)\}$. 
    
    Define the set $\mathcal{F}(X_0,X_1)$ to be the space of functions $f:\overline{\mathcal{S}}\rightarrow X_0+X_1$ which are complex differentiabile in $\mathcal{S}$, continuous on $\overline{\mathcal{S}}$
    and bounded on $\partial \mathcal{S}$.
    
    It is true that $\mathcal{F}(X_0,X_1)$ is a Banach space under the norm,
    \begin{equation*}
        \|f\|_{\mathcal{F}(X_0,X_1)} = \max\{\sup_{t \in \Rl} \|f(it)\|,\sup_{t \in \Rl}\|f(1+it)\|\}.
    \end{equation*}
    
    Let $\theta \in (0,1)$. Then, define
    \begin{equation*}
        (X_0,X_1)_\theta = \{ f(\theta)\;:\;f \in \mathcal{F}(X_0,X_1)\}.
    \end{equation*}
    We define the norm,
    \begin{equation*}
        \|x\|_{(X_0,X_1)_\theta} = \inf\{\|f\|_{\mathcal{F}(X_0,X_1)} \;:\;x = f(\theta)\}.
    \end{equation*}
\end{definition}

\begin{proposition}
    The mapping $(X_0,X_1)\rightarrow (X_0,X_1)_\theta$ is an interpolation
    functor, which is exact of exponent $\theta$.
\end{proposition}

\section{Using interpolation}
Once we know how to describe certain interpolation spaces, results
such as the Riesz-Thorin theorem become immediate.
\begin{proposition}
    Let $X$ be a measure space.
    Let $p_0,p_1 \in [1,\infty]$, and $p_\theta^{-1} = \theta p_0^{-1}+(1-\theta)p_1^{-1}$
    for $\theta \in (0,1)$.
    Then
    \begin{equation*}
        (L^{p_0}(X),L^{p_1}(X))_\theta = L^{p_\theta}(X).
    \end{equation*}
\end{proposition}

Another immediate corollary:
\begin{proposition}
    Let $G$ be a locally compact group abelian group equipped
    with bivariant Haar measure $\mu$, and $p \in [1,\infty]$. For $\varphi \in L^1(G,\mu)$, the mpa
    $Tf = \varphi * f$ is bounded on $L^p(G,\mu)$, with norm less than or equal
    to $\|\varphi\|_1$.
\end{proposition}
\begin{proof}
    To prove the case $p = 1$, we let $f \in L^p(G)$, and compute,
    \begin{align*}
        \|\varphi * f\|_1 &= \int_G \left|\int_G\varphi(x-y)f(y)\;d\mu(y)\right|\;d\mu(x)\\
        &\leq \int_G \int_G |\varphi(x-y)||f(y)|\;d\mu(y)\;d\mu(x)\\
        &= \|\varphi\|_1\|f\|_1
    \end{align*}
    where the interchange of integrals is justified by Tonelli's theorem.
    
    Now for $p = \infty$, we compute,
    \begin{equation*}
        \|\varphi*f\|_\infty = \operatorname{ess-sup}_{x \in G} \left|\int_G \varphi(x-y)f(y)\;d\mu(y)\right|
    \end{equation*}
    By H\"older's inequality, this can be bounded by $\|\varphi\|_1\|f\|_\infty$.
    
    The rest of the cases
    follow from complex interpolation of the pair $(L^1(G,\mu),L^\infty(G,\mu))$.
\end{proof} 

\begin{proposition}
    
\end{proposition}

