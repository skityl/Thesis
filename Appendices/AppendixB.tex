% Appendix A

\chapter{Lorentz Spaces} % Main appendix title

\label{AppendixB} % For referencing this appendix elsewhere, use \ref{AppendixA}

\lhead{Appendix B. \emph{Lorentz Spaces}} % This is for the header on each page - perhaps a shortened title


\section{Introduction}
Lorentz spaces form a broad generalisation of $L^p$ spaces
and weak $L^p$ spaces. These notes cover the basic definitions.

\section{Weak $L^p$-spaces}
Let $(X,\mathcal{A},\mu)$ be a $\sigma$-finite measure space. 
Given a measurable funciton $f:X\rightarrow \Cplx$, we may define the 
distribution function 
\begin{equation*}
    d_f(\alpha) = \mu\{x \in X\;:\;|f(x)| \geq \alpha\}.
\end{equation*}
for $\alpha \geq 0$. 

By Markov's inequality, for $p > 0$,
\begin{equation*}
    d_f(\alpha) \leq \frac{\|f\|_p^p}{\alpha^p}.
\end{equation*}

The linear span of the class of functions $f$ for which
\begin{equation*}
    d_f(\alpha) \leq \frac{C^p}{\alpha^p}
\end{equation*}
for some constant $C$ is called the weak $L^p$ space
or $L^{p,w}(X)$. Given $f \in L^{p,w}(X)$, define
\begin{equation*}
    \|f\|_{p,w} := \sup_{\alpha > 0} \alpha d_f(\alpha)^{1/p}.
\end{equation*}

\section{Non-increasing Rearrangements}
Again let $(X,\mathcal{A},\mu)$ be a $\sigma$-finite
measure space, and $f:X\rightarrow \Cplx$ is measurable. Then define,
for $t \geq 0$,
\begin{equation*}
    f^*(t) = \inf\{s \geq 0\;:\;d_f(s) \leq t\}.
\end{equation*}
$f^*$ is called the non-increasing rearrangement of $f$.

\begin{lemma}
    \begin{equation*}
        \left(\int_X |f|^p\;d\mu\right)^{1/p} = \left(\int_0^\infty (f^*)^p\;d\mu\right)^{1/p}
    \end{equation*}
\end{lemma}
\begin{lemma}
    \begin{equation*}
        \sup_{t>0} t^qf^*(t) = \sup_{\alpha > 0} \alpha d_f(\alpha)^{1/p}.
    \end{equation*}
\end{lemma}

Inspired by the above two results is the following definition,
\begin{definition}[Lorentz spaces]
    Let $p,q > 0$. For a $\sigma$-finite measure space $(X,\mathcal{A},\mu)$, 
    and a measurable function $f:X\rightarrow \Cplx$, define
    \begin{equation*}
        \|f\|_{p,q} = \left(\int_0^\infty (t^{1/p}f^*(t))^q\frac{dt}{t}\right)^{1/q}
    \end{equation*}
    and
    \begin{equation*}
        \|f\|_{p,\infty} = \sup_{t\geq 0} t^{1/p} f^*(t).
    \end{equation*}
\end{definition}

\section{Discrete spaces}
When $X$ is a set with counting measure, the $L^{p,q}$ spaces
are denotes $\ell^{p,q}(X)$. 

