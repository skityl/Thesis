% Chapter 1

\chapter{The Chain Rule in Quantised Calculus} % Main chapter title

\label{TheChainRule} % For referencing the chapter elsewhere, use \ref{Chapter1} 

\lhead{Chapter \ref{TheChainRule}. \emph{The Chain Rule in Quantised Calculus}} % This is for the header on each page - perhaps a shortened title

%----------------------------------------------------------------------------------------

\section{Introduction}
In classical analysis, the chain rule states that if $M$,$N$ and $P$ are manifolds, 
 and the maps $f:M\rightarrow N$ and $g:N\rightarrow P$ are smooth, then
\begin{equation*}
    d(f\circ g) = df \circ dg.
\end{equation*}
Where both sides of the equation are maps $TM\rightarrow TP$.

\begin{remark}
    The chain rule is simply a consequence of the functoriality of the tangent
    bundle construction, $M\mapsto TM$.
\end{remark}

 It is desirable
to find a noncommutative generalisation of this identity.


\section{The setting}
We let $\mathcal{L}^{p,\infty}_0$ be the closure of the finite rank
operators in the $\mathcal{L}^{p,\infty}$-metric topology.

The important property of $\mathcal{L}^{p,\infty}$ that we will needed
for this setting is that it is a symmetrically normed ideal. [CITE]

Let $(\A,\Hilb,\D)$ be a spectral triple, satisfying the following properties:
\begin{enumerate}
    \item{} For all $a \in \A$, $\qd a \in \mathcal{L}^{p,\infty}$.
    \item{} $\A$ is closed under the holomorphic functional calculus.
    \item{} Let $\A_0 \subseteq \mathcal{B}(\Hilb)$ be the collection of all
    $T$ such that $[\sgn(\D),T]$ is finite rank. Then $\A$ is contained
    within the norm-closure of $\A_0$.
    \item{} $\A$ is commutative.
\end{enumerate}

\section{The Commutator Lemma}
\begin{lemma}
\label{commutator}
    Let $(\A,\Hilb,\D)$ be as above, and $a,b \in \A$. Then
    \begin{equation*}
        [\qd a,b] \in \mathcal{L}^{p,\infty}_0.
    \end{equation*}
\end{lemma}
\begin{proof}
    Consider the map $T:\A\to\mathcal{L}^{p,\infty}$,
    \begin{equation*}
        T(x) = [\qd a,x].
    \end{equation*}
    Since $T(x)$
    is a polynomial in $x$, $T$ is continuous from $\A$ to $\mathcal{L}^{p,\infty}$
    since $\mathcal{L}^{p,\infty}$ is a symmetrically normed ideal. Now since
    $\A$ is commutative,
    \begin{equation*}
        [\qd a,x] = [a,\qd x].
    \end{equation*}
    
    Hence, for $x \in \A_0$, we have $T(x)$ is finite rank. Since $\A_0$
    is dense in $\A$, we conclude that $T(x) \in \mathcal{L}^{p,\infty}_0$
    for $x \in \A$. Now set $x = b$ and the claim is proved.
\end{proof}

\section{The Chain rule}

\begin{lemma}
    Let $(\A,\Hilb,\D)$ be as above. Let $p \in \Cplx[x_1,\ldots,x_n]$ be a polynomial,
    and $a_1,\ldots,a_n \in \A$. 
    Then,
    \begin{equation}
        \qd p(a_1,\ldots,a_n) \equiv \sum_{k=1}^n \frac{\partial p}{\partial x_k}(a_1,\ldots,a_n)\;\qd a_k\; \mod{\mathcal{L}^{p,\infty}_0}
    \end{equation}
\end{lemma}
\begin{proof}
    First we consider the case where $p$ is a monomial of a single variable,
    that is $p(x_1,\ldots,x_n) = x_j^k$ for some $j,k$. We can compute,
    \begin{align}
        \qd p(a_1,\ldots,a_n) &= [F,a_j^k] \\
        &= \sum_{m=1}^{k} a_j^{m-1} [F,a_j] a_j^{k-m}
    \end{align}
    By lemma \ref{commutator}, this implies
    \begin{align}
        \qd p(a_1,\ldots,a_n) \equiv ka_j^{k-1}\; \qd a_j\; \mod{\mathcal{L}^{p,\infty}_0}.
    \end{align}
    
    Hence the claim is proved for $p(x_1,\ldots,x_n) = x_j^k$. The general claim
    follows from the following Leibniz rule:
    \begin{align}
        \qd (ab) &= (\qd a)b + b(\qd a)\\
                 &\equiv b(\qd a) + a(\qd b) \;\mod{\mathcal{L}^{p,\infty}_0}
    \end{align}
    and linearity.
\end{proof}

\begin{proposition}
    Let $(\A,\Hilb,\D)$ be a spectral triple as above. Let $a := (a_1,\ldots,a_n) \in \A^n$,
    and let $\varphi$ be a complex valued function that is holomorphic
    on a neighbourhood of the joint spectrum, $\sigma(a_1)\times \cdots \sigma(a_n) \subset \Cplx^n$.
    
    Since $a_1,\ldots,a_n$ pairwise commute, we can define $\varphi(a)$
    by the holomorphic functional calculus.    
    
    Then,
    \begin{equation*}
        \qd \varphi(a) \equiv d\varphi \qd a \;\mod{\mathcal{L}^{p,q}_0},
    \end{equation*}
    where this is shorthand for,
    \begin{equation*}
        \qd \varphi(a_1,\ldots,a_n) \equiv \sum_{j=1}^n \frac{\partial \varphi}{\partial z_j}(a)\qd a_j \;\mod{\mathcal{L}^{p,\infty}_0}.
    \end{equation*}
\end{proposition}
\begin{proof}
    
\end{proof}


