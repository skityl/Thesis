% Chapter 1

\chapter{The Chain Rule in Quantised Calculus} % Main chapter title

\label{TheChainRule} % For referencing the chapter elsewhere, use \ref{Chapter1} 

\lhead{Chapter \ref{TheChainRule}. \emph{The Chain Rule in Quantised Calculus}} % This is for the header on each page - perhaps a shortened title

%----------------------------------------------------------------------------------------

\section{Introduction}
In classical analysis, the chain rule states that if $M$,$N$ and $P$ are manifolds, 
 and the maps $f:M\rightarrow N$ and $g:N\rightarrow P$ are smooth, then
\begin{equation*}
    d(f\circ g) = df \circ dg.
\end{equation*}
Where both sides of the equation are maps $TM\rightarrow TP$.

\begin{remark}
    The chain rule is simply a consequence of the functoriality of the tangent
    bundle construction, $M\mapsto TM$.
\end{remark}

 It is desirable
to find a noncommutative generalisation of this identity.


\section{The setting}
We let $\mathcal{L}^{p,\infty}_0$ be the closure of the finite rank
operators in the $\mathcal{L}^{p,\infty}$-metric topology.

Let $(\A,\Hilb,\D)$ be a spectral triple, satisfying the following properties:
\begin{enumerate}
    \item{} For all $a \in \A$, $\qd a \in \mathcal{L}^{p,\infty}$.
    \item{} $\A$ is closed under the holomorphic functional calculus.
    \item{} Let $\A_0 \subseteq \mathcal{B}(\Hilb)$ be the collection of all
    $T$ such that $[\sgn(\D),T]$ is finite rank. Then $\A$ is contained
    within the norm-closure of $\A_0$.
    \item{} $\A$ is commutative.
\end{enumerate}

\section{The Commutator Lemma}
\begin{lemma}
    Let $(\A,\Hilb,\D)$ be as above, and $a \in \A$. Then
    \begin{equation*}
        [\qd a,a] \in \mathcal{L}^{p,\infty}_0.
    \end{equation*}
\end{lemma}
\begin{proof}
    By our assumptions on $(\A,\Hilb,\D)$ there exists a sequence $\{a_n\}_{n=0}^\infty$
    with $\|a_n-a\| \rightarrow 0$, and 
    $\qd a_n$ is finite rank. Hence $[\qd a_n,a]$ is finite rank, and
    \begin{equation}
        [\qd a,a_n] &= \qd([a_n,a]) + [\qd a_n,a]
                    &= [\qd a_n,a] \in \mathcal{L}^{p,\infty}_0.
    \end{equation}
\end{proof}

\begin{lemma}
    
\end{lemma}
