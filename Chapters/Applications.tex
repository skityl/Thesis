% Chapter Template

\chapter{Applications} % Main chapter title

\label{Applications} % Change X to a consecutive number; for referencing this chapter elsewhere, use \ref{ChapterX}

\lhead{Chapter \ref{Applications}. \emph{Applications}} % Change X to a consecutive number; this is for the header on each page - perhaps a shortened title

%----------------------------------------------------------------------------------------
%	SECTION 1
%----------------------------------------------------------------------------------------

\section{Introduction}
So far we have only covered quantised calculus on its own, 
with the aim of proving analogies of classical facts. It is natural
to wonder whether there is any application of quantised calculus. This
chapter very briefly, and without proof, collects
some results about the application of quantised calculus to the computation
of the Hausdorff measure of Julia sets.

\section{The non-commutative integral}
We have not yet discussed integration in quantised calculus. We give
a very brief overview here. The book \cite{Connes94} goes into further
detail, and the book \cite{SingularTraces} covers more technical topics.

\begin{definition}
    A functional $\omega:\ell^\infty(\Ntrl)$ is called an extended
    limit if it satisfies 
\end{definition}    






