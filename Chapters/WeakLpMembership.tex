% Chapter 1

\chapter{Weak $\mathcal{L}^p$ membership of quantised differentials} % Main chapter title

\label{WeakLpMembership} % For referencing the chapter elsewhere, use \ref{Chapter1} 

\lhead{Chapter \ref{WeakLpMembership}. \emph{Weak $\mathcal{L}^p$ membership of quantised differentials}} % This is for the header on each page - perhaps a shortened title

%----------------------------------------------------------------------------------------


\section{Introduction}
We have already given a thorough description of the quantised differentials $\qd f$
when $f$ is a function on the circle. It so happens that we can find
analogous, although weaker, results in far greater generality.

It is in fact easier to move to a setting that is extremely general.

Recall from Chapter \ref{Introduction} that a general non-commutative
space should be thought of as a pair $(\A,\Hilb)$, where $\A$
is an algebra of operators on a Hilbert space $\Hilb$. 

A (commutative) example of this is as follows: Let $(M,g)$ be a compact Riemannian manifold,
and let $\A = C^\infty(M)$, and $\Hilb = L^2(M,g)$. However
this is not enough information to recover the geometry of $M$. A convenient
way to recover the geometry of $M$ from algebraic data is to give a Dirac operator.
The definition of a general Dirac operator is given in Appendix \ref{Dirac}. 

\section{The setting}
We shall consider noncommutative spaces equipped with a ``Dirac Operator", $(\A,\Hilb,\D)$
such that $\A$ is contianed in a semi-finite Von Neumann
algebra $\N$ which is contained in $\mathcal{B}(\Hilb)$.


Formally,
\begin{definition}[Spectral Triple]
    A (semifinite) spectral triple is a triple $(\A,\Hilb,\D)$, where
    $\Hilb$ is a Hilbert space with $\A \subseteq \mathcal{B}(\Hilb)$
    a $*$-algebra of operators.
    
    $\D$ is a densely defined unbounded operator on $\Hilb$ satisfying the follownig
    two properties:
    \begin{enumerate}
        \item{} $[\D,a]$ is densely defined and extends to a bounded operator in $\N$
        for all $a \in \A$.
        \item{} $(\lambda -\D)^{-1}$ is $\tau$-compact for all $\lambda \in \Cplx\setminus \Rl$.
    \end{enumerate}
    
    A spectral triple $(\A,\Hilb,\D)$ can be either \emph{even} or \emph{odd}:
    \begin{itemize}
        \item{} We say that $(\A,\Hilb,\D)$ is even if there exists a
        $\Itgr/2\Itgr$ grading on the linear operators on $\Hilb$ such that $\A$
        is even and $\D$ is odd. Equivalently, there is an operator $\Gamma$
        on $\Hilb$ with $\Gamma^2 = 1$ and $\Gamma^* = \Gamma$ such that $a\Gamma = \Gamma a$ for all $a \in \A$
        and $\D\Gamma=-\Gamma\D$. 
        \item{} If no such operator $\Gamma$ exists, then we say that $(\A,\Hilb,\D)$
        is odd.
    \end{itemize}
\end{definition}

\begin{definition}[Quantum Differentiability]
    A spectral triple $(\A,\Hilb,\D)$ is called $QC^k$ for $k \geq 0$
    if $\A$ is contained in the domain of the operator $\delta^k$, where $\delta(a) = [|\D|,a]$.
\end{definition}

\begin{definition}[Summability]
    A spectral triple $(\A,\Hilb,\D)$ is called $(p,\infty)$ summable
    for $p > 0$ if $(1+\D^2)^{-1/2} \in \mathcal{L}^{p,\infty}$.
\end{definition}

        
\section{The Theorem}

\begin{theorem}
    Let $k\geq 0$ and $p \geq 1$, and
    let $(\A,\Hilb,\D)$ be a $(p,\infty)$ summable $QC^k$ spectral triple such that $\D$
    is invertible. Let $F = \sgn(\D)$ such that $\D = F|\D|$. Then for all $a \in \A$,
    \begin{equation}
        [F,a] \in \mathcal{L}^{p,\infty}.
    \end{equation}
\end{theorem}
\begin{proof}
    Since $\D = |\D|F$, we can compute
    \begin{equation}
        [\D,a] = |\D|[F,a] + [|\D|,a]F.
    \end{equation}
    Or in other words,
    \begin{equation}
        da = |\D|\qd a + \delta(a)F.
    \end{equation}
    Since $\D$ is invertible, $|\D|$ is invertible. Thus,
    \begin{equation*}
        \qd a = |\D|^{-1}da + |\D|^{-1}\delta(a)F.
    \end{equation*}
    By assumption, $da,\delta(a) \in \mathcal{B}(\Hilb)$. Thus,
    since $(\A,\Hilb,\D)$ is $(p,\infty)$-summable, $|\D|^{-1} \in \mathcal{L}^{p,\infty}$.
    Hence, $\qd a \in \mathcal{L}^{p,\infty}$.

%    By the continuous functional calculus, we have the formula,
%    \begin{equation}
%        |\D|^{-1} = \frac{1}{\pi}\int_{0}^\infty \frac{(\lambda+\D^2)^{-1}}{\sqrt{\lambda}}\;d\lambda.
%    \end{equation}
%    
%    Let $a \in \A$,
%    Since by definition $F = \D|\D|^{-1}$,
%    \begin{align*}
%        [F,a] &= [\D|\D|^{-1},a]\\
%              &= [\D,a]|\D|^{-1}+\D[|\D|^{-1},a].
%    \end{align*}
%    Now we use the integral representation of $|\D|^{-1}$,
%    \begin{equation}
%        [F,a] = \frac{1}{\pi}\int_0^\infty ([\D,a](\lambda+\D^2)^{-1}+\D[(\lambda+\D^2)^{-1},a])\frac{d\lambda}{\sqrt{\lambda}}.
%    \end{equation}
%    Using the fact that $\D^2(\lambda+\D^2)^{-1} = 1-\lambda(\lambda+\D^2)^{-1}$,
%    we get
%    \begin{equation}
%        [F,a] = \frac{1}{\pi}\int_0^\infty (\lambda(\lambda+\D^2)^{-1}[\D,a](\lambda+\D^2)^{-1}-\D(\lambda+\D^2)^{-1}[\D,a]\D(\lambda+\D^2)^{-1})\frac{d\lambda}%{\sqrt{\lambda}}.
%    \end{equation}
%    
%    Now assume that $a^* = -a$, so that $[\D,a]^* = [\D,a]$ and $[F,a]^* = [F,a]$.
%    
%    Now for any $T \in \N$, we have
%    \begin{equation}
%        -\|[\D,a]\|T^*T \leq T^*[\D,a]T \leq \|[\D,a]\|T^*T.
%    \end{equation}
%    Hence,
%    \begin{align*}
%        [F,a] \leq \frac{\|[\D,a]\|}{\pi} \int_0^\infty (\lambda+\D^2)^{-1} \frac{d\lambda}{\sqrt{\lambda}}
%        &= \|[\D,a]\||\D|^{-1}.
%    \end{align*}
%    Similarly,
%    \begin{equation}
%        [F,a] \geq -\|[\D,a]\||\D|^{-1}.
%    \end{equation} 
%    
%    Thus since $|\D|^{-1} \in \mathcal{L}^{p,\infty}$, 
%    we have $[F,a] \in \mathcal{L}^{p,\infty}$ for $a^* = -a$,
%    which we extend to all $a \in \A$ by linearity.
\end{proof} 
