% Chapter 

\chapter{Operator Ideal Membership of Quantised Differentials} % Main chapter title

\label{IdealMembership} % Change X to a consecutive number; for referencing this chapter elsewhere, use \ref{ChapterX}

\lhead{Chapter \ref{IdealMembership}. \emph{Operator Ideal Membership of Quantised Differentials}} % Change X to a consecutive number; this is for the header on each page - perhaps a shortened title

\section{Transference from Hankel Operators to Quantised differentials}
Let $\varphi \in L^2(\Circ)$. We consider $\qd \varphi$
as an operator on $H^2(\Circ)\oplus H^2_-(\Circ)$. Then we have the description,
\begin{equation}
    \qd \varphi = 2\begin{pmatrix}
        0 & -H_{\overline{\varphi_+}}^*\\
        H_{\varphi_-}  & 0
    \end{pmatrix}
\end{equation}

Hence, to determine when $\qd \varphi$ falls into some ideal of operators,
it frequently suffices to check $H_{\overline{\varphi_+}}$ and $H_{\varphi_-}$.

The weakest condition that we can place on an operator is that it be bounded.
The Nehari theorem \ref{nehari} gives us necessary and sufficient conditions
for a quantised differential to be bounded.
\begin{proposition}
    Let $\varphi \in L^1(\Circ)$. Then $\qd \varphi$ defines a bounded
    linear operator on $L^2(\Circ)$ if and only if $\varphi \in \BMO(\Circ)$.
\end{proposition}
\begin{proof}
    By theorem \ref{nehari}, we have that $\qd \varphi$ is bounded
    if and only if $\overline{\varphi_+},\varphi_- \in \BMO(\Circ)$. 
    
    By the description of $\BMO(\Circ)$ as the set of all $\varphi \in L^1(\Circ)$
    such that
    \begin{equation}
        \sup_{I} \int_{I} |f-\int_I f\;d\ha| \;d\ha,
    \end{equation}
    it is clear that $f \in \BMO(\Circ)$ if and only if $\overline{f} \in \BMO(\Circ)$.
    
    By the Fefferman decomposition, given in \cite{Garnett},
    \begin{equation}
        \BMO(\Circ) = L^\infty(\Circ)+\Proj_+L^\infty(\Circ)
    \end{equation}
    it follows that $f \in \BMO(\Circ)$ if and only if $\Proj_+f,\Proj_-f \in \BMO(\Circ)$.
    
    Thus, $\qd \varphi$ is bounded if and only if $\varphi \in \BMO(\Circ)$.
\end{proof}

On the other hand, the \emph{strongest} conditions that one
can place on an operator is that it be finite rank. Kronecker's theorem \ref{kronecker}
gives us conditions for a quantised differential to be finite rank.
\begin{proposition}
    Let $\varphi \in L^1(\Circ)$. Then the operator $\qd \varphi$
    on $L^2(\Circ)$ is finite rank if and only if $\varphi$
    is a rational function.
\end{proposition}


To complete our description, we need the following:
\begin{lemma}
    Let $\varphi \in L^1(\Circ)$. Then $\varphi \in K(B_{11}^1,\VMO)_{\theta,q}$ if and only
    if $\overline{\varphi_+},\varphi_- \in K(B_{11}^1,\VMO)_{\theta,q}$.
\end{lemma}
\begin{proof}
    
\end{proof}
    
So we get the following:
\begin{corollary}
    Let $\varphi\in K(B_{11}^1,\VMO)$. Then $\qd \varphi \in \mathcal{L}_{p,q}$. 
\end{corollary}

\section{Quantised Differentials on $\Rl$}

