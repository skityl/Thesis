% Chapter 1

\chapter{Higher Dimensions} % Main chapter title

\label{Chapter7} % For referencing the chapter elsewhere, use \ref{Chapter1} 

\lhead{Chapter 7. \emph{Higher Dimensions}} % This is for the header on each page - perhaps a shortened title

%----------------------------------------------------------------------------------------
\section{Introduction}

Let $d > 0$ be a positive integer. 
Let $\A_{\lambda}$ be the $d$-dimensional non-commutative torus, that is the universal $C^*$-algebra
generated by unitaries $\{U_j\}_{j=1}^d$ subject to the commutation
relation $U_j U_k = \lambda_{jk} U_k U_j$ for a unimodular
scalar constant $\lambda_{jk}$, where we assume that $\lambda_{jj} = 1$
for all $j$, and $\lambda_{jk} = \lambda_{kj}^{-1}$.

$\A_\lambda$ has a unique normalised trace $\tau$.
The Hilbert space $L^2(\A_\lambda,\tau)$ 
has orthonormal basis $\{u(n)\}_{n \in \Itgr^d}$ where $u(n) = U_1^{n_1}U_2^{n_2}\cdots U_d^{n_d}$.

Define $\omega_{j,k}$ for $j,k \in \Itgr^d$ to be such
that $u(j)u(k) = \omega_{j,k}u(k+j)$.

Given $f \in L^2(\A_\lambda,\tau)$, we define
\begin{equation}
    \hat{f}(n) := \tau(fu(n)^*).
\end{equation}
Then we have
\begin{equation}
    f = \sum_{k \in \Itgr^d} \hat{f}(k)u(k),
\end{equation}
which converges in the $L^2$-sense. Note that $\tau(f) = \hat{f}(0)$.

For a function $F:\Itgr^d\rightarrow \Rl$, define $F(D) u(k) = F(k)u(k)$
as a densely defined linear operator on $L^2(\A_\lambda,\tau)$. 

It is easy to see that $F(D)$ is bounded on $L^2(\A_\lambda,\tau)$
if and only if $F \in \ell^\infty(\Itgr^d)$, with operator norm
$\|F\|_\infty$.

We shall identify $u(n) \in \A_\lambda$ with the multiplication operator
$M_{u(n)}f = u(n)f$ for $f \in L^2(\A_\lambda,\tau)$

\section{Commutators}
Let $n \in \Itgr^d$. We are interested in $[F(D),u(n)]$. 

Let $f \in L^2(\A_\lambda,\tau)$. 

Then,
\begin{align}
    u(n)f &= \sum_{k \in \Itgr^d} \hat{f}(k)u(n)u(k)\\
          &= \sum_{k \in \Itgr^d} \hat{f}(k)\omega_{n,k} u(n+k).
\end{align}
Hence,
\begin{equation}
    F(D) u(n)f = \sum_{k \in \Itgr^d} \hat{f}(k) \omega_{n,k} F(n+k) u(n+k).
\end{equation}

Now we compute,
\begin{equation}
    u(n)F(D)f = \sum_{k \in \Itgr^d} \hat{f}(k) F(k) \omega_{n,k} u(n+k).
\end{equation}

Thus,
\begin{equation}
    [F(D),u(n)]f = \sum_{k \in \Itgr^d} \hat{f}(k) (F(n+k)-F(k))\omega_{n,k}u(n+k).
\end{equation}

We can simplify this further,
\begin{align}
    [F(D),u(n)]f &= \sum_{k \in \Itgr^d} \hat{f}(k) (F(n+k)-F(k))u(n)u(k)\\
                 &= u(n)\sum_{k \in \Itgr^d}  \hat{f}(k) F(n+k)u(k) - u(n) \sum_{k \in \Itgr^d} \hat{f}(k)F(k)u(k)\\
                 &= u(n) F(n+D)f - u(n)F(D)f\\
                 \label{commutatorFormula}
                 &= u(n)F(n+D)f - u(n)F(D)f
\end{align}
So $[F(D),u(n)] = u(n)(F(n+D)-F(D))$. 

\section{Hilbert-Schmidt Commutators}
It is of interest to ask when $[F(D),u(n)] \in \mathcal{L}^2$. This is equivalent to,
\begin{equation}
    \|[F(D),u(n)]\|_{2}^2 := \sum_{j,k \in \Itgr^d} |\tau(u(j)^* [F(D),u(n)]u(k))|^2 < \infty.
\end{equation}

So we compute,
\begin{align}
    [F(D),u(n)]u(k) &= u(n)F(n+D)u(k) - u(n)F(D)u(k)\\
                    &= u(n)F(n+k)u(k) - u(n)F(k)u(k)\\
                    &= (F(n+k)-F(k))u(n)u(k).
\end{align}
Hence,
\begin{align}
    |\tau(u(j)^*[F(D),u(n)]u(k))| &= |F(n+k)-F(k)||\tau(u(j)^*u(n)u(k))|\\
                                  &= |F(n+k)-F(k)|\delta_{j,n+k}\\
\end{align}

So therefore
\begin{align}
    \|[F(D),u(n)]\|_{2}^2 &= \sum_{j,k \in \Itgr^d} |F(n+k)-F(k)|^2 \delta_{j,n+k}\\
                          &= \sum_{k \in \Itgr^d} |F(n+k)-F(k)|^2.
\end{align}

So we have necessary and sufficient conditions for $[F(D),u(n)] \in \mathcal{L}^2$. 

\section{Membership of $\mathcal{L}^p$}
We can see that since $F(D)$ and $F(n+D)$ are self adjoint, 
\begin{equation}
    [F(D),u(n)]^* = (F(n+D)-F(D))u(n)^*
\end{equation}
Hence,
\begin{equation}
    [F(D),u(n)]^*[F(D),u(n)] = (F(n+D)-F(D))^2.
\end{equation}
Hence,
\begin{equation}
    |[F(D),u(n)]| = |F(n+D)-F(D)|.
\end{equation}

Now,
\begin{align}
    \operatorname{Tr}(|[F(D),u(n)]|^p) &= \sum_{k \in \Itgr^d} \tau(u(k)^*|F(D+n)-F(D)|^pu(k))\\
    &= \sum_{k \in \Itgr^d} \tau(u(k)^*|F(k+n)-F(k)|^pu(k))\\
    &= \sum_{k \in \Itgr^d} |F(k+n)-F(k)|^p.
\end{align}

Hence, for any $p > 0$, we have $[F(D),u(n)] \in \mathcal{L}^p$ if and only if
$\{F(k+n)-F(k)\} \in \ell^p(\Itgr^d)$ with an equality of (quasi-)norms.


\section{Double commutators}

We are also interested in the properties of the double commutators,.
\begin{equation*}
    [[F(D),u(n)],u(n)]
\end{equation*}
for $n \in \Itgr^d$. By equation \ref{commutatorFormula}, this is
\begin{align}
    [[F(D),u(n)],u(n)] &= [F(D),u(n)]u(n) - u(n)[F(D),u(n)] \\
                       &= u(n)(F(n+D)-F(D))u(n)-u(n)u(n)(F(D+n)-F(D))\\
                       &= (F(D)-F(D-n))u(n)^2-(F(D-n)-F(D-2n))u(n)^2\\
                       &= (F(D-2n)-2F(D-n)+F(D))u(n)^2\\
                       &= u(n)(F(D+n)-2F(D)+F(D-n))u(n).
\end{align}
Hence,
\begin{equation*}
    |[[F(D),u(n)],u(n)]| = |F(D+2n)-2F(D+n)+F(D)|.
\end{equation*}

So we have $[[F(D),u(n)],u(n)] \in \mathcal{L}^p$
if and only if $\{F(k+n)-2F(k)+F(k-n)\}_{k\in \Itgr^d} \in \ell^p(\Itgr^d)$.



