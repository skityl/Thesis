% Chapter 3

\chapter{Basic Properties of Hankel Operators} % Main chapter title

\label{PropertiesOfHankelOperators} % For referencing the chapter elsewhere, use \ref{Chapter1} 

\lhead{Chapter \ref{PropertiesOfHankelOperators}. \emph{Basic Properties of Hankel Operators}} % This is for the header on each page - perhaps a shortened title

%----------------------------------------------------------------------------------------


\section{Definition of a Hankel matrix}
A Hankel matrix is an infinite matrix $\{M_{j,k}\}_{j,k \geq 0}$
whose $(j,k)th$ entry depends only on $j+k$. If $a = \{a_j\}_{j\geq 0}$,
Let $M_a = \{a_{j+k}\}_{j,k\geq 0}$ be the Hankel matrix with $(j,k)$th
entry $a_{j+k}$. That is,
\begin{equation*}
    M_a = \begin{pmatrix}
        a_0 & a_1 & a_2 & a_3 & \cdots\\
        a_1 & a_2 & a_3 & a_4 & \cdots\\
        a_2 & a_3 & a_4 & a_5 & \cdots\\
        a_3 & a_4 & a_5 & a_6 & \cdots\\
        \vdots & \vdots & \vdots & \vdots & \ddots
    \end{pmatrix}.
\end{equation*}
An infinite matrix does not necessarily define an operator on $\ell^2(\Ntrl)$, 
however any infinite matrix can be identified with a linear operator 
on the dense subset $c_{00}(\Ntrl) \subset \ell^2(\Ntrl)$ of sequences
of finite support. 

For a sequence $a \in c_{00}(\Ntrl)$, and an infinite matrix $M = (M_{j,k})_{j,k\geq 0}$, 
we define $Ma \in c_{00}(\Ntrl)$ as $Ma = \{\sum_{k=0}^\infty M_{j,k}a_k\}_{j=0}^\infty$.

Hence we shall interchangeably talk about infinite matrices and linear
operators $c_{00}(\Ntrl)\rightarrow c_{00}(\Ntrl)$.

Since $c_{00}(\Ntrl)$ is dense in $\ell^2(\Ntrl)$, if an infinite matrix
considered as an operator on $c_{00}(\Ntrl)$ is bounded on $c_{00}(\Ntrl)$
in the $\ell^2$-norm, then the matrix extends uniquely to an operator on $\ell^2(\Ntrl)$.
Conversely, any operator on $c_{00}(\Ntrl)$ which extends to a bounded
operator on $\ell^2(\Ntrl)$ is bounded on $c_{00}(\Ntrl)$ in the $\ell^2$-norm,
and the extension to $\ell^2(\Ntrl)$ is unique.

Denote the inner product on $\ell^2(\Ntrl)$ 
as $(a,b) := \sum_{n=0}^\infty \overline{a_n}b_n$, which we note
is linear in the second argument.

We shall be chiefly concerned with determining conditions on the sequence $\{a_n\}_{n=0}^\infty$
such that the associated Hankel matrix $\{a_{j+k}\}_{j,k\geq 0}$ lies in some ideal
of compact operators. According to the ``Calkin correspondence" philosophy, (explained
in detail in Appendix \ref{IdealsOfOperators}), an ideal of compact operators
can be determined by an associated rate of decay of singular values.

For a (complex separable) Hilbert space $\Hilb$, and an operator $T \in \mathcal{B}(\Hilb)$, 
we say that $T \in \mathcal{L}^{p,q}$ if $\{\mu_n\}_{n=0}^\infty \in \ell^{p,q}(\Ntrl)$.
The main goal of this chapter is to determine sufficient conditions on $\{a_n\}_{n=0}^\infty$
such that the infinite matrix $\{a_{j+k}\}_{j,k\geq 0}$ defines an operator
on $\ell^2(\Ntrl)$ that is $\mathcal{L}^{p,q}$.


See Appendix \ref{ClassicalHarmonicAnalysis} for the elementary properties of the Fourier transform. 

A \emph{polynomial} on $\Circ$ is a finite linear combination of the \emph{monomials}
$\{z^n\}_{n \in \Itgr}$. We call the space of polynomials $P(\Circ)$.
An \emph{analytic polynomial} is a polynomial consisting only of non-negative powers
of $z$, we denote $P_A(\Circ)$ for the space of analytic polynomials. 

It is easy to see that the Fourier transform gives a vector space isomorphism between $P(\Circ)$
and $c_{00}(\Itgr)$, and $P_A(\Circ)$ and $c_{00}(\Ntrl)$.

\section{Bounded Hankel operators}
It if of interest to determine when a Hankel operator defines
a bounded linear operator on $\ell^2(\Ntrl)$. This is answered completely by the
\emph{Nehari theorem}, which we cover now.
\begin{theorem}
\label{nehari}
    Let $a = \{a_j\}_{j=0}^\infty$ be a sequence. Then the 
    associated Hankel matrix $M_a$ defines a bounded linear operator on $\ell^2(\Ntrl)$
    if and only if there exists $\psi \in L^\infty(\Circ,\ha)$ such that
    \begin{equation*}
        \hat{\psi}(m) = a_m
    \end{equation*}
    for $m \geq 0$.
\end{theorem}
\begin{proof}
    Suppose first that there is $\psi \in L^\infty(\Circ)$
    such that $\hat{\psi}(m) = a_m$ for $m\geq 0$. 
    
    Choose $f,h \in P_A(\Circ)$, so that $\hat{f},\hat{h} \in c_{00}(\Ntrl)$.
    
    Let $g \in P_A(\Circ)$ be given by $g = \sum_{n=0}^\infty \overline{\hat{g}(n)}z^n$.
    Let $q = fg$.
    
    Then we compute,
    \begin{align*}
        (\hat{h},M_a\hat{f}) &= \sum_{j,k \geq 0} \overline{\hat{h}(j)}a_{j+k}\hat{f}(k)\\
        &= \sum_{j,k\geq 0} \overline{\hat{h}(j)}\hat{\psi}(j+k)\hat{f}(k)\\
        &= \sum_{j \geq 0} \hat{\psi}(j) \sum_{k=0}^j \hat{g}(j)\hat{f}(j-k)\\
        &= \sum_{j \geq 0} \hat{\psi}(j) \hat{q}(j)\\
        &= \int_\Circ \psi(\zeta) q(\overline{\zeta}) \;d\ha(\zeta).
    \end{align*}
    Hence,
    \begin{align*}
        |(\hat{h},M_a\hat{f})| &\leq \|\psi\|_{\infty} \|q\|_1\\
                               &\leq \|\psi\|_\infty \|f\|_2 \|h\|_2\\
                               &= \|\psi\|_\infty \|\hat{f}\|_2\|\hat{h}\|_2.
    \end{align*}
    And thus $M_a$ is bounded on $\ell^2(\Ntrl)$.
    
    Conversely, suppose that $M_a$ is bounded on $\ell^2(\Ntrl)$. 
    
    Let $\mathcal{L}$ be the linear functional on $P(\Circ)$ defined by
    \begin{equation*}
        \mathcal{L}(q) := \sum_{n\geq 0} a_n \hat{q}(n).
    \end{equation*}
    If $a \in \ell^1(\Ntrl)$, then $\mathcal{L}$ is bounded on $H^1(\Circ)$,
    since the inverse fourier transform of $a$ is in $L^\infty(\Circ)$. 
    Now let us prove in this case that $\|\mathcal{L}\| \leq \|M_a\|$. 
    
    Let $q \in H^1(\Circ)$, with $\|q\|_1 \leq 1$. Then $q = fg$
    for some $f,g \in H^2(\Circ)$ with $\|f\|_2,\|g\|_2 \leq 1$. 
    
    Then we can compute,
    \begin{align*} 
        |\mathcal{L}(q)| &= \left|\sum_{n=0}^\infty a_n \sum_{m=0}^n \hat{f}(m)\hat{g}(n-m)\right|\\
        &= \sum_{n,m\geq 0} a_{n+m} \hat{f}(n) \hat{g}(m)\\
        &= (M_a \hat{f},\overline{\hat{g}}). 
    \end{align*}
    And hence,
    \begin{equation*}
        |\mathcal{L}(q)| \leq \|M_a\|\|f\|_2\|g\|_2 \leq \|M_a\|.
    \end{equation*}
    So $\mathcal{L}(q)$ is bounded on $H^1(\Circ)$ whenever $a \in \ell^1(\Ntrl)$. 
    
    Now we consider $a$ to be an arbitrary sequence such that $M_a$ is bounded. Let $r \in (0,1)$
    and
    \begin{equation*}
        a^{(r)} = \{r^ja_j\}_{j\geq 0}.
    \end{equation*}
    Then $a^{(r)} \in \ell^1(\Ntrl)$,
    
    Now we can see that $M_{a^{(r)}} = D_r M_a D_r$, where $D_r$
    is multiplication by the sequence $\{r^j\}_{j\geq 0}$. Since $\|D_r\| \leq 1$, 
    we must have $\|M_{a^{(r)}}\| \leq \|M_a\|$. Since $a^{(r)} \in \ell^1(\Ntrl)$, 
    we have that the linear functional
    \begin{equation*}
        \mathcal{L}_r(q) := \sum_{n=0}^\infty a_nr^n \hat{q}(n)
    \end{equation*}
    is bounded on $H^1(\Circ)$, and the functionals $\{\mathcal{L}_r\}_{r \in (0,1)}$
    converge strongly to $\mathcal{L}$, and are uniformly bounded. Hence $\mathcal{L}$
    is continuous on $H^1(\Circ)$. 
    
    Now by the Hahn-Banach theorem, since $\mathcal{L}$ is a linear
    functional on the subspace $H^1(\Circ)$ which has norm bounded
    by $\|M_a\|$, it is the restriction
    of a linear functional on $L^1(\Circ)$, with norm bounded by $\|M_a\|$.
    Hence $\mathcal{L}(q) = (\psi,q)$ for some $\psi \in L^\infty(\Circ)$
    with $\|\psi\|_\infty \leq \|M_a\|$. This proves the result.    
\end{proof}

The key idea of this theorem is that it relates the sequence defining
a Hankel operator with a function on $\Circ$. We can state this in a slightly more
elegant way using the following result of Fefferman, which can be found in \cite{Garnett},
\begin{proposition}
    The space $\BMO(\Circ)$ is defined as the set of measurable
    functions $f$ on $\Circ$ (modulo almost-everywhere equivalence) such that
    \begin{equation*}
        \sup_I \int_I \left|f-\frac{1}{\ha(I)}\int_I f\;d\ha\right|\;d\ha < \infty
    \end{equation*}
    where $I$ is taken over all arcs in $\Circ$.
    
    $\BMO(\Circ)$ is a Banach space when equipped with the norm,
    \begin{equation*}
        \|f\|_{\BMO(\Circ)} = \sup_I \int_I \left|f-\frac{1}{\ha(I)}\int_I f\;d\ha\right|\;d\ha + |\hat{f}(0)|
    \end{equation*}
    
    Then 
    \begin{equation*}
        \BMO(\Circ) = L^\infty(\Circ)+\Proj_+L^\infty(\Circ) = \{f+\Proj_+g\;:\;f,g \in L^\infty(\Circ)\}.
    \end{equation*}
\end{proposition}

Using this description of $\BMO(\Circ)$, we can prove the following,
\begin{corollary}
    Let $a = \{a_j\}_{j=0}^\infty$ be a sequence. Then $M_a$ defines
    a bounded operator on $\ell^2(\Ntrl)$ if and only if
    \begin{equation*}
        \varphi := \sum_{n=0}^\infty a_n z^n \in \BMO(\Circ)\cap H^1(\Circ).
    \end{equation*}
\end{corollary}
\begin{proof}
    By theorem \ref{nehari}, $M_a$ is bounded if and only if $\varphi = \Proj_+\psi$
    for some $\psi \in L^\infty(\Circ)$. Hence if $M_a$ is bounded,
    then $\varphi \in \BMO(\Circ)\cap H^1(\Circ)$. 
    
    Conversely, if $\varphi \in \BMO(\Circ)\cap H^1(\Circ)$,
    then $\varphi = f+\Proj_+g$ for $f,g \in L^\infty$. Thus
    $\varphi = \Proj_+\varphi = \Proj_+(f+g)$, so $M_a$ is bounded.
\end{proof}

From now on, we are no longer interested in Hankel matrices $M_a$ defined by an arbitrary
sequence $a$, we are only interested in those matrices $M_a$ such that $a$
arises from the fourier transform of a function. 
\begin{definition}
    Let $\varphi \in H^1(\Circ)$. Let $\Gamma_\varphi$
    be the Hankel matrix with $(i,j)$th entry $\hat{\varphi}(i+j)$.
\end{definition}

The properties of Hankel operators, as we shall now demonstrate in part, can be summarised
in the following table: \\
\begin{center}
\begin{tabular}{|c|c|}
\hline
$\varphi$ & $\Gamma_\varphi$\\
\hline
Rational & Finite rank\\
$\VMO$ & Compact\\
$B_{pp}^{1/p}$ & $\mathcal{L}^p$\\
$\BMO$ & Bounded\\
\hline
\end{tabular}\\
\end{center}
The left hand column of the table is a list of function spaces,
and the right hand side is a list of operator spaces. This is a table
of necessary and sufficient conditions on $\varphi$ so that $\Gamma_\varphi$
lies in some operator space.

At the moment not all of the terms in the table have been defined. The Besov
classes $B_{pp}^{1/p}$ will be introduced in Definition \ref{besovDefinition},
and the class $\VMO$ is defined in Proposition \ref{feffVMO}.

We will not prove every relation implied by the above table. In particular, we will
only prove that $\varphi \in \VMO(\Circ)$ implies that $\Gamma_\varphi$ is compact,
and that $\varphi \in B_{pp}^{1/p}$ implies that $\Gamma_\varphi \in \mathcal{L}^p$.
Using these sufficient conditions, we will be able to find sufficient conditions
on $\varphi$ so that $\Gamma_\varphi$ is in $\mathcal{L}^{p,q}$ for $p \in [1,\infty]$
and $q \in (0,\infty]$. 

More detailed results are proved in the book by V.V. Peller, \cite{Peller2003},
where both the necessity and sufficiency of all the results in the above table are proved.

The presentation here is based on Peller's, but we will only prove the results
necessary for a basic introduction to quantised calculus.

\section{Finite Rank Hankel operators}
The strongest condition that we can put on a Hankel matrix $\Gamma_\varphi$
is that it is a finite rank operator on $\ell^2(\Ntrl)$. The problem
of determining $\varphi$ such that $\Gamma_\varphi$ is finite rank
was solved by Kronecker, as follows.
\begin{theorem}
\label{kronecker}
    Let $\varphi \in H^1(\Circ)$, and $\Gamma_\varphi$ be the associated Hankel
    matrix. Then $\Gamma_\varphi$ defines a bounded operator on $\ell^2(\Ntrl)$
    if and only if $\varphi$ is a rational function.
\end{theorem}
\begin{proof}
    Suppose that $\rank(\Gamma_\varphi) = n$. Then the first $n+1$
    columns of the matrix of $\Gamma_\varphi$ are linearly dependent. Let $B$
    denote the backward shift operator, $B(a_0,a_1,\ldots) := (a_1,a_2,\ldots)$
    and let $F$ be the forward shift operator, $F(a_0,a_1,\ldots) := (0,a_0,a_1,\ldots)$.
    Let $a = \hat{\varphi}$.
    
    Hence there exist complex scalars $\{c_0,c_1,\ldots,c_n\}$ not all
    equal to zero such that
    \begin{equation*}
        c_0a+c_1Ba+\cdots+c_nB^na = 0.
    \end{equation*}
    Now let $n,k \geq 0$. It is elementary that
    \begin{equation*}
        F^n B^k a = F^{n-k}a-F^{n-k}(a_0,a_1,\ldots,a_{k-1},0,0,\ldots).
    \end{equation*}
    So hence we have,
    \begin{align*}
        0 &= F^n\sum_{k=0}^n c_k B^ka \\
          &= \sum_{k=0}^n c_kF^n B^ka\\
          &= \sum_{k=0}^n c_kF^{n-k}a - p
    \end{align*}
    where $p$ is a finitely supported sequence. 
    
    Let $q = (c_n,c_{n-1},\ldots,c_0,0,0,\ldots)$. Then we have,
    \begin{equation*}
        0 = q * a - p.
    \end{equation*}
    Where the $*$ is convolution. Therefore, if we take the inverse fourier transform,
    \begin{equation*}
        \varphi\check{a} = \check{p}.
    \end{equation*}
    And hence $\varphi$ is a quotient of two polynomials.
    
    Conversely, suppose that $\varphi$ is a rational function. Suppose
    that $\varphi = p/q$, where $p,q \in P(\Circ)$. 
    Let $n = \max\{\deg p,\deg q\}$.
    If 
    \begin{equation*}
        q = \sum_{k=0}^n c_{n-k}z^k
    \end{equation*}
    then since $\varphi q = p$, we have
    \begin{equation*}
        \sum_{k=0}^n c_k F^{n-k} a = p.
    \end{equation*}
    Now multiply by $B^n$,
    \begin{align*}
        B^n\sum_{k=0}^n  c_k F^{n-k} a &= \sum_{k=0}^n c_k B^k a\\
        &= 0.
    \end{align*}
    
    Let $m \leq n$ be the largest number for which $c_m \neq 0$. Then $B^m a$
    is a linear combination of the $B^k a$ with $k \leq m-1$,
    \begin{equation*}
        B^m a = \sum_{k=0}^{m-1} d_k B^k a
    \end{equation*}
    for some coefficients $d_k$. 
    
    We now proceed by induction to show that any row is a linear combination
    of the first $n$ rows. 
    
    Let $k > m$. Then we have,
    \begin{align*}
        B^k a &= B^{k-m} B^m a\\
        &= \sum_{j=0}^{m-1} d_j B^{k-m+1}a.
    \end{align*}
    Since $k-m+j < k$, we have that the terms on the right hand side are
    linear combinations of the first $m$ rows by the inductive hypothesis.
    Hence $\rank(\Gamma_\varphi) \leq m$. 
\end{proof}

\section{Compactness of Hankel Operators}
If $\varphi \in L^1(\Circ)$, we are interested in conditions
on $\varphi$ such that $\Gamma_\varphi$ is compact. 

Our first result shows that Hankel matrices are continuous in their symbol.
\begin{proposition}
    Let $\varphi \in L^\infty(\Circ)$, then
    \begin{equation*}
        \|\Gamma_\varphi\| \leq \|\varphi\|_\infty.
    \end{equation*}
    and therefore if $\varphi \in C(\Circ)$, then $\Gamma_\varphi$ is compact.
\end{proposition}
\begin{proof}
    It was shown in the proof of theorem \ref{nehari} that if $g,f$ are sequences
    of finite support, then
    \begin{equation*}
        |(g,\Gamma_\varphi f)| \leq \|\varphi\|_\infty \|g\|_2\|f\|_2.
    \end{equation*}
    Hence $\|\Gamma_\varphi\| \leq \|\varphi\|_\infty$.
\end{proof}

\begin{corollary}
\label{ctsCmpt}
    If $\varphi \in C(\Circ)$, then $\Gamma_\varphi$ is compact.
\end{corollary}
\begin{proof}
    Since $\Gamma_\varphi$ is finite rank for $\varphi$ a polynomial,
    and $\|\Gamma_\varphi\| \leq \|\varphi\|_\infty$, the result follows.
\end{proof} 

To complete our characterisation of compact Hankel operators, we require
the following result of Fefferman, found in \cite{Garnett}:
\begin{proposition}
\label{feffVMO}
    The class $\VMO(\Circ)$ is the set of measurable functions $f:\Circ\rightarrow\Cplx$
    such that
    \begin{equation*}
        \lim_{\ha(I)\rightarrow0} \int_I \left|f-\frac{1}{\ha(I)}\int_I f\;d\ha\right|\;d\ha = 0
    \end{equation*}
    where the limit is over all arcs $I \subseteq \Circ$. 
    
    $\VMO(\Circ)$ is alternatively viewed as the closure of the trigonometric
    polynomials in $\BMO(\Circ)$.
    
    It is a result of Fefferman \cite{Garnett} that,
    \begin{equation*}
        \VMO(\Circ) = C(\Circ)+\Proj_+C(\Circ).
    \end{equation*}
\end{proposition}

So we have the following:
\begin{proposition}
    Let $\varphi \in \VMO(\Circ)$. Then $\Gamma_\varphi$ is compact.
\end{proposition}
\begin{proof}
    if $\varphi \in \VMO(\Circ)$, then by proposition \ref{feffVMO}, $\varphi = f+\Proj_+g$ for $f,g \in C(\Circ)$. Hence,
    $\Proj_+\varphi = \Proj_+(f+g)$. Hence, there exists $h \in C(\Circ)$
    with $\Gamma_{\varphi} = \Gamma_h$, simply by choosing $h = f+g$. Hence by corollary \ref{ctsCmpt}, we see that $\Gamma_\varphi$
    is compact.
\end{proof}



\section{Hankel Operators of Trace Class}
We recall the definition of the $\mathcal{L}^1$ norm on $\mathcal{B}(\Hilb)$,
\begin{definition}
    Let $\Hilb$ be a separable Hilbert space, and let $T \in \mathcal{B}(\Hilb)$. 
    For a non-negative integer $n$, we define the $n$th singular value,
    \begin{equation*}
        \mu_n(T) := \inf\{\|T-F\| \;:\;F \in \mathcal{B}(\Hilb),\rank(F) \leq n\}.
    \end{equation*}
    For $p \in (0,\infty)$, we define the $\mathcal{L}^p$ norm of $T$
    as
    \begin{equation*}
        \|T\|_p = \left(\sum_{n=0}^\infty |\mu_n(T)|^p\right)^{1/p}
    \end{equation*}
    with the convention that $\|T\|_p = \infty$ if the sum does not
    converge. The space $\mathcal{L}^p$ is the set of $T \in \mathcal{B}(\Hilb)$
    such that $\|T\|_p < \infty$.
\end{definition}



In particular we are interested in the case $p = 1$. We are interested
in finding conditions on a function $\varphi$ holomorphic in the unit disc
such that $\Gamma_\varphi$ is in $\mathcal{L}^1$. 

\begin{lemma}
    Let $\Hilb$ be a separable Hilbert space. Suppose
    that $x,y \in \Hilb$, and let $T$ be the rank one operator
    defined by $T(\zeta) = \langle x,\zeta\rangle y$. Then $\|T\|_1 = \|x\|\|y\|$.
\end{lemma}

The answer is provided by the \emph{Besov classes}. Many different definitions
of Besov spaces can be found, but the one of most relevance to us is given below.
\begin{definition}
    \label{besovDefinition}
    We define a sequence of polynomials $\{W_n\}_{n \in \Itgr} \subset P(\Circ)$ as follows.
    First,
    \begin{equation*}
        W_0 = z^{-1}+1+z.
    \end{equation*}
    And now for $n > 0$, we define $W_n$ by asserting that $\widehat{W}(2^n) = 1$,
    $\widehat{W}(2^{n-1}) = 0$, $\widehat{W}(2^{n+1}) = 0$, and $\widehat{W}$
    is a linear increasing function between $2^{n-1}$ and $2^n$, and a
    linear decreasing function between $2^n$ and $2^{n+1}$. We
    assert that $\widehat{W}(n)$ is symmetric in $n$, and is zero for all
    values not already defined.
    
    Now, for $p,q > 0$, and $s \geq 0$, we define the \emph{Besov class}
    $B_{pq}^s(\Circ)$ to be the space of distributions $f$ on $\Circ$ such that
    \begin{equation*}
        \sum_{n\geq 0} 2^{nsq} \|W_n*f\|_{p}^q < \infty.
    \end{equation*}  
    We shall denote $B_{pp}^s$ as $B_p^s$.
\end{definition}

In particular, we are going to prove that if $\varphi \in B_1^1(\Circ)$, then
$\Gamma_\varphi \in \mathcal{L}^1$. First, we need a lemma.
\begin{lemma}
    Let $f \in P_A(\Circ)$ be an analytic polynomial of degree at most $m$. Then,
    \begin{equation*}
        \|\Gamma_f\|_1 \leq (m+1)\|f\|_1.
    \end{equation*}
\end{lemma}
\begin{proof}
    Let $\zeta \in \Circ$, now define the following elements
    of $\ell^2(\Ntrl)$, 
    \begin{align*}
        x_\zeta(j) &= \begin{cases}
            \zeta^j,\;0 \leq j \leq m,\\
            0,\;j > m
        \end{cases}\\
        y_\zeta(j) &= \begin{cases}
            f(\zeta)\zeta^{-k},\;0\leq k \leq m,\\
            0,\;k > m.
        \end{cases}
    \end{align*}
    That is, $x_\zeta = (1,\zeta,\zeta^2,\ldots,\zeta^m,0,0,\ldots)$
    and $y_\zeta = f(\zeta)\overline{x_\zeta}$.
    
    Let $A_\zeta$ be the rank one operator, $A_\zeta(x) = (x_\zeta,x)y_\zeta$,
    so that $\|A_\zeta\|_1 = \|x_\zeta\|_2\|y_\zeta\|2 = (m+1)|f(\zeta)|$
    
    Then we have a componentwise equality of infinite matrices,
    \begin{equation*}
        \Gamma_f = \int_\Circ A_\zeta\;d\ha(\zeta).
    \end{equation*}
    Hence, $\|\Gamma_f\|_1 \leq (m+1)\|f\|_1$ by the triangle inequality.
\end{proof}
\begin{theorem}
    Let $\varphi \in B_1^1$. Then $\Gamma_\varphi \in \mathcal{L}^1$. 
\end{theorem}
\begin{proof}
    We have the following $L^\infty$-convergent sequence,
    \begin{equation*}
        \varphi = \sum_{n\geq 0} W_n*\varphi.
    \end{equation*}
    Hence,
    \begin{equation*}
        \Gamma_\varphi = \sum_{n\geq 0} \Gamma_{W_n*\varphi}.
    \end{equation*}
    So since the degree of $W_n$ is $2^{n+1}$, we have
    \begin{equation*}
        \|\Gamma_\varphi\|_1 \leq \sum_{n\geq 0} 2^{n+1}\|W_n*\varphi\|_1.
    \end{equation*}
\end{proof}

This proves the sufficiency of the condition $\varphi \in B_1^1$ so that $\Gamma_\varphi$
is trace class. The proof of the necessity of this condition is more difficult,
\begin{theorem}
    Let $\varphi$ be a function holomorphic in the unit disc. Then
    if $\Gamma_\varphi \in \mathcal{L}^1$, then $\varphi \in B_1^1$. 
\end{theorem}
\begin{proof}
    Define a pair of sequences of polynomials $\{Q_n\}_{n=0}^\infty$ as follows,
    \begin{equation*}
        \widehat{Q}_n(k) = \begin{cases}
            0, \;k \leq 2^{n-1}\\
            1-\frac{|k-2^n|}{2^{n-1}},\;2^{n-1}\leq k \leq 2^n+2^{n-1},\\
            0\;k \geq 2^n+2^{n-1}.
        \end{cases}
    \end{equation*}
    and a sequence $\{R_n\}_{n=0}^\infty$,
    \begin{equation*}
        \widehat{R}_n(k) = \begin{cases}
                0,\;k \leq 2^n\\
                1-\frac{|k-2^n-2^{n-1}|}{2^{n-1}},\;2^{n}\leq 2^{n+1}\\
                0,\;k \geq 2^{n+1}.
        \end{cases}
    \end{equation*}
    This is a decomposition of the sequence $\{W_n\}_{n=0}^\infty$, 
    given by $W_n = Q_n+\frac{1}{2}R_n$. 
    
    First we prove that
    \begin{equation*}
        \sum_{n\geq 0} 2^{2n+1}\|Q_{2n+1}*\varphi\|_1 < \infty.
    \end{equation*}
    
    To this end, we wish to construct an operator $B$
    such that
    \begin{equation*}
        \langle \Gamma_\varphi,B\rangle = \sum_{n\geq 0} 2^{2n} \|Q_{2n+1}*\varphi\|_1.
    \end{equation*}
    
    Now define the sequence of squares,
    for $n\geq 1$,
    \begin{equation*}
        S_n = [2^{2n-1},2^{2n-1}+2^{2n}-1]\times [2^{2n-1}+1,2^{2n-1}+2^{2n}].
    \end{equation*}
    Note that this sequence is pairwise disjoint.
    
    Let $\{\psi_n\}_{n=0}^\infty$ be a sequence in $L^\infty(\Circ)$, yet to be
    defined with $\|\psi_n\|_{\infty} \leq 1$. Now we
    define the matrix $B = \{B_{j,k}\}_{j,k\geq 0}$ by
    \begin{equation*}
        B_{j,k} = \begin{cases}
            \widehat{\psi}_n(j+k),(j,k) \in S_n,n\geq 1,\\
            0,(j,k) \notin \bigcup_{n\geq 1} S_n.
        \end{cases}
    \end{equation*}
    We wish to prove that $B$ is bounded, and in fact $\|B\|\leq 1$. 
    Let $\{e_n\}_{n\geq 0}$ be the standard basis for $\ell^2(\Ntrl)$,
    with $e_n(m) = \delta_{n,m}$. Define the subspaces
    \begin{align*}
        \Hilb_n &= \span\{e_j\;:\;2^{2n-1}\leq j \leq 2^{2n-1}+2^{2n}-1\},\\
        \Hilb_n' &= \span\{e_j\;:\;2^{2n-1}+1\leq j\leq 2^{2n-1}+2^{2n}\}.
    \end{align*}
    Let $P_n$ and $P_n'$ be the orthogonal projection onto $\Hilb_n$ and $\Hilb_n'$
    respectively.
    So that
    \begin{equation*}
        B = \sum_{n\geq 1} P_n'\Gamma_{\psi_n}P_n,
    \end{equation*}
    where $P_n$ and $P_n'$ are the orthogonal projections onto $\Hilb_n$
    and $\Hilb_n'$ respectively.
    
    Now since the spaces $\{\Hilb_n\}_{n\geq 1}$ are pairwise orthogonal,
    as are the spaces $\{\Hilb_n'\}_{n\geq 1}$, we have
    \begin{align*}
        \|B\| &\leq \sup_{n\geq 1} \|P_n'\Gamma_{\psi_n}P_n\|\\
         &\leq \sup_{n} \|\Gamma_{\psi_n}\|\\
         &\leq \sup_{n} \|\psi_n\|_\infty \\
         &\leq 1.
    \end{align*}
    
    Now, we compute
    \begin{align*}
        \langle \Gamma_\varphi,B \rangle &= \sum_{n\geq 1} \langle \Gamma_\varphi,P_n'\Gamma_{\psi_n}P_n\rangle\\
        &= \sum_{n\geq 1} \sum_{j=2^{2n}}^{2^{2n}+2^{2n+1}}(2^{2n}-|j-2^{2n+1}|\overline{\widehat{\varphi}}(j)\widehat{\psi_n}(j)\\
        &= \sum_{n\geq 1} 2^{2n} (Q_{2n+1}*\varphi,\psi_n).
    \end{align*}
    Now, using the sharpness of H\"older's inequality, we can choose
    a sequence $\{\psi_n\}_{n=0}^\infty$ so that $\langle Q_{2n+1}*\varphi,\psi_n\rangle$
    is arbitrarily close to $\|Q_{2n+1}*\varphi\|_1$. Hence,
    \begin{equation*}
        \sum_{n\geq 1}2^{2n+1}\|Q_{2n+1}*\varphi\|_1 = 2\langle \Gamma_\varphi,B\rangle \leq 2\|\Gamma_\varphi\|_1.
    \end{equation*}
    In exactly the same way, we may prove that
    \begin{equation*}
        \sum_{n\geq 1}2^{2n}\|Q_{2n}*\varphi\|_1 < \infty
    \end{equation*} 
    that
    \begin{equation*}
        \sum_{n\geq 0} 2^{2n+1} \|R_{2n+1}*\varphi\|_1 < \infty
    \end{equation*}
    and
    \begin{equation*}
        \sum_{n\geq 1} 2^{2n}\|R_{2n}*\varphi\|_1 < \infty
    \end{equation*}
    and therefore that $\varphi \in B_1^1$.
\end{proof} 

\begin{remark}
    We thus conclude that
    \begin{equation*}
        \frac{1}{6}\sum_{n\geq 1}2^n \|W_n*\varphi\|_1 \leq \|\Gamma_\varphi\|_1 \leq 2\sum_{n\geq 0} 2^n\|W_n*\varphi\|_1.
    \end{equation*}
\end{remark}

\section{Interpolation}
For this section, $\mathcal{L}^\infty$ denotes the space of compact operators.

Up to now we have proved necessary and sufficient conditions
on $\varphi$ so that $\Gamma_\varphi$ lies in a certain ideal one at a time.
In particular, we have with great effort proved that $\Gamma_\varphi \in \mathcal{L}^1$
if and only if $\varphi \in B_{11}^1(\Circ)$.
This amounts to a continuous
injection:
\begin{equation}
\label{emb1}
    B_{11}^1(\Circ) \to \mathcal{L}^1
\end{equation}
with equivalence of norms.
We denote $\Gamma\mathcal{L}^\infty$
as the class of compact Hankel operators.

We also know that we have an injection
\begin{equation}
\label{emb2}
    \VMO(\Circ) \to \mathcal{L}^\infty.
\end{equation}


It is a remarkable fact that using the embeddings \ref{emb1}
and \ref{emb2} alone we can find sufficient conditions on $\varphi$
so that $\Gamma_\varphi \in \mathcal{L}^{p,q}$ for $p\in [1,\infty]$,$q \in (0,\infty]$.

We accomplish this via interpolation theory.
The basic material is covered in Appendix \ref{Interpolation}. The pair $(B_{11}^1(\Circ),\VMO(\Circ))$.
The map $\varphi \mapsto \Gamma_\varphi$ takes the pair
$(B_{11}^1(\Circ),\VMO(\Circ))$ to $(\mathcal{L}^1,\mathcal{L}^\infty)$.
Hence for any interpolation functor $F:\NLS_1\to\NLS$ (using the notation
of Appendix \ref{Interpolation}, we have that if $\varphi \in F(B_{11}^1(\Circ),\VMO(\Circ))$,
then $\Gamma_{\varphi} \in F(\mathcal{L}^1,\mathcal{L}^\infty)$.

In order to therefore find sufficient conditions on $\varphi$ so that $\Gamma_\varphi \in \mathcal{L}^{p,q}$,
we simply need to find an interpolation functor $F_{p,q}$ such that $F_{p,q}(\mathcal{L}^1,\mathcal{L}^\infty) = \mathcal{L}^{p,q}$.

It turns out that there is indeed such a functor, the $K$ functor from real interpolation theory.
For a definition see Appendix \ref{Interpolation}. We can choose $F_{p,q} = K(\cdot,\cdot)_{\theta,q}$
where $\theta = 1-p^{-1}$. This is due to the following result:
\begin{proposition}
    Let $p \in (0,\infty)$ and $q \in (0,\infty]$. If $\theta = 1-p^{-1}$, then $K(\mathcal{L}^1,\mathcal{L}^\infty)_{\theta,q} = \mathcal{L}^{p,q}$.
\end{proposition}
\begin{proof}
    The details of this interpolation result are included by in \cite{DDP92}.
    
    The key estimate is as follows.
    Let $T \in \mathcal{L}^1+\mathcal{L}^\infty = \mathcal{L}^\infty$.
    Using a result from G.E. Karadzhov \cite{Karadzhov},
    the $K$ functional, $K(t,T,\mathcal{L}^1,\mathcal{L}^\infty)$, given in
    definition \ref{Kdefinition}, can be estimated by
    \begin{equation}
        c_1\left(\sum_{j=0}^n \mu_j(T)^p\right)^{1/p} \leq K(t,T,\mathcal{L}^1,\mathcal{L}^\infty) \leq c_2\left(\sum_{j=0}^n \mu_j(T)^p\right)^{1/p},
    \end{equation}
    for $n^{1/p} \leq t \leq (n+1)^{1/p}$, for some positive constants $c_1,c_2$. This
    estimate can be found in \cite{Karadzhov}.
\end{proof}


%The following is proved in \cite{DDP92}:
%\begin{lemma}
%    Let $K$ be the real interpolation functor, described in \cite{interp}. 
%    Let $p,q \in (0,\infty]$
%    Then
%    \begin{equation*}
%        K(\mathcal{L}^1,\mathcal{K})_{\theta,q} = \mathcal{L}_{p,q}.
%    \end{equation*} 
%    for $p = (1-\theta)^{-1}$.
%\end{lemma}


%\begin{corollary}
%    Hence, we have that $\Gamma_\varphi \in \mathcal{L}_{p,q}$
%    if $\varphi \in K(B_1^1,\VMO)_{\theta,q}$, where $p = (1-\theta)^{-1}$.
%\end{corollary}

So we have shown that if $\varphi \in K(B_{11}^1(\Circ),\VMO(\Circ))_{1-1/p,q}$,
then $\Gamma_\varphi \in \mathcal{L}^{p,q}$. Peller \cite[Theorem 4.4, p. 256]{Peller2003} shows
that this condition is not just sufficient but also necessary. Of particular interest
is the case $p = q$. This is solved by the following result, which is \cite[Theorem 4.3, p.255]{Peller2003}:
\begin{proposition}
    Let $p \in (1,\infty)$. Then $K(B_{11}^1(\Circ),\VMO(\Circ)) = B_{pp}^{1/p}(\Circ)$. 
\end{proposition}

\section{Extrapolation}
We have determined sufficient conditions for $\Gamma_\varphi \in \mathcal{L}^{p,q}$,
for $p \in [1,\infty]$ and $q \in (0,\infty]$.
We now turn our attention to more exotic ideals of $\mathcal{B}(\ell^2(\Ntrl))$. The
Macaev-Dixmier ideal $\M_{1,\infty}$ is described in detail Appendix \ref{IdealsOfOperators}.

Most important for our purposes is the equivalence of norms, proved in \cite{CRSS},
\begin{equation}
    \|x\|_{\M_{1,\infty}} = \sup_{s \in (0,1)} s\|x\|_{\mathcal{L}_{s+1}}.
\end{equation}

First we need the following result, proved in the proof of \cite[Theorem 3.1,p.250]{Peller2003}.
\begin{proposition}
    Let $\varphi \in H^1(\Circ)$, and let $p \in (1,2)$. Then there is a constant $C$
    not depending on $p$ such that
    \begin{equation}
        \|\Gamma_{\varphi}\|_{\mathcal{L}^p}^p \leq 2^{1-p} \sum_{n\geq 0} 2^n\|W_n*\varphi\|_p^p
    \end{equation}
    Where the right hand side is defined to be $\|\varphi\|_{B_{pp}^{1/p}}$.
\end{proposition}

Hence, we have
\begin{proposition}
    Let $\varphi$ be a function on $\Circ$ such that $\sup_{s \in (0,1)} s\|\varphi\|_{B_{s+1,s+1}^{1/(s+1)}} < \infty$.
    Then $\qd \varphi \in \M_{1,\infty}$.
\end{proposition}





