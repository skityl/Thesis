% Chapter 1

\chapter{Quantised Differentials on $\Rl$ and $\Circ$} % Main chapter title

\label{QuantisedDifferentialsOnRandT} % For referencing the chapter elsewhere, use \ref{Chapter1} 

\lhead{Chapter \ref{QuantisedDifferentialsOnRandT}. \emph{Quantised Differentials on $\Rl$ and $\Circ$}} % This is for the header on each page - perhaps a shortened title

%----------------------------------------------------------------------------------------

\section{The Quantised Differential on $\Circ$}
We shall introduce a Fourier analytic description of the quantised differential
of a function on the circle. An introduction to elementary Fourier analysis and
notation is given in Appendix \ref{AppendixA}.

\begin{proposition}
    For $\varphi \in L^1(\Circ)$, the \emph{quantised differential} of $\varphi$ is the (potentially unbounded, densely defined) linear operator
    \begin{equation*}
        \qd\varphi := 2[\Proj_+,M_\varphi].
    \end{equation*}
\end{proposition}
\begin{proof}
    This is a consequence of the representation $F = 2\Proj_+-\Id$,
    where $\Id$ is the identity operator.
    
    We now prove that $F = 2\Proj_+-\Id$. Since $F$ is the image
    of a self adjoint operator under a bounded function, $F$ is bounded,
    and $2\Proj_+-\Id$ is bounded. Hence $F$ and $2\Proj_+-\Id$
    are equal if and only if they are equal on the orthonormal basis $\{z^n\}_{n\in\Itgr}$. 
    
    For $n \in \Itgr$, $z^n$ is an eigenfunction of $\D$, with eigenvalue
    $n$, so $F(z^n) = \sgn(n)z^n$. Since $(2\Proj_+-\Id)(z^n) = \sgn(n)z^n$,
    the equality is proved.
\end{proof}
In this chapter, we will discuss alternative descriptions of $\qd \varphi$.

Since $H^2(\Circ)$
is a closed subspace of $L^2(\Circ)$, being the image of $L^2(\Circ)$
under a bounded projection, it has an orthogonal complement $H^2_-(\Circ)$.

Hence, we may consider the quantised derivative $\qd\varphi$ as an operator on
the Hilbert space $H^2(\Circ)\oplus H^2_-(\Circ)$.

\begin{lemma}
    Let $\varphi \in L^2(\Circ)$ and define $\varphi_+ := \Proj_+ \varphi$ and $\varphi_- := \Proj_- \varphi$. Then $\qd\varphi:H^2(\Circ)\oplus H^2_-(\Circ)\rightarrow H^2(\Circ)\oplus H^2_-(\Circ)$ may be written as
    \begin{equation*}
        \qd\varphi(f\oplus g) = 
            2(\Proj_+M_{\varphi_+})g \oplus
            -2(\Proj_-M_{\varphi_-})f
    \end{equation*}
    for $f \in H^2(\Circ)$ and $g \in H^2_-(\Circ)$.
\end{lemma}
\begin{proof}
    This is a simple computation. Let $f \in H^2(\Circ)$ and $g \in H^2_-(\Circ)$. Then,
    \begin{equation*}
        \qd\varphi(f+g) = 2[\Proj_+,M_{\varphi_+}+M_{\varphi_-}](f+g).
    \end{equation*}
    Hence,
    \begin{align*}
        \qd\varphi(f+g) &= 2[\Proj_+,M_{\varphi_+}]f + 2[\Proj_+,M_{\varphi_-}]f + 2[\Proj_+,M_{\varphi_+}]g + 2[\Proj_+,M_{\varphi_-}]g\\
        &= 2(\Proj_+M_{\varphi_-})f+2(\Proj_+M_{\varphi_+})g-2M_{\varphi_-}f
    \end{align*}
    since $\Proj_+f = f$ and $\Proj_+g = 0$.
    
    By the identity $\Proj_+ = \Id-\Proj_-$, we find
    \begin{equation*}
        \qd\varphi(f+g) = 2(\Proj_+M_{\varphi_+})g - 2(\Proj_-M_{\varphi_-})f.
    \end{equation*}
\end{proof}


The problem of determining the boundedness of $\qd\varphi$ is then reduced to the problem of determining
the boundedness of operators of the form $\Proj_+M_\psi:H^2_-(\Circ)\rightarrow H^2(\Circ)$ and $\Proj_-M_\psi:H^2(\Circ)\rightarrow H^2_-(\Circ)$ for $\psi \in L^2(\Circ)$. We may simplify
this further with the following lemma:
\begin{lemma}
    Let $\psi \in L^2(\Circ)$. Then
    \begin{equation*}
        (\Proj_+M_\psi)^* = \Proj_-M_{\overline{\psi}}.
    \end{equation*}
    and therefore $\Proj_+M_\psi$ is bounded if and only if $\Proj_-M_{\overline{\psi}}$ is.
\end{lemma}
\begin{proof}
    Let $e_k(z) = z^k$. 

    This is again a simple computation. Let $m,n\in \Intgr$ with $m \geq 0$ and $n < 0$. Then,
    \begin{align*}
        \langle (\Proj_+M_\psi)e_n,e_m\rangle &= \int_{\Circ}\sum_{k>-n} \hat{\psi}(k)\zeta^{k+n-m}\;d\ha(\zeta)\\
        &= \hat{\psi}(m-n).
    \end{align*}
    
    Similarly,
    \begin{align*}
        \langle e_n, (\Proj_-M_{\overline{\psi}})e_m \rangle &= \int_{\Circ} \sum_{k > m} \hat{\varphi}(k) \zeta^{n-m+k}\;d\ha(\zeta)\\
                                                &= \hat{\psi}(m-n).
    \end{align*}
    
    Hence, $(\Proj_+M_\psi)^* = \Proj_-M_{\overline{\psi}}$.

\end{proof}

For $\psi \in L^2(\Circ)$, define
\begin{equation*}
    H_\psi := \Proj_-M_\psi:H^2\rightarrow H^2_-.
\end{equation*}


In other words, we may write $\qd \varphi:H^2(\Circ)\oplus H^2_-(\Circ)\to H^2(\Circ)\oplus H^2_-(\Circ)$
as a matrix,
\begin{equation}
    \qd \varphi = 2\begin{pmatrix}
        0 & -H_{\overline{\varphi_+}}^*\\
        H_{\varphi_-}  & 0
    \end{pmatrix}
\end{equation}

Therefore, we only need to study operators of the form $H_\psi$.

We study these operators using the Fourier transform. Use the standard basis $\{z^n\}_{n\geq 0}$
on $H^2(\Circ)$ and the standard basis with negative indices $\{z^{-n}\}_{n \geq 0}$ on $H^2(\Circ)$.

Let $\psi \in L^2(\Circ)$. Then in the bases above, $H_\psi$ has matrix representation
with $(n,k)$th entry $\hat{\psi}(-n-k)$. 

This means that $H_\psi$ is represented by a \emph{Hankel matrix}. So we require
results on Hankel matrices. This is covered in Chapter \ref{PropertiesOfHankelOperators}.

\section{Integral representation of the quantised differential on $\Circ$}
The following theorem allows us to describe the quantised differential as an integral operator.
\begin{lemma}
\label{singularIntegral}
    \begin{equation*}
        \mathrm{p.v.}\int_\Circ \frac{1}{\tau-1}\;d\ha(\tau) = -\frac{1}{2}
    \end{equation*}
    where the principal value is defined to be
    \begin{equation*}
        \lim_{\varepsilon\rightarrow 0} \int_{|\tau-1|>\varepsilon} \frac{1}{\tau-1}\;d\ha(\tau).
    \end{equation*}
\end{lemma} 
\begin{proof}
    Note that
    \begin{equation*}
        \pvint_\Circ \frac{1}{\tau-1}\;d\ha(\tau) = \pvint_{\im(\tau) > 0} \frac{1}{\overline{\tau}-1}+\frac{1}{\tau-1}\;d\ha(\tau).
    \end{equation*}
    Hence,
    \begin{equation*}
        \pvint_\Circ \frac{1}{\tau-1}\;d\ha(\tau) = \pvint_{\operatorname{Im}(\tau)>0} 2\operatorname{Re}\left(\frac{1}{\tau-1}\right)\;d\ha(\tau).
    \end{equation*}
    However, if $\tau = \exp(i\theta) \neq 1$, then
    \begin{align*}
        \operatorname{Re}\left(\frac{1}{\tau-1}\right) &= \operatorname{Re}\left(\frac{e^{-i\theta/2}}{2i\sin(\theta/2)}\right)\\
        &= -\frac{1}{2}.
    \end{align*}
    Hence,
    \begin{equation*}
        \pvint_\Circ \frac{1}{\tau-1}\;d\ha(\tau) = 2\pvint_{\operatorname{Im}(\tau)>0} -\frac{1}{2}\;d\ha(\tau) = -\frac{1}{2}.
    \end{equation*}
\end{proof}
\begin{theorem}
    Let $\varphi \in L^2(\Circ)$. Then
    \begin{equation*}
        \Proj_+\varphi(z) = \mathrm{p.v.}\int_\Circ \frac{\varphi(\tau)}{1-\overline{\tau}z}\;d\ha(\tau)+\frac{1}{2}\varphi(z)
    \end{equation*}
    and hence,
    \begin{equation*}
        F\varphi(z) = 2\mathrm{p.v.}\int_\Circ \frac{\varphi(\tau)}{1-\overline{\tau}z}\;d\ha(\tau).
    \end{equation*}
    where in both equations, the principal value means that the integral is to be taken along the set $\{ \zeta\;:|\zeta-z| > \varepsilon\}$
    and then consider the limit $\varepsilon\rightarrow 0$.
\end{theorem}
\begin{proof}
    It is sufficient to check this on the basis elements $e_n(z) = z^n$ for $n \in \Intgr$.
    
    First let $n \geq 0$. Then
    \begin{equation*}
        \mathrm{p.v.}\int_\Circ \frac{\tau^n}{1-\overline{\tau}z}\;d\ha(\tau) = \mathrm{p.v.}\int_\Circ \frac{z^n\tau^n}{1-\overline{\tau}}\;d\ha(\tau)
    \end{equation*}
    by translation invariance.
    Hence,
    \begin{align*}
        \mathrm{p.v.}\int_{\Circ} \frac{\tau^n}{1-\overline{\tau}z}\;d\ha(\tau) &= z^n\mathrm{p.v.}\int_\Circ \frac{\tau^{n+1}}{\tau-1}\;d\ha(\tau) \\
        &= z^n \mathrm{p.v.}\int_\Circ \frac{\tau^{n+1}-1}{\tau-1}+\frac{1}{\tau-1}\;d\ha(\tau)\\
        &= z^n \mathrm{p.v.}\int_\Circ 1+\tau+\tau^2+\cdots+\tau^{n}\;d\ha(\tau)+z^n\mathrm{p.v.}\int_\Circ \frac{1}{\tau-1}\;d\ha(\tau)\\
        &= z^n + z^n\mathrm{p.v.}\int_\Circ \frac{1}{\tau-1}\;d\ha(\tau)\\
        &= \frac{1}{2}z^n
    \end{align*}
    where the last step follows from lemma \ref{singularIntegral}.
    
    Suppose $n > 0$, then
    \begin{equation*}
        \mathrm{p.v.}\int_\Circ \frac{\tau^{-n}}{1-\overline{\tau}z}\;d\ha(\tau) = z^{-n} \mathrm{p.v.}\int_{\Circ} \frac{\tau{1-n}}{\tau-1}\;d\ha(\tau)
    \end{equation*}
    by translation invariance. Hence,
    \begin{align*}
        \mathrm{p.v.}\int_\Circ \frac{\tau^{-n}}{1-\overline{\tau}z}\;d\ha(\tau) &= z^{-n} \mathrm{p.v.} \int_\Circ \frac{1}{\tau^n-\tau^{n-1}}\;d\ha(\tau)\\
        &= z^{-n}\overline{\mathrm{p.v.}\int_\Circ \frac{\tau^n}{1-\tau}}\\
        &= -\frac{1}{2}z^{-n}.
    \end{align*}
    
    Hence, 
    \begin{equation*}
        \pvint_\Circ \frac{\tau^n}{1-\overline{\tau}z}\;d\ha(\tau) = \begin{cases}
            \frac{1}{2}z^n\text{ if }n \geq 0\\
            -\frac{1}{2}z^n\text{ if }n < 0.
        \end{cases}
    \end{equation*}
    
    So the result follows.
    
\end{proof}

So we have the following integral form of the quantised derivative. Let $\varphi,f \in L^2(\Circ)$.
Then
\begin{equation*}
    \qd\varphi(f)(z) = ([F,M_\varphi]f)(z) = 2\pvint_\Circ \frac{\varphi(z)-\varphi(\tau)}{1-\overline{\tau}z}f(\tau)\;d\ha(\tau).
\end{equation*}

\section{The Cayley Transform}

\subsection{Notation}
$\Half = \{z \in \Cplx\;:\;\Im(z) > 0\}$ denotes the upper half plane,
and $\Disc = \{z\in \Cplx\;:\; |z| < 1\}$ denotes the open unit ball.
$\Circ = \{z \in \Cplx\;:\; |z| = 1\}$.

We use normalised Haar measure on $\Circ$, denoted $\ha$. Lebesgue
measure on $\Rl$ is denoted $\lambda$, and two dimensional Lebesgue measure on
$\Cplx$ is denoted $\ha_2$.

Throughout these notes, $\omega$ denotes the \emph{Cayley transform}.
$\omega:\Cplx\setminus\{1\}\rightarrow \Cplx$, and 
\begin{equation*}
    \omega(\zeta) = i\frac{1+\zeta}{1-\zeta},\;\zeta \in \Disc.
\end{equation*}

For a Banach space $E$, and a measure space $(X,\Sigma,\mu)$, we define
\begin{equation*}
    \|f\|_{L^p(X;E)} = \left(\int_X \|f\|_E^p \;d\mu\right)^{1/p}
\end{equation*}
for $p \in (0,\infty)$, and
\begin{equation*}
    \|f\|_{L^\infty(X;E)} = \sup_{x \in X} \|f(x)\|_E
\end{equation*}
for a weakly measurable $f:X\rightarrow E$. We define $L^p(X;E)$ as the set
of measurable $f:X\rightarrow E$ with $\|f\|_{L^p(X;E)} < \infty$. As usual, 
we identify together functions on a measure space $(X,\Sigma,\mu)$ 
which agree $\mu$-almost everywhere.

$L^0(X;E)$ denotes the set of all ($\mu$-almost everywhere equivalence classes of)
weakly measurable functions from $X$ to $E$.

When $X$ is a set with counting measure, we denote $L^p(X;E)$ as $\ell^p(X;E)$.

Suppose $\zeta \in \Circ$. Provided that $\zeta \neq 1$, we see that $\omega(\zeta)$
is defined, and $\omega$ maps $\Circ\setminus\{1\}$ smoothly to $\Rl$. Thus for
$f \in L^0(\Rl;E)$, we can define $\tilde{f} \in L^0(\Circ;E)$
by 
\begin{equation*}
    \tilde{f} := f\circ \omega^{-1}.
\end{equation*}

Thus we can define the important operator $U:L^0(\Circ;E)\rightarrow L^0(\Rl;E)$,
\begin{equation*}
    (U f)(x) = \frac{1}{\sqrt{\pi}}\frac{(f\circ \omega^{-1})(x)}{x+i}.
\end{equation*}

It is the purpose of these notes to collate various results concerning the images
of certain subspaces of $L^0(\Circ;E)$ under $U$.

