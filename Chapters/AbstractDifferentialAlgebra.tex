% Chapter 3

\chapter{Abstract Differential Algebra and Spectral Triples} % Main chapter title

\label{AbstractDifferentialAlgebra} % For referencing the chapter elsewhere, use \ref{Chapter1} 

\lhead{Chapter \ref{AbstractDifferentialAlgebra}. \emph{Abstract Differential Algebra and Spectral Triples}} % This is for the header on each page - perhaps a shortened title

%----------------------------------------------------------------------------------------


\section{Introduction}
We now return to the discussion of the origin of quantised calculus. Non-commutative
geometry, as described in Chapter \ref{Introduction}, is envisaged
as the study of pairs $(\A,\Hilb)$, where $\A$ is an algebra
of operators on a Hilbert space $\Hilb$.

A classical example is $\Hilb_1 = L^2([0,1])$, and $\A_1 = L^\infty([0,1])$,
encoded as pointwise multiplication operators. 
Since we have an isomorphism between $\Hilb_1$ and $\Hilb_2 = L^2(\Circ)$, and
this induces an algebra isomorphism between $\A_1$ and $\A_2 = L^\infty(\Circ)$.

Therefore at the level of operator algebras, the pair $(\A_1,\Hilb_1)$ is indistinguishable
from $(\A_2,\Hilb_2)$. However, topologically, $[0,1]$ is very distinct
from $\Circ$. Therefore, to encode the ``geometry" of a non-commutative
space $(\A,\Hilb)$, we require more information. To attempt to find
an solution to this problem, we shall explore how geometry is encoded
in algebra in differential geometry.

\section{Classical Differential Algebra}
Let $M$ be an $n$ dimensional manifold. For
an encyclopaedic introduction to differential
geometry, see the text \cite{diffGeom}. The cotangent bundle $\Omega^1(M)$
is a rank $n$ vector bundle on $M$. We build higher bundles by wedge products,
\begin{equation*}
    \Omega^p(M) := \bigwedge_p \Omega^1(M).
\end{equation*}
and define $\Omega^0(M) := C^\infty(M)$. See that $\Omega^p(M)$ is
a rank $\binom{n}{p}$ vector bundle.


The exterior algebra bundle is the direct sum of all the $\Omega^p(M)$,
\begin{equation*}
    \Omega(M) := \bigoplus_{p=0}^\infty \Omega^p(M).
\end{equation*}


$\Omega(M)$ is a \emph{graded algebra}.


In general, if $A$ is an algebra over a ring $R$, we say that $A$ is $\Ntrl$-graded
if there exists a decomposition into submodules $A^{(p)}$,
\begin{equation*}
    A = \bigoplus_{p=0}^\infty A^{(p)}
\end{equation*}
such that $A^{(n)}A^{(m)} \subseteq A^{(n+m)}$.

In the case $A = \Omega(M)$, we have $R = \Rl$, and $A^{(p)} = \Omega^p(M)$.

The exterior derivative, $d:\Omega(M)\rightarrow \Omega(M)$ acts on the grading by,
\begin{equation*}
    d:\Omega^p(M) \rightarrow \Omega^{p+1}(M).
\end{equation*}
and $d^2 = 0$. Hence we have a sequence,

\begin{equation*}
    0\rightarrow \Omega^{0}(M) \xrightarrow{d} \Omega^1(M) \xrightarrow{d} \cdots \xrightarrow{d} \Omega^n(M) \xrightarrow{d} 0.
\end{equation*}
Denote $d_p$ as the restriction of $d$ to $\Omega^p(M)$. Then we have the \emph{de Rham cohomology}
spaces,
\begin{equation*}
    H^p_{dR}(M) = \frac{\ker(d_{p})}{\im(d_{p-1})}
\end{equation*}
This is a sequence of real vector spaces, and their dimensions, called
Betti numbers, are well known to be topological
invariants of $M$.

The maps $d_p$ satisfy a graded version of Leibniz's rule, for $a \in \Omega^n(M)$
and $b \in \Omega^m(M)$, we have:
\begin{equation*}
    d_{n+m}(ab) = d_n(a)b+(-1)^nad_m(b)
\end{equation*}

\section{Abstract Differential Algebra}
\subsection{Graded Differential Algebras}
We now take the ideas of the previous section and move them to a more abstract setting.
Let $R$ be a commutative ring, and let $A$ be an $\Ntrl$-graded algebra over $R$, with decomposition
\begin{equation*}
    A = \bigoplus_{p=0}^\infty A^{(p)}
\end{equation*}

There is also an $R$-linear map $d:A\rightarrow A$ such that,
\begin{equation*}
    d:A^{(p)}\rightarrow A^{(p+1)}.
\end{equation*}
and $d^2 = 0$.
If we denote the restriction of $d$ to $A^{(p)}$ as $d_p$, we
require that the maps $d_p$ satisfy a graded Leibniz rule, for $a \in A^{(n)}$
and $b \in A^{(m)}$,
\begin{equation*}
    d_{n+m}(ab) = d_n(a)b+(-1)^nad_m(b).
\end{equation*}

A pair $(A,d)$ satisfying these conditions is called a \emph{differential graded algebra}.

Thus we have a sequence,
\begin{equation*}
    0\rightarrow A^{(0)} \xrightarrow{d} A^{(1)} \xrightarrow{d} \cdots \xrightarrow{d} A^{(n)} \xrightarrow{d} \cdots
\end{equation*}
The quotient $R$-modules,
\begin{equation*}
    H^p_{dR}(M) := \frac{\ker(d_{p})}{\im(d_{p-1})}.
\end{equation*}
are the de Rham cohomology modules for the graded differential algebra $(A,d)$.
\subsection{K\"ahler Differentials}
Given an $R$-algebra $A$, we would like to be able to build an algebra
of differential forms over $A$, in a manner analogous to how $\Omega^1(M)$
is constructed from $C^\infty(M)$. It turns out that there is a good way of doing
this, called the algebra of \emph{K\"ahler differentials}. This is simplest in the commutative case,
which we briefly outline here.

Let $R$ be a commutative ring, and let $A$ be a unital commutative $R$-algebra. The module
$\Omega^1_{\com}(A)$
of K\"ahler differentials is defined as
\begin{equation*}
    \Omega^1_{\com}(A) := \frac{A\otimes_R A}{\langle c\otimes(ab)-(ca)\otimes b-(bc)\otimes a\rangle}
\end{equation*}
The idea here is that $\Omega^1_{\com}(A)$ is the left $A$-module spanned by all
symbols of the form $adb$, where $d(ab) = adb+bda$. We think of $a\otimes b$
as $adb$.

More precisely, we let $d:A\rightarrow \Omega^1_{\com}(A)$ be given by
\begin{equation*}
    da := 1_A\otimes a
\end{equation*}
Where $1_A$ is the unit in $A$.

The utility of $\Omega^1_{\com}(A)$ is that it allows us to study all
derivations on $A$. 

In full abstraction, a derivation on $A$ is a map $\theta:A\rightarrow M$, where
$M$ is some left $A$-module, such that $\theta$ satisfies the Leibniz rule,
\begin{equation*}
    \theta(ab) = a\theta(b)+b\theta(a).
\end{equation*}

We see that $d$ is a derivation on $A$ to the $A$-module $\Omega^1_{\com}(A)$.
It is in fact universal with this property,
\begin{theorem}
    Let $A$ be a unital commutative $R$-algebra, and let $\theta:A\rightarrow M$
    be a derivation to some left $A$-module $M$. There exists a unique $R$-linear 
    map $\Omega(\theta)$ such that the following diagram commutes,\\
    \begin{displaymath}
    \xymatrix{
        A \ar[r]^d \ar[rd]^\theta & 
        \Omega^1_{\com}(A) \ar@{.>}[d]^{\Omega(\theta)} &\\
         &
        M
    } 
  \end{displaymath}
  In other words, there is an isomorphism of $R$-modules,
  \begin{equation*}
    D(A,M) \cong \Hom_R(\Omega^1_{\com}(A),M).
  \end{equation*}
  Where $D(A,M)$ is the set of derivations from $A$ to $M$.
  Note that this universal property defines $\Omega^1_{\com}(A)$ up
  to unique isomorphism.
\end{theorem}
\begin{proof}
    Basically, $\Omega(\theta)$ maps $adb$ to $a\theta(b)$. Checking the universal property
    is routine.
\end{proof}

We would now like to create a similar algebra of differentials for a non-commutative
associative algebra $A$ over $R$. In the non-commutative case, we must restrict
attention to derivations that take values in $A$-bimodules, rather than left
$A$ modules.
\begin{definition}
    Let $A$ be an associative
    unital algebra over a commutative ring $R$. 
    Let $m:A\otimes A\rightarrow A$ be the multiplication map. We define
    \begin{equation*}
        \Omega^1(A) = \ker(m)
    \end{equation*}
    This is an $A$ bimodule.
\end{definition}
This the motivation behind this definition is not at all clear. However, this
does agree with the commutative case and this provides the appropriate definition
for non-commutative K\"ahler differentials. To see this, we define the map
$d:A\rightarrow \Omega^1(A)$, by
\begin{equation*}
    d(a) = 1_A\otimes a-a\otimes 1_A.
\end{equation*}
We see that $d$ is a derivation. In fact, $\Omega^1(A)$
should be thought of as the space of all linear combinations
of terms of the form $ad(b)$.


$\Omega^1(A)$ satisfies the 
same universal property as $\Omega^1_{\com}$. Namely, if $M$ is an $A$-bimodule,
and $\theta:A\rightarrow M$ is a derivation, then there exists a unique $R$-linear
map $\Omega(\theta)$ such that the following diagram commutes,

    \begin{displaymath}
    \xymatrix{
        A \ar[r]^d \ar[rd]^\theta & 
        \Omega^1(A) \ar@{.>}[d]^{\Omega(\theta)} &\\
         &
        M
  } 
  \end{displaymath}
  
\subsection{Universal Differential Algebra}
Given an associative unital algebra $R$ over a commutative ring $R$,
we define
\begin{equation*}
    \Omega^p(A) := \bigotimes_{A,p} \Omega^1(A).
\end{equation*} 
And the algebra,
\begin{equation*}
    \Omega A = \bigoplus_{p} \Omega^p(A).
\end{equation*}
We extend the function $d:A\rightarrow \Omega^1(A)$ to $\Omega A$
by
\begin{equation*}
    d(a\otimes da_1\otimes da_2\otimes \cdots \otimes da_n) = da\otimes da_1\otimes da_2\otimes \cdots\otimes da_n.
\end{equation*}
 
\begin{theorem}
    $\Omega A$ is the ``largest" graded differential algebra generated by $A$. 
    
    If $(\Gamma,\Delta)$ is a graded differential algebra, with grading $\Gamma = \bigoplus_n \Gamma^{(n)}$,
    and $\rho:A\rightarrow \Gamma^{(0)}$ is an algebra homomorphism, then $\rho$
    extends uniquely to a morphism $\Omega A\rightarrow \Gamma$ such that the following
    diagram commutes,
    \begin{displaymath}
    \xymatrix{
        \Omega^p(A) \ar[r]^\rho \ar[d]^{d} & 
        \Gamma^{(p)} \ar[d]^\Delta&\\
        \Omega^{p+1}(A) \ar[r]^\rho & 
        \Gamma^{(p+1)}&
    }
    \end{displaymath}
\end{theorem}

\subsection{The Insufficiency of K\"ahler Differentials}
The ostensible purpose of K\"ahler differentials is to provide
a generalisation of the construction of the exterior algebra
bundle of a manifold to the non-commutative setting.

However the graded algebra of K\"ahler differentials
is in general considerably bigger than necessary for this purpose. 

For a manifold $M$, the classical exterior algebra
bundle $\Omega(M)$ has the property that $\Omega^k(M)$
is zero dimensional when $k$ exceeds the dimension of $M$. However 
for an algebra $A$, the dimensions of the spaces $\Omega^k(A)$
of K\"ahler differentials never become zero unless $A$
is already zero dimensional. 

Therefore in order to create a definition of the exterior
algebra bundle over a (potentially non-commutative) algebra $A$ which
more closely resembles the exterior algebra bundle over a manifold.

The solution is to give more information about $A$, and construct
an algebra of differential forms as a specific quotient of $\Omega(A)$.




\section{Non-commutative Geometry}
\subsection{Spectral Triples}
Recall from Chapter \ref{Introduction} that a general non-commutative
space should be thought of as a pair $(\A,\Hilb)$, where $\A$
is an algebra of operators on a Hilbert space $\Hilb$. 


We shall consider non-commutative spaces equipped with a ``Dirac Operator", $(\A,\Hilb,\D)$
such that $\A$ is contained in $\mathcal{B}(\Hilb)$.

A (commutative) example of this is as follows: Let $(M,g)$ be a compact Riemannian manifold,
and let $\A = C^\infty(M)$, and $\Hilb = L^2(M,g)$. However
this is not enough information to recover the geometry of $M$. A convenient
way to recover the geometry of $M$ from algebraic data is to give a Dirac operator.
The definition of a general Dirac operator affiliated
to a manifold is given in Appendix \ref{Dirac}. 


\begin{definition}[Spectral Triple]
    A spectral triple is a triple $(\A,\Hilb,\D)$, where
    $\Hilb$ is a Hilbert space with $\A \subseteq \mathcal{B}(\Hilb)$
    a $*$-algebra of operators.
    
    $\D$ is a densely defined unbounded operator on $\Hilb$ satisfying the following
    two properties:
    \begin{enumerate}
        \item{} $[\D,a]$ is densely defined and extends to a bounded operator
        for all $a \in \A$.
        \item{} $a(\lambda -\D)^{-1}$ is compact for all $a \in \A$ and $\lambda \in \Cplx\setminus \Rl$.
    \end{enumerate}
    
    A spectral triple $(\A,\Hilb,\D)$ can be either \emph{even} or \emph{odd}:
    \begin{itemize}
        \item{} We say that $(\A,\Hilb,\D)$ is even if there exists a
        $\Itgr/2\Itgr$ grading on the linear operators on $\Hilb$ such that $\A$
        is even and $\D$ is odd. Equivalently, there is an operator $\Gamma$
        on $\Hilb$ with $\Gamma^2 = 1$ and $\Gamma^* = \Gamma$ such that $a\Gamma = \Gamma a$ for all $a \in \A$
        and $\D\Gamma=-\Gamma\D$. 
        \item{} If no such operator $\Gamma$ exists, then we say that $(\A,\Hilb,\D)$
        is odd.
    \end{itemize}
\end{definition}



\subsection{Connes Differentials}
Given a spectral triple, $(\A,\Hilb,\D)$, we would like to construct an ``exterior algebra"
on $\A$. Connes does this by identifying the $1$-form $da$ with $[D,a]$.

Since $[D,a]$ is a derivation on $\A$, by the universal property we have a map, $\pi:\Omega\A\rightarrow \mathcal{B}(\Hilb)$
given by $\pi(ada_1da_2\cdots da_n) = a[D,a_1][D,a_2]\cdots[D,a_n]$.

One may then na\"ively define the algebra of differential forms as $\pi(\Omega \A)$,
but this does not work since there exists $a \in \Omega \A$ such that $\pi(a) = 0$
but $\pi(da) \neq 0$. These are called ``junk forms" and we must factor them out to get
a good differential algebra. Hence, define
\begin{definition}
    Let $J_0$ be the graded ideal of $\Omega \A$ defined by 
    \begin{equation*}
        J_0^{(p)} = \{a \in \Omega^p(\A) \;:\; \pi(a) = 0\}
    \end{equation*}
    And define $J^{(p)} = J_0^{(p)} + dJ_0^{(p)}$. Then $J = \bigoplus_p J^{(p)}$.


Now we can define the algebra of Connes' forms,
\begin{equation*}
    \Omega_{\D}\A = \frac{\Omega \A}{J} \cong \frac{\pi(\Omega \A)}{\pi(dJ_0)}
\end{equation*}

$\Omega_\D \A$ is naturally graded by the gradings on $\Omega \A$ and $J$, with the 
space of $p$-forms being $\Omega^p_\D \A = \Omega^p(\A)/J^{(p)}$.

Since $J$ is a differential ideal, the operator $d$ on $\Omega \A$
extends to $\Omega_\D \A$.  
\end{definition}

\begin{definition}[Quantum Differentiability]
    A spectral triple $(\A,\Hilb,\D)$ is called $QC^k$ for $k \geq 0$
    if $\A$ is contained in the domain of the operator $\delta^k$, where $\delta(a) = [|\D|,a]$.
\end{definition}

\begin{definition}[Summability]
    A spectral triple $(\A,\Hilb,\D)$ is called $(p,\infty)$ summable
    for $p > 0$ if $(1+\D^2)^{-1/2} \in \mathcal{L}^{p,\infty}$.
\end{definition}

        
\section{Quantised Differentials in Non-commutative Geometry}
We have already given a thorough description of the quantised differentials $\qd f$
when $f$ is a function on the circle. It so happens that we can find
analogous, although weaker, results in far greater generality.

It is in fact easier to move to a setting that is extremely general.

\begin{definition}
    Let $(\A,\Hilb,\D)$ be a spectral triple, either even or odd. Since $\D$
    is a self adjoint operator, by the Borel functional calculus
    we can define the operator $F := \sgn(\D)$. For $a \in \A$, we define
    \begin{equation}
        \qd a := [F,a].
    \end{equation}
\end{definition}

It is here that we see the distinction between \emph{derivatives}
and \emph{differentials}. In a spectral triple, the algebra
of Connes forms plays the role of derivatives of functions, and the quantised
differentials are a different object entirely.

\begin{theorem}
    Let $p \geq 1$, and
    let $(\A,\Hilb,\D)$ be a $(p,\infty)$ summable $QC^1$ spectral triple such that $\D$
    is invertible. Then, for all $a \in \A$
    \begin{equation}
        \qd a \in \mathcal{L}^{p,\infty}.
    \end{equation}
\end{theorem}
\begin{proof}
    Since $\D = |\D|F$, we can compute
    \begin{equation}
        [\D,a] = |\D|[F,a] + [|\D|,a]F.
    \end{equation}
    Or in other words,
    \begin{equation}
        da = |\D|\qd a + \delta(a)F.
    \end{equation}
    Since $\D$ is invertible, $|\D|$ is invertible. Thus,
    \begin{equation*}
        \qd a = |\D|^{-1}da + |\D|^{-1}\delta(a)F.
    \end{equation*}
    By assumption, $da,\delta(a) \in \mathcal{B}(\Hilb)$. Thus,
    since $(\A,\Hilb,\D)$ is $(p,\infty)$-summable, $|\D|^{-1} \in \mathcal{L}^{p,\infty}$.
    Hence, $\qd a \in \mathcal{L}^{p,\infty}$.
\end{proof} 

