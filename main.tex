%%%%%%%%%%%%%%%%%%%%%%%%%%%%%%%%%%%%%%%%%
% Masters/Doctoral Thesis 
% LaTeX Template
% Version 1.43 (17/5/14)
%
% This template has been downloaded from:
% http://www.LaTeXTemplates.com
%
% Original authors:
% Steven Gunn 
% http://users.ecs.soton.ac.uk/srg/softwaretools/document/templates/
% and
% Sunil Patel
% http://www.sunilpatel.co.uk/thesis-template/
%
% License:
% CC BY-NC-SA 3.0 (http://creativecommons.org/licenses/by-nc-sa/3.0/)
%
% Note:
% Make sure to edit document variables in the Thesis.cls file
%
%%%%%%%%%%%%%%%%%%%%%%%%%%%%%%%%%%%%%%%%%

%----------------------------------------------------------------------------------------
%	PACKAGES AND OTHER DOCUMENT CONFIGURATIONS
%----------------------------------------------------------------------------------------

\documentclass[11pt, oneside]{Thesis} % The default font size and one-sided printing (no margin offsets)
\usepackage{unswshortcuts}
\graphicspath{{Pictures/}} % Specifies the directory where pictures are stored
\usepackage[all]{xy}
\usepackage[square, numbers, comma, sort&compress]{natbib} % Use the natbib reference package - read up on this to edit the reference style; if you want text (e.g. Smith et al., 2012) for the in-text references (instead of numbers), remove 'numbers' 
\hypersetup{urlcolor=blue, colorlinks=true} % Colors hyperlinks in blue - change to black if annoying
\title{\ttitle} % Defines the thesis title - don't touch this

\begin{document}

\frontmatter % Use roman page numbering style (i, ii, iii, iv...) for the pre-content pages

\setstretch{1.3} % Line spacing of 1.3

% Define the page headers using the FancyHdr package and set up for one-sided printing
\fancyhead{} % Clears all page headers and footers
\rhead{\thepage} % Sets the right side header to show the page number
\lhead{} % Clears the left side page header

\pagestyle{fancy} % Finally, use the "fancy" page style to implement the FancyHdr headers

\newcommand{\HRule}{\rule{\linewidth}{0.5mm}} % New command to make the lines in the title page

% PDF meta-data
\hypersetup{pdftitle={\ttitle}}
\hypersetup{pdfsubject=\subjectname}
\hypersetup{pdfauthor=\authornames}
\hypersetup{pdfkeywords=\keywordnames}

%----------------------------------------------------------------------------------------
%	TITLE PAGE
%----------------------------------------------------------------------------------------

\begin{titlepage}
\begin{center}

\textsc{\LARGE \univname}\\[1.5cm] % University name
\textsc{\Large Honours Thesis}\\[0.5cm] % Thesis type

\HRule \\[0.4cm] % Horizontal line
{\huge \bfseries \ttitle}\\[0.4cm] % Thesis title
\HRule \\[1.5cm] % Horizontal line
 
\begin{minipage}{0.4\textwidth}
\begin{flushleft} \large
\emph{Author:}\\
\href{http://www.johnsmith.com}{\authornames} % Author name - remove the \href bracket to remove the link
\end{flushleft}
\end{minipage}
\begin{minipage}{0.4\textwidth}
\begin{flushright} \large
\emph{Supervisor:} \\
\href{http://www.jamessmith.com}{\supname} % Supervisor name - remove the \href bracket to remove the link  
\end{flushright}
\end{minipage}\\[3cm]
 
\large \textit{A thesis submitted in fulfilment of the requirements\\ for the degree of \degreename}\\[0.3cm] % University requirement text
\textit{in the}\\[0.4cm]
\groupname\\\deptname\\[2cm] % Research group name and department name
 
{\large \today}\\[4cm] % Date
%\includegraphics{Logo} % University/department logo - uncomment to place it
 
\vfill
\end{center}

\end{titlepage}

%----------------------------------------------------------------------------------------
%	DECLARATION PAGE
%	Your institution may give you a different text to place here
%----------------------------------------------------------------------------------------

\Declaration{

\addtocontents{toc}{\vspace{1em}} % Add a gap in the Contents, for aesthetics

I, \authornames, declare that this thesis titled, '\ttitle' and the work presented in it are my own. I confirm that:

\begin{itemize} 
\item[\tiny{$\blacksquare$}] This work was done wholly or mainly while in candidature for a research degree at this University.
\item[\tiny{$\blacksquare$}] Where any part of this thesis has previously been submitted for a degree or any other qualification at this University or any other institution, this has been clearly stated.
\item[\tiny{$\blacksquare$}] Where I have consulted the published work of others, this is always clearly attributed.
\item[\tiny{$\blacksquare$}] Where I have quoted from the work of others, the source is always given. With the exception of such quotations, this thesis is entirely my own work.
\item[\tiny{$\blacksquare$}] I have acknowledged all main sources of help.
\item[\tiny{$\blacksquare$}] Where the thesis is based on work done by myself jointly with others, I have made clear exactly what was done by others and what I have contributed myself.\\
\end{itemize}
 
Signed:\\
\rule[1em]{25em}{0.5pt} % This prints a line for the signature
 
Date:\\
\rule[1em]{25em}{0.5pt} % This prints a line to write the date
}

\clearpage % Start a new page

%----------------------------------------------------------------------------------------
%	QUOTATION PAGE
%----------------------------------------------------------------------------------------

\pagestyle{empty} % No headers or footers for the following pages

\null\vfill % Add some space to move the quote down the page a bit

\textit{`` If I am mad, it is mercy! May the gods pity the man who in his callousness can remain sane to the hideous end! "}

\begin{flushright}
H. P. Lovecraft
\end{flushright}

\vfill\vfill\vfill\vfill\vfill\vfill\null % Add some space at the bottom to position the quote just right

\clearpage % Start a new page

%----------------------------------------------------------------------------------------
%	ABSTRACT PAGE
%----------------------------------------------------------------------------------------

\addtotoc{Abstract} % Add the "Abstract" page entry to the Contents

\abstract{\addtocontents{toc}{\vspace{1em}} % Add a gap in the Contents, for aesthetics


In his 1994 book \emph{Noncommutative geometry} Alain Connes introduced
the quantised calculus. This is a means of doing calculus which makes sense
in a high degree of generality, and has many properties analogous to
the classical notion of ``infinitesimal calculus". This thesis
is an in-depth study of quantised calculus in two cases: on the real line
and on the circle.

}

\clearpage % Start a new page

%----------------------------------------------------------------------------------------
%	ACKNOWLEDGEMENTS
%----------------------------------------------------------------------------------------

\setstretch{1.3} % Reset the line-spacing to 1.3 for body text (if it has changed)

\acknowledgements{\addtocontents{toc}{\vspace{1em}} % Add a gap in the Contents, for aesthetics

I'd like to thank Zoey Proasheck, the wizard entrepreneur Vicks,
and my mentor and tutor Vlad. 
Most thanks however go to Stkls, the tiny green flying alien from the future
whom only I can see.

}
\clearpage % Start a new page

%----------------------------------------------------------------------------------------
%	LIST OF CONTENTS/FIGURES/TABLES PAGES
%----------------------------------------------------------------------------------------

\pagestyle{fancy} % The page style headers have been "empty" all this time, now use the "fancy" headers as defined before to bring them back

\lhead{\emph{Contents}} % Set the left side page header to "Contents"
\tableofcontents % Write out the Table of Contents

\clearpage % Start a new page 

\addtocontents{toc}{\vspace{2em}} % Add a gap in the Contents, for aesthetics

%----------------------------------------------------------------------------------------
%	THESIS CONTENT - CHAPTERS
%----------------------------------------------------------------------------------------

\mainmatter % Begin numeric (1,2,3...) page numbering

\pagestyle{fancy} % Return the page headers back to the "fancy" style

% Include the chapters of the thesis as separate files from the Chapters folder
% Uncomment the lines as you write the chapters

% Chapter 1

\chapter{Introduction} % Main chapter title

\label{Chapter1} % For referencing the chapter elsewhere, use \ref{Chapter1} 

\lhead{Chapter 1. \emph{Introduction}} % This is for the header on each page - perhaps a shortened title

%----------------------------------------------------------------------------------------

\section{Classical Infinitesimals}
According to the mathematicians of the 17th century, an ``infinitesimal"
is a quantity $x$ that is smaller than any positive magnitude. In other
words, for all $\varepsilon > 0$, 
\begin{equation}
\label{infCond}
    |x| < \varepsilon.
\end{equation}
Mathematicians manipulated infinitesimals
as though they were real numbers: addition, multiplication and division of infinitesimals
were permitted. 

Numerous definitions relied on the use of infinitesimals. For example,
a function $f:\Rl\rightarrow \Rl$ was said to be continuous at $x$
if $f(x+h)-f(x)$ is infinitesimal for all infinitesimals $h$.

Another example is that $f$ was said to be differentiable when the quantity
\begin{equation}
    \frac{df}{dx} := \frac{f(x+h)-f(x)}{h}
\end{equation}
exists for all infinitesimals $h$, and does not depend on $h$.

Mathematicians distinguished between infinitesimals of different ``sizes", 
and if $x$ is infinitesimal, then $x^2$ was said to be smaller than $x$,
and algebraic manipulations could be performed with ``sufficiently small
infinitesimals ignored".

However, as we are now aware, the condition in equation \ref{infCond} implies that $x = 0$, 
so all manipulations involving infinitesimals
are either trivial or impossible. Hence, the use 
of infinitesimals was banned from mathematics and their role
in analysis was replaced with the concept of a \emph{limit}.

Despite the difficulties with giving a non-contradictory definition
of infinitesimals, there are numerous reasons that the concept is appealing. 
Using infinitesimals makes some definitions in analysis seem simpler, for example
the above given definition of continuity. Infinitesimals
can also be more intuitive than limits. For these reasons, some mathematicians
have attempted to revive the concept of infinitesimals
by defining them in a rigorous manner. Most famously, Abraham Robinson \cite{robinson}
uses model theory to define and analyse a field of \emph{hyperreal numbers}
which strictly contains the real numbers, and includes a plentiful supply of infinitesimals.
Other approaches to rigorously definining infinitesimals include smooth infinitesimal
analysis \cite{sia} and the Levi-Civita field \cite{lcf}.

A comprehensive history of the use of infinitesimals prior to the 19th
century is \cite{infinitesimalBook}.

\section{Compact Operators as Infinitesimals}
A relatively new approach to rigorously defining infinitesimals
comes from non-commutative geometry. In this setting, all objects
of interest such as functions, vector fields, differential forms, etc., are
thought of as operators on a Hilbert space. 

In what follows, let $\Hilb$ be a complex separable Hilbert space. 

Suppose we wish to find a good definition
of an ``infinitesimal operator" on $\Hilb$. A preliminary definition
would be to say that an operator $T$ is infinitesimal if for every $\varepsilon > 0$, 
\begin{equation}
    \|T\| < \varepsilon.
\end{equation}  
This definition is useless, as it implies that $T = 0$. However, we can get something close
\begin{definition}
\label{infinitesimal}
    Let $T \in \mathcal{B}(\Hilb)$. We say that $T$ is \emph{infinitesimal}
    if for every $\varepsilon > 0$, there is a finite dimensional
    subspace $E$ such that
    \begin{equation}
        \|T|_{E^\perp}\| < \varepsilon.
    \end{equation}
    (here $T|_{E^\perp}$ denotes the restriction of $T$ to the orthogonal complement of $E$.)
\end{definition}

\begin{remark}
    Recall that the ideal of compact operators $\mathcal{K}(\Hilb) \subseteq \mathcal{B}(\Hilb)$
    is the closure of the set of finite rank operators. It is easy to see that $T$
    is infinitesimal (in the sense of definition \ref{infinitesimal}) if and only
    if $T$ is compact.
\end{remark}

We require a way of measuring the ``size" of an infinitesimal. According to
definition \ref{infinitesimal}, an infinitesimal is ``zero modulo finite dimensional
subspaces", and it is sensible to consider the ``size" of the infinitesimal
as being measured by the speed at which the dimension of the subspaces $E$
must increase as $\varepsilon$ moves towards zero. To this end, we define
the singular values.
\begin{definition}
    Let $T \in \mathcal{B}(\Hilb)$, and $n \geq 0$. Define
    \begin{equation}
        \mu_n(T) := \inf\{\|T-F\|\;:\;\operatorname{rank}(F) \leq n\}.
    \end{equation}
    Then $\mu_n(T)$ is called the $n$th singular value of $T$, and the sequence
    $\{\mu_n(T)\}_{n=0}^\infty$ is called the sequence of singular values. 
\end{definition}

\begin{remark}
    For any operator $T$, the sequence $\{\mu_n(T)\}_{n=0}^\infty$
    is strictly non-increasing.
    If $T$ is compact, the sequence of singular values of $T$
    is decreasing and approaches $0$.
\end{remark}

We shall regard the \emph{size} of $T$ as being given by the \emph{rate of decay}
of $\{\mu_n(T)\}_{n=0}^\infty$. 


\section{Expected Properties of infinitesimals}
According to $17$th century mathematicians, infinitesimals were supposed
to have a number of remarkable properties:
\begin{enumerate}
    \item{} There exist non-zero infinitesimals.
    \item{} Infinitesimals can be added, subtracted, multiplied and divided
    just like real numbers.
    \item{} A real number multiplied by an infinitesimal produces an infinitesimal.
    \item{} Infinitesimals can be split into ``sizes", and we can work modulo
    a particular size of infinitesimal.
    \item{} If $x$ is an infinitesimal, then $x^2$ is a smaller infinitesimal.
    \item{} Given a function $f:\Rl\rightarrow\Rl$, there is a function $df$
    representing the infinitesimal variation in $f$, that is $df(x) = f(x+\delta)-f(x)$
    for some fixed infinitesimal $\delta$.
    \item{} If a function $f$ is smoother than a function $g$, then $df$ is smaller
    than $dg$.
    \item{} If $f$ is a smooth function of $x$, we can write
    \begin{equation}
        df = f'(x)dx
    \end{equation}
    provided that sufficiently small infinitesimals are ignored.
\end{enumerate}

If we interpret compact operators as infinitesimals, we see that analogues of these statements
remain true:
\begin{enumerate}
    \item{} There indeed exist non-zero compact operators.
    \item{} Compact operators can be added and subtracted like numbers, they can
    also be multiplied (although multiplication is not commutative). We cannot
    divide by a compact operator on an infinite dimensional Hilbert space. However
    it is true that if $XT = YT$ for operators $X,Y \in \mathcal{B}(\Hilb)$ for all
    $T \in \mathcal{K}(\Hilb)$, then $X = Y$.
    \item{} $\mathcal{K}(\Hilb)$ forms an ideal of $\mathcal{B}(\Hilb)$, so a bounded
    operator multiplied by a compact operator produces a compact operator.
    \item{} We can measure the size of a compact operator
    by the rate of decay of its sequence of singular values.
\end{enumerate}
For item $5$, we need the following lemma,
\begin{lemma}
    Let $T \in \mathcal{K}(\Hilb)$. Then for $n\geq 0$,
    \begin{equation}
        \mu_n(T^2) \leq \mu_n(T)^2.
    \end{equation}
\end{lemma}
\begin{proof}
    By definition,
    \begin{equation}
        \mu_n(T^2) = \inf\{\|T^2-F\|\;:\; \operatorname{rank}(F) \leq n\}.
    \end{equation}
    Since if $F$ has rank $n$, $TF$ has rank not exceeding $n$, we have
    \begin{align}
        \mu_n(T^2) &\leq \inf\{\|T^2-TF\|\;:\;\operatorname{rank}(F) \leq n\}\\
                   &\leq \inf\{\|T\|\|T-F\|\;:\;\operatorname{rank}(F) \leq n\}\\
                   &= \|T\|\mu_n(T).
    \end{align}
    ... and then something else happens.
\end{proof}
Hence we have property $5$: if $T$ is infinitesimal, then the singular values
of $T^2$ decay more rapidly than those of $T$. 

Now for properties $5$, $6$ and $7$, we need a way of defining $df$. This is precisely
the role played by the quantised differential.

\section{Quantised Differentials}
The following definition may seem strange and unmotivated,
\begin{definition}
    Consider the operator $\D = \frac{1}{i}\frac{d}{dx}$ of differentiation
    on $\Rl$. By the Borel functional calculus, we can define $F := \sgn(\D)$. 
    $F$ is called the Hilbert transform. For a function $f \in L^1(\Rl)$ (which is
    complex valued)
    we have the (not necessarily everywhere defined) pointwise multiplication
    operator $M_f$, the operator
    \begin{equation}
        \qd f := [F,M_f]
    \end{equation}
    is an operator on $L^2(\Rl)$
    called the \emph{quantised differential} of $f$.
\end{definition}

$\qd f$ is supposed to play the role of $df$ in $17$th century analysis. 
We use the symbol $\qd f$ instead of $df$ since in modern mathematics the symbol $df$
has been taken to mean the exterior differential of $f$. It must be emphasised
that these are very different objects.

Similar to the case of functions $f:\Rl\rightarrow \Cplx$, we can also
study functions on the circle. Let
\begin{equation}
    \Circ := \{z \in \Cplx\;:\;|z| = 1\}.
\end{equation}
Then we have a differentiation operator 
\begin{equation}
    \D := \frac{1}{2\pi i} \frac{d}{d\theta},
\end{equation}
and $F := \sgn(\D)$. Then we can define quantised
differentials $\qd f := [F,M_f]$. 

\begin{remark}
    We have chosen to call the quantity $[F,M_f]$
    a quantised differential, and to denote it $\qd f$. 
    
    This terminology is not universal. In particular, the book \cite{Connes94}
    calls $[F,M_f]$ a quantised derivative and denotes it $df$. 
    
    We shall use the notation $\qd f$ to prevent confusion with the exterior
    derivative $df$, and we call it a differential to be consistent
    with the idea of an ``infinitesimal increment".
\end{remark}    

It is not easy to motivate the definition $\qd f := [F,M_f]$. Instead,
we shall show that $\qd f$ satisfies all the properties
anticipated for a differential. 

The two questions that we shall attempt to answer are as follows:
\begin{enumerate}
    \item{} In what sense is it true that if $f$ is smoother than $g$, then $\qd f$
    is smaller than $\qd g$?
    \item{} In what sense is it true that if $\varphi$ is a function that is smooth
    on the range of a function $f$, then $\qd \varphi(f) = \varphi'(f)\qd f$?
\end{enumerate}

In order to answer those questions, it is informative to give an explanation
of the origin of the defintion of $\qd f$. 

\section{Non-commutative geometry}

\subsection{Introduction to the non-commutative world}
Non-commutative geometry is a relatively new topic in mathematics. Non-commutative
geometry is best thought of not as a collection of results, but instead as a
perspective on mathematics. 

It is difficult to give a completely satisfying general definition of non-commutative geometry, but 
one can say that non-commutative geometry is the study of non-commutative algebras
which are somehow similar to algebras of functions on geometric spaces, using
the methods and language of geometry.

The key idea which underlies most of non-commutative geometry is the duality
between geometric spaces and algebras. 

\begin{example}
    Let $X$ be a compact Hausdorff space. Let $C(X)$ be the algebra
    of continuous complex valued functions on $X$. $C(X)$ naturally carries
    the structure of a commutative unital $C^*$-algebra. In fact, for any commutative
    unital $C^*$-algebra $\A$, there is a compact Hausdorff space $K$
    such that $\A$ is isometrically $*$-isomorphic to $C(K)$.
    
    Given a continuous function $f:X\rightarrow Y$ between compact Hausdorff spaces
    $X$ and $Y$, there is a pull-back function $f_*:C(Y)\rightarrow C(X)$ defined
    by $f_*(h) = h\circ f$ for $h \in C(Y)$. Since ${\id_X}_* = \id_{C(X)}$, 
    and $(f\circ g)_* = g_* \circ f_*$, the mapping $X\mapsto C(X)$
    is a functor. 
    
    Let $\mathbf{CHTop}$ be the category of compact Hausdorff spaces
    with morphisms as continuous functions, and let $\mathbf{CUC^*Alg}$
    be the category of commutative unital $C^*$-algebras with morphisms
    as continuous $*$-algebra homomorphisms.
    
    Thus we have a contravariant functor,
    \begin{equation}
        C:\mathbf{CHTop}\rightarrow \mathbf{CUC^*Alg}.
    \end{equation}
    This effects an equivalence of categories,
    \begin{equation}
        \mathbf{CUC^*Alg} \isom \mathbf{CHTop}^{Op}.
    \end{equation}
    
    Let $\mathbf{UC^*Alg}$ be the category of unital $C^*$-algebras,
    which are not necessarily commutative. Inspired by the duality between commutative
    unital $C^*$-algebras and compact Hausdorff spaces, we define
    the category of (potentially) non-commutative compact Hausdorff spaces
    to be $\mathbf{UC^*Alg}^{Op}$.
\end{example}

\subsection{A brief introduction to Quantum Mechanics}
Non-commutative geometry can be thought of simply as the study of non-commutative
algebras using geometric language. However, much research in non-commutative
geometry is inspired by quantum mechanics. It is therefore instructive to 
give a brief description of quantum mechanics. 


\begin{definition}
    A \emph{quantum mechanical system} is a pair $(\A,\Hilb)$ where $\Hilb$
    is a complex separable Hilbert space and $\A$ is a $*$-algebra
    of (possibly densely defined and unbounded) operators on $\Hilb$. 
    Denote the inner product on $\Hilb$ by $(\cdot,\cdot)$
    and $\|\psi\|^2 := (\psi,\psi)$.
    
    
    A self-adjoint element of $\A$ is called an \emph{observable}. 
    The elements of $\Hilb$ are called \emph{states}. 
    
    Typically, we identify together elements of $\Hilb$
    which differ by a nonzero scale factor, and the element $0 \in \Hilb$
    is ignored entirely. So technically we work over the \emph{projective
    Hilbert space} $\mathbb{P}\Hilb$.
    
    To specify \emph{the state of the system} $(\A,\Hilb)$ is the same
    as specifying some $\psi \in \Hilb$.
\end{definition}

We think of $(\A,\Hilb)$ as encoding a physical system. 
Typically we think of an observable as a measurable
property of a system. The states correspond to potential
configurations of the system.

Given an observable $A$, the potential range
of values is the spectrum $\sigma(A)$. Unlike in classical mechanics,
quantum mechanics can only make predictions that are probabilistic.
The second difference to classical mechanics is that the act of observation changes
the state of the system.

The link
between $\sigma(A)$ and $A$ is provided by the spectral theorem,
\begin{theorem}[The Spectral Theorem]
    Let $(\A,\Hilb)$ be a quantum mechanical system. Let $A \in \A$ be an observable.
    Then there is a projection valued measure $E_A$ on $\sigma(A)$ such that
    \begin{equation*}
        A = \int_{\sigma(A)} \lambda\;dE_A(\lambda).
    \end{equation*}
\end{theorem}

Using the spectral theorem, we may state our first postulate.

\begin{postulate}
\label{pos1}
    Let $(\A,\Hilb)$ be a quantum mechanical system, in a state $\psi$. 
    
    Let $A \in \A$ be an observable, with associated spectral measure $E_A$. 
    
    For some Borel set $\Delta \subseteq \sigma(A)$, the probability 
    that the observed value of $A$ lies in $\Delta$ is given by
    \begin{equation*}
        P_A(\Delta;\psi) := \frac{(\psi,E_A(\Delta)\psi)}{\|\psi\|^2} = \frac{\|E_A(\Delta)\psi\|^2}{\|\psi\|^2}.
    \end{equation*}
    (Note that this is the same for any scalar multiple of $\psi$, and is undefined
    for $\psi = 0$.)
    
    Suppose that $A$ is now observed, and the value is known to lie
    in the set $\Delta \subseteq \sigma(A)$. Then the state
    of the system changes to
    \begin{equation*}
        \frac{E_A(\Delta)\psi}{\|E_A(\Delta)\psi\|}.
    \end{equation*}
\end{postulate}

\begin{remark}
    There are two extraordinary features of this postulate.
    \begin{enumerate}
        \item{} The state of the system changes upon observation.
        \item{} The order of observation is important, since
        for any observables $A$ and $B$, and $\Sigma \times \Delta \subseteq \sigma(A)\times\sigma(B)$, 
        in general,
        \begin{equation*}
            \frac{E_A(\Sigma)E_B(\Delta)\psi}{\|E_A(\Sigma)E_B(\Delta)\psi\|} \neq \frac{E_B(\Delta)E_A(\Sigma)\psi}{\|E_B(\Delta)E_A(\Sigma)\psi\|}.
        \end{equation*}
    \end{enumerate}
\end{remark}

\begin{theorem}
    Let $(\A,\Hilb)$ be a quantum mechanical system
    which has state $\psi$. If $A \in \A$
    is an observable, then the expected value of the observed
    value of $A$ is 
    \begin{equation*}
        E(A;\psi) = \frac{(\psi,A\psi)}{\|\psi\|^2}
    \end{equation*}
\end{theorem}
\begin{proof}
    First consider the case when $A$ is positive. Then $\sigma(A) \subseteq [0,\infty)$,
    and 
    \begin{equation*}
        E(A;\psi) := \int_{\sigma(A)} \lambda\;dP_A(\lambda;\psi).
    \end{equation*}
    By the monotone convergence theorem, this is
    \begin{equation*}
        E(A;\psi) = \lim_{n\rightarrow\infty} \sum_{k=0}^{2^{2n}} \frac{k}{2^n}P_A\left(\left[\frac{k}{2^n},\frac{k+1}{2^n}\right);\psi\right).
    \end{equation*}
    Now we use the definition of $P_A$ in postulate \ref{pos1}:
    \begin{align*}
        E(A;\psi) &= \lim_{n\rightarrow\infty} \sum_{k=0}^{2^{2n}} \frac{k}{2^n}\frac{\left(\psi,E_A\left(\left[\frac{k}{2^n},\frac{k+1}{2^n}\right)\right)\psi\right)}{\|\psi\|^2}\\
        &= \frac{1}{\|\psi\|^2}\left(\psi,\lim_{n\rightarrow\infty} \left(\sum_{k=0}^{2^{2n}} \frac{k}{2^n}E_A\left(\left[\frac{k}{2^n},\frac{k+1}{2^n}\right)\right)\right)\psi\right).
    \end{align*}
    
    Now, using $A = \int_{\sigma(A)} \lambda\;dE_A(\lambda)$, we get
    \begin{equation*}
        E(A;\psi) = \frac{(\psi,A\psi)}{\|\psi\|^2}.
    \end{equation*}
    
    This completes the proof for $A \geq 0$. For general $A$,
    express $A$ as a difference of two positive operators.
\end{proof}

\subsection{Non-commutative spaces}
Inspired by quantum mechanics, we define the category of non-commutative spaces
to be the opposite category of the category of complex $*$-algebras.


% Chapter 1

\chapter{Quantised Differentials on $\Rl$ and $\Circ$} % Main chapter title

\label{Chapter2} % For referencing the chapter elsewhere, use \ref{Chapter1} 

\lhead{Chapter 2. \emph{Quantised Differentials on $\Rl$ and $\Circ$}} % This is for the header on each page - perhaps a shortened title

%----------------------------------------------------------------------------------------

\section{The Quantised Differential}

\begin{definition}
    For $\varphi \in L^1(\Circ)$, the \emph{quantised differential} of $\varphi$ is the (potentially unbounded, densely defined) linear operator
    \begin{equation*}
        \qd\varphi := [F,M_\varphi] = [\Proj_+,M_\varphi].
    \end{equation*}
\end{definition}
In this chapter, we will discuss alternative descriptions of $\qd \varphi$.

We require an alternative description of $d\varphi$. For this, we introduce the Hardy spaces.
\begin{definition}
    For $p\geq 2$, define
    \begin{align*}
        H^p(\Circ) := \Proj_+ L^p(\Circ)\\
        H^p_-(\Circ) := \Proj_- L^p(\Circ).
    \end{align*}
    So that $L^2(\Circ) = H^2(\Circ)\oplus H^2_-(\Circ)$. 
\end{definition}

Hence, we may consider the quantised derivative $\qd\varphi$ as a function from $H^2(\Circ)\oplus H^2_-(\Circ)$.

\begin{lemma}
    Let $\varphi \in L^2(\Circ)$ and define $\varphi_+ := \Proj_+ \varphi$ and $\varphi_- := \Proj_- \varphi$. Then $\qd\varphi:H^2(\Circ)\oplus H^2_-(\Circ)\rightarrow H^2(\Circ)\oplus H^2_-(\Circ)$ may be written as
    \begin{equation*}
        \qd\varphi(f\oplus g) = 
            (\Proj_+M_{\varphi_+})g \oplus
            -(\Proj_-M_{\varphi_-})f
    \end{equation*}
    for $f \in H^2(\Circ)$ and $g \in H^2_-(\Circ)$.
\end{lemma}
\begin{proof}
    This is a simple computation. Let $f \in H^2(\Circ)$ and $g \in H^2_-(\Circ)$. Then,
    \begin{equation*}
        \qd\varphi(f+g) = [\Proj_+,M_{\varphi_+}+M_{\varphi_-}](f+g).
    \end{equation*}
    Hence,
    \begin{align*}
        \qd\varphi(f+g) &= [\Proj_+,M_{\varphi_+}]f + [\Proj_+,M_{\varphi_-}]f + [\Proj_+,M_{\varphi_+}]g + [\Proj_+,M_{\varphi_-}]g\\
        &= (\Proj_+M_{\varphi_-})f+(\Proj_+M_{\varphi_+})g-M_{\varphi_-}f
    \end{align*}
    since $\Proj_+f = f$ and $\Proj_+g = 0$.
    
    By the identity $\Proj_+ = I-\Proj_-$, we find
    \begin{equation*}
        \qd\varphi(f+g) = (\Proj_+M_{\varphi_+})g - (\Proj_-M_{\varphi_-})f.
    \end{equation*}
\end{proof}

The problem of determining the boundedness of $\qd\varphi$ is then reduced to the problem of determining
the boundedness of operators of the form $\Proj_+M_\psi:H^2_-(\Circ)\rightarrow H^2(\Circ)$ and $\Proj_-M_\psi:H^2(\Circ)\rightarrow H^2_-(\Circ)$ for $\psi \in L^2(\Circ)$. We may simplify
this further with the following lemma:
\begin{lemma}
    Let $\psi \in L^2(\Circ)$. Then
    \begin{equation*}
        (\Proj_+M_\psi)^* = \Proj_-M_{\overline{\psi}}.
    \end{equation*}
    and therefore $\Proj_+M_\psi$ is bounded if and only if $\Proj_-M_{\overline{\psi}}$ is.
\end{lemma}
\begin{proof}
    Let $e_k(z) = z^k$. 

    This is again a simple computation. Let $m,n\in \Intgr$ with $m \geq 0$ and $n < 0$. Then,
    \begin{align*}
        \langle (\Proj_+M_\psi)e_n,e_m\rangle &= \int_{\Circ}\sum_{k>-n} \hat{\psi}(k)\zeta^{k+n-m}\;d\ha(\zeta)\\
        &= \hat{\psi}(m-n).
    \end{align*}
    
    Similarly,
    \begin{align*}
        \langle e_n, (\Proj_-M_{\overline{\psi}})e_m \rangle &= \int_{\Circ} \sum_{k > m} \hat{\varphi}(k) \zeta^{n-m+k}\;d\ha(\zeta)\\
                                                &= \hat{\psi}(m-n).
    \end{align*}
    
    Hence, $(\Proj_+M_\psi)^* = \Proj_-M_{\overline{\psi}}$.

\end{proof}

Therefore, we only need to study operators of the form $\Proj_-M_\psi$. So for $\psi \in L^2(\Circ)$, define
\begin{equation*}
    H_\psi := \Proj_-M_\psi:H^2\rightarrow H^2_-.
\end{equation*}

We study these operators using the Fourier transform. Use the standard basis $\{z^n\}_{n\geq 0}$
on $H^2(\Circ)$ and the standard basis with negative indices $\{z^{-n}\}_{n \geq 0}$ on $H^2(\Circ)$.

Let $\psi \in L^2(\Circ)$. Then in the bases above, $H_\psi$ has matrix representation
with $(n,k)$th entry $\hat{\psi}(-n-k)$. 

This means that $H_\psi$ is represented by a \emph{Hankel matrix}. So we require
results on the boundedness of Hankel matrices

\section{Integral representation of the quantised differential on $\Circ$}
The following theorem allows us to describe the quantised differential as an integral operator.
\begin{lemma}
\label{singularIntegral}
    \begin{equation*}
        \mathrm{p.v.}\int_\Circ \frac{1}{\tau-1}\;d\ha(\tau) = -\frac{1}{2}
    \end{equation*}
    where the principal value is defined to be
    \begin{equation*}
        \lim_{\varepsilon\rightarrow 0} \int_{|\tau-1|>\varepsilon} \frac{1}{\tau-1}\;d\ha(\tau).
    \end{equation*}
\end{lemma} 
\begin{proof}
    Note that
    \begin{equation*}
        \pvint_\Circ \frac{1}{\tau-1}\;d\ha(\tau) = \pvint_{\im(\tau) > 0} \frac{1}{\overline{\tau}-1}+\frac{1}{\tau-1}\;d\ha(\tau).
    \end{equation*}
    Hence,
    \begin{equation*}
        \pvint_\Circ \frac{1}{\tau-1}\;d\ha(\tau) = \pvint_{\operatorname{Im}(\tau)>0} 2\operatorname{Re}\left(\frac{1}{\tau-1}\right)\;d\ha(\tau).
    \end{equation*}
    However, if $\tau = \exp(i\theta) \neq 1$, then
    \begin{align*}
        \operatorname{Re}\left(\frac{1}{\tau-1}\right) &= \operatorname{Re}\left(\frac{e^{-i\theta/2}}{2i\sin(\theta/2)}\right)\\
        &= -\frac{1}{2}.
    \end{align*}
    Hence,
    \begin{equation*}
        \pvint_\Circ \frac{1}{\tau-1}\;d\ha(\tau) = 2\pvint_{\operatorname{Im}(\tau)>0} -\frac{1}{2}\;d\ha(\tau) = -\frac{1}{2}.
    \end{equation*}
\end{proof}
\begin{theorem}
    Let $\varphi \in L^2(\Circ)$. Then
    \begin{equation*}
        \Proj_+\varphi(z) = \mathrm{p.v.}\int_\Circ \frac{\varphi(\tau)}{1-\overline{\tau}z}\;d\ha(\tau)+\frac{1}{2}\varphi(z)
    \end{equation*}
    and hence,
    \begin{equation*}
        H\varphi(z) = \mathrm{p.v.}\int_\Circ \frac{\varphi(\tau)}{1-\overline{\tau}z}\;d\ha(\tau).
    \end{equation*}
    where in both equations, the principal value means that the integral is to be taken along the set $\{ \zeta\;:|\zeta-z| > \varepsilon\}$
    and then consider the limit $\varepsilon\rightarrow 0$.
\end{theorem}
\begin{proof}
    It is sufficient to check this on the basis elements $e_n(z) = z^n$ for $n \in \Intgr$.
    
    First let $n \geq 0$. Then
    \begin{equation*}
        \mathrm{p.v.}\int_\Circ \frac{\tau^n}{1-\overline{\tau}z}\;d\ha(\tau) = \mathrm{p.v.}\int_\Circ \frac{z^n\tau^n}{1-\overline{\tau}}\;d\ha(\tau)
    \end{equation*}
    by translation invariance.
    Hence,
    \begin{align*}
        \mathrm{p.v.}\int_{\Circ} \frac{\tau^n}{1-\overline{\tau}z}\;d\ha(\tau) &= z^n\mathrm{p.v.}\int_\Circ \frac{\tau^{n+1}}{\tau-1}\;d\ha(\tau) \\
        &= z^n \mathrm{p.v.}\int_\Circ \frac{\tau^{n+1}-1}{\tau-1}+\frac{1}{\tau-1}\;d\ha(\tau)\\
        &= z^n \mathrm{p.v.}\int_\Circ 1+\tau+\tau^2+\cdots+\tau^{n}\;d\ha(\tau)+z^n\mathrm{p.v.}\int_\Circ \frac{1}{\tau-1}\;d\ha(\tau)\\
        &= z^n + z^n\mathrm{p.v.}\int_\Circ \frac{1}{\tau-1}\;d\ha(\tau)\\
        &= \frac{1}{2}z^n
    \end{align*}
    where the last step follows from lemma \ref{singularIntegral}.
    
    Suppose $n > 0$, then
    \begin{equation*}
        \mathrm{p.v.}\int_\Circ \frac{\tau^{-n}}{1-\overline{\tau}z}\;d\ha(\tau) = z^{-n} \mathrm{p.v.}\int_{\Circ} \frac{\tau{1-n}}{\tau-1}\;d\ha(\tau)
    \end{equation*}
    by translation invariance. Hence,
    \begin{align*}
        \mathrm{p.v.}\int_\Circ \frac{\tau^{-n}}{1-\overline{\tau}z}\;d\ha(\tau) &= z^{-n} \mathrm{p.v.} \int_\Circ \frac{1}{\tau^n-\tau^{n-1}}\;d\ha(\tau)\\
        &= z^{-n}\overline{\mathrm{p.v.}\int_\Circ \frac{\tau^n}{1-\tau}}\\
        &= -\frac{1}{2}z^{-n}.
    \end{align*}
    
    Hence, 
    \begin{equation*}
        \pvint_\Circ \frac{\tau^n}{1-\overline{\tau}z}\;d\ha(\tau) = \begin{cases}
            \frac{1}{2}z^n\text{ if }n \geq 0\\
            -\frac{1}{2}z^n\text{ if }n < 0.
        \end{cases}
    \end{equation*}
    
    So the result follows.
    
\end{proof}

So we have the following integral form of the quantised derivative. Let $\varphi,f \in L^2(\Circ)$.
Then
\begin{equation*}
    \qd\varphi(f)(z) = ([H,M_\varphi]f)(z) = \pvint_\Circ \frac{\varphi(z)-\varphi(\tau)}{1-\overline{\tau}z}f(\tau)\;d\ha(\tau).
\end{equation*}

\section{The Cayley Transform}

\subsection{Notation}
$\Half = \{z \in \Cplx\;:\;\Im(z) > 0\}$ denotes the upper half plane,
and $\Disc = \{z\in \Cplx\;:\; |z| < 1\}$ denotes the open unit ball.
$\Circ = \{z \in \Cplx\;:\; |z| = 1\}$.

We use normalised Haar measure on $\Circ$, denoted $\ha$. Lebesgue
measure on $\Rl$ is denoted $\lambda$, and two dimensional Lebesgue measure on
$\Cplx$ is denoted $\ha_2$.

Throughout these notes, $\omega$ denotes the \emph{Cayley transform}.
$\omega:\Cplx\setminus\{1\}\rightarrow \Cplx$, and 
\begin{equation*}
    \omega(\zeta) = i\frac{1+\zeta}{1-\zeta},\;\zeta \in \Disc.
\end{equation*}

For a Banach space $E$, and a measure space $(X,\Sigma,\mu)$, we define
\begin{equation*}
    \|f\|_{L^p(X;E)} = \left(\int_X \|f\|_E^p \;d\mu\right)^{1/p}
\end{equation*}
for $p \in (0,\infty)$, and
\begin{equation*}
    \|f\|_{L^\infty(X;E)} = \sup_{x \in X} \|f(x)\|_E
\end{equation*}
for a weakly measurable $f:X\rightarrow E$. We define $L^p(X;E)$ as the set
of measurable $f:X\rightarrow E$ with $\|f\|_{L^p(X;E)} < \infty$. As usual, 
we identify together functions on a measure space $(X,\Sigma,\mu)$ 
which agree $\mu$-almost everywhere.

$L^0(X;E)$ denotes the set of all ($\mu$-almost everywhere equivalence classes of)
weakly measurable functions from $X$ to $E$.

When $X$ is a set with counting measure, we denote $L^p(X;E)$ as $\ell^p(X;E)$.

Suppose $\zeta \in \Circ$. Provided that $\zeta \neq 1$, we see that $\omega(\zeta)$
is defined, and $\omega$ maps $\Circ\setminus\{1\}$ smoothly to $\Rl$. Thus for
$f \in L^0(\Rl;E)$, we can define $\tilde{f} \in L^0(\Circ;E)$
by 
\begin{equation*}
    \tilde{f} := f\circ \omega^{-1}.
\end{equation*}

Thus we can define the important operator $U:L^0(\Circ;E)\rightarrow L^0(\Rl;E)$,
\begin{equation*}
    (U f)(x) = \frac{1}{\sqrt{\pi}}\frac{(f\circ \omega^{-1})(x)}{x+i}.
\end{equation*}

It is the purpose of these notes to collate various results concerning the images
of certain subspaces of $L^0(\Circ;E)$ under $U$.

 
% Chapter 3

\chapter{Basic Properties of Hankel Operators} % Main chapter title

\label{Chapter3} % For referencing the chapter elsewhere, use \ref{Chapter1} 

\lhead{Chapter 3. \emph{Basic Properties of Hankel Operators}} % This is for the header on each page - perhaps a shortened title

%----------------------------------------------------------------------------------------


\section{Definition of a Hankel matrix}
A Hankel matrix is an infinite matrix $\{M_{j,k}\}_{j,k \geq 0}$
whose $(j,k)th$ entry depends only on $j+k$. If $a = \{a_j\}_{j\geq 0}$,
Let $M_a = \{a_{j+k}\}_{j,k\geq 0}$ be the Hankel matrix with $(j,k)$th
entry $a_{j+k}$. That is,
\begin{equation*}
    M_a = \begin{pmatrix}
        a_0 & a_1 & a_2 & a_3 \cdots\\
        a_1 & a_2 & a_3 & a_4 \cdots\\
        a_2 & a_3 & a_4 & a_5 \cdots\\
        a_3 & a_4 & a_5 & a_6 \cdots\\
        \vdots & \vdots & \vdots & \vdots 
    \end{pmatrix}.
\end{equation*}
An infinite matrix does not necessarily define an operator on $\ell^2(\Ntrl)$, 
however any infinite matrix can be identified with a linear operator 
on the dense subset $c_{00}(\Ntrl) \subset \ell^2(\Ntrl)$ of sequences
of finite support. 

For a sequence $a \in c_{00}(\Ntrl)$, and an infinite matrix $M = (M_{j,k})_{j,k\geq 0}$, 
we define $Ma \in c_{00}(\Ntrl)$ as $Ma = \{\sum_{k=0}^\infty M_{j,k}a_k\}_{j=0}^\infty$.

Hence we shall interchangeably talk about infinite matrices and linear
operators $c_{00}(\Ntrl)\rightarrow c_{00}(\Ntrl)$.

Since $c_{00}(\Ntrl)$ is dense in $\ell^2(\Ntrl)$, if an infinite matrix
considered as an operator on $c_{00}(\Ntrl)$ is bounded on $c_{00}(\Ntrl)$
in the $\ell^2$-norm, then the matrix extends uniquely to an operator on $\ell^2(\Ntrl)$.
Conversely, any operator on $c_{00}(\Ntrl)$ which extends to a bounded
operator on $\ell^2(\Ntrl)$ is bounded on $c_{00}(\Ntrl)$ in the $\ell^2$-norm,
and the extension to $\ell^2(\Ntrl)$ is unique.

Denote the inner product on $\ell^2(\Ntrl)$ 
as $(a,b) := \sum_{n=0}^\infty \overline{a_n}b_n$, which we note
is linear in the second argument.

We let $\Circ = \{\zeta \in \Cplx\;:\; |\zeta| = 1\}$. $\Circ$ is a compact
topological group, and we let $\ha$ be the normalised Haar measure (or arc 
length measure). Let $z:\Circ\rightarrow\Circ$ be the identity function. 
From now on, denote $L^p(\Circ) := L^p(\Circ,\ha)$.

Given $\varphi \in L^1(\Circ,\ha)$. For $n \in \Itgr$, the $n$th Fourier
coefficient is defined as
\begin{equation*}
    \hat{\varphi}(n) := \int_\Circ z^{-n}\varphi\;d\ha.
\end{equation*}

See \cite{meCesaro} for the elementary properties of the Fourier transform. 

It is proved in \cite{meAbel}, that for any $f \in L^1(\Circ,\ha)$, we have
\begin{equation*}
    f = \lim_{r\rightarrow 1^{-}} \sum_{n \in \Itgr} \hat{f}(n)r^n z^n.
\end{equation*}
where the limit is in the $L^1$ sense. Hence we define the Riesz projection $\Proj_+$,
for $f \in L^1(\Circ,\ha)$,
\begin{equation*}
    \Proj_+ f = \lim_{r\rightarrow 1^-} \sum_{n\geq 0} \hat{f}(n) r^n z^n.
\end{equation*}
This limit exists in $L^1$ by the dominated convergence theorem.

For $p \in [1,\infty]$, we define the $p$th Hardy space $H^p(\Circ)$
as the image of $L^p(\Circ)$ under $\Proj_+$.

A \emph{polynomial} on $\Circ$ is a finite linear combination of the \emph{monomials}
$\{z^n\}_{n \in \Itgr}$. We call the space of polynomials $P(\Circ)$.
An \emph{analytic polynomial} is a polynomial consisting only of non-negative powers
of $z$, we denote $P_+(\Circ)$ for the space of analytic polynomials. 

The Fourier transform gives a vector space isomorphism between $P(\Circ)$
and $c_{00}(\Itgr)$, and $P_A(\Circ)$ and $c_{00}(\Ntrl)$.

\section{Bounded Hankel operators}
It if of interest to determine when a Hankel operator defines
a bounded linear operator on $\ell^2(\Ntrl)$. This is answered completely by the
\emph{Nehari theorem}, which we cover now.
\begin{theorem}
\label{nehari}
    Let $a = \{a_j\}_{j=0}^\infty$ be a sequence. Then the 
    associated Hankel matrix $M_a$ defines a bounded linear operator on $\ell^2(\Ntrl)$
    if and only if there exists $\psi \in L^\infty(\Circ,\ha)$ such that
    \begin{equation*}
        \hat{\psi}(m) = a_m
    \end{equation*}
    for $m \geq 0$.
\end{theorem}
\begin{proof}
    Suppose first that there is $\psi \in L^\infty(\Circ)$
    such that $\hat{\psi}(m) = a_m$ for $m\geq 0$. 
    
    Choose $f,h \in P_A(\Circ)$, so that $\hat{f},\hat{h} \in c_{00}(\Ntrl)$.
    
    Let $g \in P_A(\Circ)$ be given by $g = \sum_{n=0}^\infty \overline{\hat{g}(n)}z^n$.
    Let $q = fg$.
    
    Then we compute,
    \begin{align*}
        (\hat{h},M_a\hat{f}) &= \sum_{j,k \geq 0} \overline{\hat{h}(j)}a_{j+k}\hat{f}(k)\\
        &= \sum_{j,k\geq 0} \overline{\hat{h}(j)}\hat{\psi}(j+k)\hat{f}(k)\\
        &= \sum_{j \geq 0} \hat{\psi}(j) \sum_{k=0}^j \hat{g}(j)\hat{f}(j-k)\\
        &= \sum_{j \geq 0} \hat{\psi}(j) \hat{q}(j)\\
        &= \int_\Circ \psi(\zeta) q(\overline{\zeta}) \;d\ha(\zeta).
    \end{align*}
    Hence,
    \begin{align*}
        |(\hat{h},M_a\hat{f})| &\leq \|\psi\|_{\infty} \|q\|_1\\
                               &\leq \|\psi\|_\infty \|f\|_2 \|h\|_2\\
                               &= \|\psi\|_\infty \|\hat{f}\|_2\|\hat{h}\|_2.
    \end{align*}
    And thus $M_a$ is bounded on $\ell^2(\Ntrl)$.
    
    Conversely, suppose that $M_a$ is bounded on $\ell^2(\Ntrl)$. 
    
    Let $\mathcal{L}$ be the linear functional on $P(\Circ)$ defined by
    \begin{equation*}
        \mathcal{L}(q) := \sum_{n\geq 0} a_n \hat{q}(n).
    \end{equation*}
    If $a \in \ell^1(\Ntrl)$, then $\mathcal{L}$ is bounded on $H^1(\Circ)$,
    since the inverse fourier transform of $a$ is in $L^\infty(\Circ)$. 
    Now let us prove in this case that $\|\mathcal{L}\| \leq \|M_a\|$. 
    
    Let $q \in H^1(\Circ)$, with $\|q\|_1 \leq 1$. Then $q = fg$
    for some $f,g \in H^2(\Circ)$ with $\|f\|_2,\|g\|_2 \leq 1$. 
    
    Then we can compute,
    \begin{align*} 
        |\mathcal{L}(q)| &= \left|\sum_{n=0}^\infty a_n \sum_{m=0}^n \hat{f}(m)\hat{g}(n-m)\right|\\
        &= \sum_{n,m\geq 0} a_{n+m} \hat{f}(n) \hat{g}(m)\\
        &= (M_a \hat{f},\overline{\hat{g}}). 
    \end{align*}
    And hence,
    \begin{equation*}
        |\mathcal{L}(q)| \leq \|M_a\|\|f\|_2\|g\|_2 \leq \|M_a\|.
    \end{equation*}
    So $\mathcal{L}(q)$ is bounded on $H^1(\Circ)$ whenever $a \in \ell^1(\Ntrl)$. 
    
    Now we consider $a$ to be an arbitrary sequence such that $M_a$ is bounded. Let $r \in (0,1)$
    and
    \begin{equation*}
        a^{(r)} = \{r^ja_j\}_{j\geq 0}.
    \end{equation*}
    Then $a^{(r)} \in \ell^1(\Ntrl)$,
    
    Now we can see that $M_{a^{(r)}} = D_r M_a D_r$, where $D_r$
    is multiplication by the sequence $\{r^j\}_{j\geq 0}$. Since $\|D_r\| \leq 1$, 
    we must have $\|M_{a^{(r)}}\| \leq \|M_a\|$. Since $a^{(r)} \in \ell^1(\Ntrl)$, 
    we have that the linear functional
    \begin{equation*}
        \mathcal{L}_r(q) := \sum_{n=0}^\infty a_nr^n \hat{q}(n)
    \end{equation*}
    is bounded on $H^1(\Circ)$, and the functionals $\{\mathcal{L}_r\}_{r \in (0,1)}$
    converge strongly to $\mathcal{L}$, and are uniformly bounded. Hence $\mathcal{L}$
    is continuous on $H^1(\Circ)$. 
    
    Now by the Hahn-Banach theorem, since $\mathcal{L}$ is a linear
    functional on the subspace $H^1(\Circ)$ which has norm bounded
    by $\|M_a\|$, it is the restriction
    of a linear functional on $L^1(\Circ)$, with norm bounded by $\|M_a\|$.
    Hence $\mathcal{L}(q) = (\psi,q)$ for some $\psi \in L^\infty(\Circ)$
    with $\|\psi\|_\infty \leq \|M_a\|$. This proves the result.    
\end{proof}

The key idea of this theorem is that it relates the sequence defining
a Hankel operator with a function on $\Circ$. We can state this in a slightly more
elegant way using the following result of Fefferman \cite{fefferman},
\begin{proposition}
    The space $\BMO(\Circ)$ is defined as the set of measurable
    functions $f$ on $\Circ$ (modulo almost-everywhere equivalence) such that
    \begin{equation*}
        \sup_I \int_I \left|f-\frac{1}{\ha(I)}\int_I f\;d\ha\right|\;d\ha < \infty
    \end{equation*}
    where $I$ is taken over all arcs in $\Circ$.
    
    Then 
    \begin{equation*}
        \BMO(\Circ) = L^\infty(\Circ)+\Proj_+L^\infty(\Circ) = \{f+\Proj_+g\;:\;f,g \in L^\infty(\Circ)\}.
    \end{equation*}
\end{proposition}

Using this description of $\BMO(\Circ)$, we can prove the following,
\begin{corollary}
    Let $a = \{a_j\}_{j=0}^\infty$ be a sequence. Then $M_a$ defines
    a bounded operator on $\ell^2(\Ntrl)$ if and only if
    \begin{equation*}
        \varphi := \sum_{n=0}^\infty a_n z^n \in \BMO(\Circ)\cap H^1(\Circ).
    \end{equation*}
\end{corollary}
\begin{proof}
    By theorem \ref{nehari}, $M_a$ is bounded if and only if $\varphi = \Proj_+\psi$
    for some $\psi \in L^\infty(\Circ)$. Hence if $M_a$ is bounded,
    then $\varphi \in \BMO(\Circ)\cap H^1(\Circ)$. 
    
    Conversely, if $\varphi \in \BMO(\Circ)\cap H^1(\Circ)$,
    then $\varphi = f+\Proj_+g$ for $f,g \in L^\infty$. Thus
    $\varphi = \Proj_+\varphi = \Proj_+(f+g)$, so $M_a$ is bounded.
\end{proof}

From now on, we are no longer interested in Hankel matrices $M_a$ defined by an arbitrary
sequence $a$, we are only interested in those matrices $M_a$ such that $a$
arises from the fourier transform of a function. 
\begin{definition}
    Let $\varphi \in H^1(\Circ)$. Let $\Gamma_\varphi$
    be the Hankel matrix with $(i,j)$th entry $\hat{\varphi}(i+j)$.
\end{definition}

\section{Finite Rank Hankel operators}
The strongest condition that we can put on a Hankel matrix $\Gamma_\varphi$
is that it is a finite rank operator on $\ell^2(\Ntrl)$. The problem
of determining $\varphi$ such that $\Gamma_\varphi$ is finite rank
was solved by Kronecker, as follows.
\begin{theorem}
\label{kronecker}
    Let $\varphi \in H^1(\Circ)$, and $\Gamma_\varphi$ be the associated Hankel
    matrix. Then $\Gamma_\varphi$ defines a bounded operator on $\ell^2(\Ntrl)$
    if and only if $\varphi$ is a rational function.
\end{theorem}
\begin{proof}
    Suppose that $\rank(\Gamma_\varphi) = n$. Then the first $n+1$
    columns of the matrix of $\Gamma_\varphi$ are linearly dependent. Let $B$
    denote the backward shift operator, $B(a_0,a_1,\ldots) := (a_1,a_2,\ldots)$
    and let $F$ be the forward shift operator, $F(a_0,a_1,\ldots) := (0,a_0,a_1,\ldots)$.
    Let $a = \hat{\varphi}$.
    
    Hence there exist complex scalars $\{c_0,c_1,\ldots,c_n\}$ not all
    equal to zero such that
    \begin{equation*}
        c_0a+c_1Ba+\cdots+c_nB^na = 0.
    \end{equation*}
    Now let $n,k \geq 0$. It is elementary that
    \begin{equation*}
        F^n B^k a = F^{n-k}a-F^{n-k}(a_0,a_1,\ldots,a_{k-1},0,0,\ldots).
    \end{equation*}
    So hence we have,
    \begin{align*}
        0 &= F^n\sum_{k=0}^n c_k B^ka \\
          &= \sum_{k=0}^n c_kF^n B^ka\\
          &= \sum_{k=0}^n c_kF^{n-k}a - p
    \end{align*}
    where $p$ is a finitely supported sequence. 
    
    Let $q = (c_n,c_{n-1},\ldots,c_0,0,0,\ldots)$. Then we have,
    \begin{equation*}
        0 = q * a - p.
    \end{equation*}
    Where the $*$ is convolution. Therefore, if we take the inverse fourier transform,
    \begin{equation*}
        \varphi\check{a} = \check{p}.
    \end{equation*}
    And hence $\varphi$ is a quotient of two polynomials.
    
    Conversely, suppose that $\varphi$ is a rational function. Suppose
    that $\varphi = p/q$, where $p,q \in P(\Circ)$. 
    Let $n = \max\{\deg p,\deg q\}$.
    If 
    \begin{equation*}
        q = \sum_{k=0}^n c_{n-k}z^k
    \end{equation*}
    then since $\varphi q = p$, we have
    \begin{equation*}
        \sum_{k=0}^n c_k F^{n-k} a = p.
    \end{equation*}
    Now multiply by $B^n$,
    \begin{align*}
        B^n\sum_{k=0}^n  c_k F^{n-k} a &= \sum_{k=0}^n c_k B^k a\\
        &= 0.
    \end{align*}
    
    Let $m \leq n$ be the largest number for which $c_m \neq 0$. Then $B^m a$
    is a linear combination of the $B^k a$ with $k \leq m-1$,
    \begin{equation*}
        B^m a = \sum_{k=0}^{m-1} d_k B^k a
    \end{equation*}
    for some coefficients $d_k$. 
    
    We now proceed by induction to show that any row is a linear combination
    of the first $n$ rows. 
    
    Let $k > m$. Then we have,
    \begin{align*}
        B^k a &= B^{k-m} B^m a\\
        &= \sum_{j=0}^{m-1} d_j B^{k-m+1}a.
    \end{align*}
    Since $k-m+j < k$, we have that the terms on the right hand side are
    linear combinations of the first $m$ rows by the inductive hypothesis.
    Hence $\rank(\Gamma_\varphi) \leq m$. 
\end{proof}

\section{Compactness of Hankel Operators}
If $\varphi \in L^1(\Circ)$, we are interested in conditions
on $\varphi$ such that $\Gamma_\varphi$ is compact. 

Our first result shows that Hankel matrices are continuous in their symbol.
\begin{proposition}
    Let $\varphi \in L^\infty(\Circ)$, then
    \begin{equation*}
        \|\Gamma_\varphi\| \leq \|\varphi\|_\infty.
    \end{equation*}
    and therefore if $\varphi \in C(\Circ)$, then $\Gamma_\varphi$ is compact.
\end{proposition}
\begin{proof}
    It was shown in the proof of theorem \ref{nehari} that if $g,f$ are sequences
    of finite support, then
    \begin{equation*}
        |(g,\Gamma_\varphi f)| \leq \|\varphi\|_\infty \|g\|_2\|f\|_2.
    \end{equation*}
    Hence $\|\Gamma_\varphi\| \leq \|\varphi\|_\infty$.
\end{proof}

\begin{proposition}
    The class $\VMO(\Circ)$ is the set of measurable functions $f:\Circ\rightarrow\Cplx$
    such that
    \begin{equation*}
        \lim_{\ha(I)\rightarrow0} \int_I \left|f-\frac{1}{\ha(I)}\int_I f\;d\ha\right|\;d\ha = 0
    \end{equation*}
    where the limit is over all arcs $I \subseteq \Circ$. 
    
    It is a result of Fefferman \cite{fefferman} that,
    \begin{equation*}
        \VMO(\Circ) = C(\Circ)+\Proj_+C(\Circ).
    \end{equation*}
\end{proposition}   

\begin{corollary}
    If $\varphi \in \VMO(\Circ)$, then $\Gamma_\varphi$ is compact.
\end{corollary}
\begin{proof}
    We must have that 
\end{proof} 

\section{Hankel Operators of Trace Class}
We recall the definition of the $\mathcal{L}^1$ norm on $\mathcal{B}(\Hilb)$,
\begin{definition}
    Let $\Hilb$ be a separable Hilbert space, and let $T \in \mathcal{B}(\Hilb)$. 
    For a non-negative integer $n$, we define the $n$th singular value,
    \begin{equation*}
        s_n(T) := \inf\{\|T-F\| \;:\;F \in \mathcal{B}(\Hilb),\rank(F) \leq n\}.
    \end{equation*}
    For $p \in (0,\infty)$, we define the $\mathcal{L}^p$ norm of $T$
    as
    \begin{equation*}
        \|T\|_p = \left(\sum_{n=0}^\infty |s_n(T)|^p\right)^{1/p}
    \end{equation*}
    with the convention that $\|T\|_p = \infty$ if the sum does not
    converge. The space $\mathcal{L}^p$ is the set of $T \in \mathcal{B}(\Hilb)$
    such that $\|T\|_p < \infty$.
\end{definition}



In particular we are interested in the case $p = 1$. We are interested
in finding conditions on a function $\varphi$ holomorphic in the unit disc
such that $\Gamma_\varphi$ is in $\mathcal{L}^1$. 

\begin{lemma}
    Let $\Hilb$ be a separable Hilbert space. Suppose
    that $x,y \in \Hilb$, and let $T$ be the rank one operator
    defined by $T(\zeta) = \langle x,\zeta\rangle y$. Then $\|T\|_1 = \|x\|\|y\|$.
\end{lemma}

The answer is provided by the \emph{Besov classes}. Many different definitions
of Besov spaces can be found, but the one of most relevance to us is given below.
\begin{definition}
    We define a sequence of polynomials $\{W_n\}_{n \in \Itgr} \subset P(\Circ)$ as follows.
    First,
    \begin{equation*}
        W_0 = z^{-1}+1+z.
    \end{equation*}
    And now for $n > 0$, we define $W_n$ by asserting that $\widehat{W}(2^n) = 1$,
    $\widehat{W}(2^{n-1}) = 0$, $\widehat{W}(2^{n+1}) = 0$, and $\widehat{W}$
    is a linear increasing function between $2^{n-1}$ and $2^n$, and a
    linear decreasing function between $2^n$ and $2^{n+1}$. We
    assert that $\widehat{W}(n)$ is symmetric in $n$, and is zero for all
    values not already defined.
    
    Now, for $p,q > 0$, and $s \geq 0$, we define the \emph{Besov class}
    $B_{pq}^s(\Circ)$ to be the space of distributions $f$ on $\Circ$ such that
    \begin{equation*}
        \sum_{n\geq 0} 2^{nsq} \|W_n*f\|_{p}^q < \infty.
    \end{equation*}  
    We shall denote $B_{pp}^s$ as $B_p^s$.
\end{definition}

In particular, we are going to prove that if $\varphi \in B_1^1(\Circ)$, then
$\Gamma_\varphi \in \mathcal{L}^1$. First, we need a lemma.
\begin{lemma}
    Let $f \in P_A(\Circ)$ be an analytic polynomial of degree at most $m$. Then,
    \begin{equation*}
        \|\Gamma_f\|_1 \leq (m+1)\|f\|_1.
    \end{equation*}
\end{lemma}
\begin{proof}
    Let $\zeta \in \Circ$, now define the following elements
    of $\ell^2(\Ntrl)$, 
    \begin{align*}
        x_\zeta(j) &= \begin{cases}
            \zeta^j,\;0 \leq j \leq m,\\
            0,\;j > m
        \end{cases}\\
        y_\zeta(j) &= \begin{cases}
            f(\zeta)\zeta^{-k},\;0\leq k \leq m,\\
            0,\;k > m.
        \end{cases}
    \end{align*}
    That is, $x_\zeta = (1,\zeta,\zeta^2,\ldots,\zeta^m,0,0,\ldots)$
    and $y_\zeta = f(\zeta)\overline{x_\zeta}$.
    
    Let $A_\zeta$ be the rank one operator, $A_\zeta(x) = (x_\zeta,x)y_\zeta$,
    so that $\|A_\zeta\|_1 = \|x_\zeta\|_2\|y_\zeta\|2 = (m+1)|f(\zeta)|$
    
    Then we have a componentwise equality of infinite matrices,
    \begin{equation*}
        \Gamma_f = \int_\Circ A_\zeta\;d\ha(\zeta).
    \end{equation*}
    Hence, $\|\Gamma_f\|_1 \leq (m+1)\|f\|_1$ by the triangle inequality.
\end{proof}
\begin{theorem}
    Let $\varphi \in B_1^1$. Then $\Gamma_\varphi \in \mathcal{L}^1$. 
\end{theorem}
\begin{proof}
    We have the following $L^\infty$-convergent sequence,
    \begin{equation*}
        \varphi = \sum_{n\geq 0} W_n*\varphi.
    \end{equation*}
    Hence,
    \begin{equation*}
        \Gamma_\varphi = \sum_{n\geq 0} \Gamma_{W_n*\varphi}.
    \end{equation*}
    So since the degree of $W_n$ is $2^{n+1}$, we have
    \begin{equation*}
        \|\Gamma_\varphi\|_1 \leq \sum_{n\geq 0} 2^{n+1}\|W_n*\varphi\|_1.
    \end{equation*}
\end{proof}

This proves the sufficiency of the condition $\varphi \in B_1^1$ so that $\Gamma_\varphi$
is trace class. The proof of the necessity of this condition is more difficult,
\begin{theorem}
    Let $\varphi$ be a function holomorphic in the unit disc. Then
    if $\Gamma_\varphi \in \mathcal{L}^1$, then $\varphi \in B_1^1$. 
\end{theorem}
\begin{proof}
    Define a pair of sequences of polynomials $\{Q_n\}_{n=0}^\infty$ as follows,
    \begin{equation*}
        \widehat{Q}_n(k) = \begin{cases}
            0, \;k \leq 2^{n-1}\\
            1-\frac{|k-2^n|}{2^{n-1}},\;2^{n-1}\leq k \leq 2^n+2^{n-1},\\
            0\;k \geq 2^n+2^{n-1}.
        \end{cases}
    \end{equation*}
    and a sequence $\{R_n\}_{n=0}^\infty$,
    \begin{equation*}
        \widehat{R}_n(k) = \begin{cases}
                0,\;k \leq 2^n\\
                1-\frac{|k-2^n-2^{n-1}|}{2^{n-1}},\;2^{n}\leq 2^{n+1}\\
                0,\;k \geq 2^{n+1}.
        \end{cases}
    \end{equation*}
    This is a decomposition of the sequence $\{W_n\}_{n=0}^\infty$, 
    given by $W_n = Q_n+\frac{1}{2}R_n$. 
    
    First we prove that
    \begin{equation*}
        \sum_{n\geq 0} 2^{2n+1}\|Q_{2n+1}*\varphi\|_1 < \infty.
    \end{equation*}
    
    To this end, we wish to construct an operator $B$
    such that
    \begin{equation*}
        \langle \Gamma_\varphi,B\rangle = \sum_{n\geq 0} 2^{2n} \|Q_{2n+1}*\varphi\|_1.
    \end{equation*}
    
    Now define the sequence of squares,
    for $n\geq 1$,
    \begin{equation*}
        S_n = [2^{2n-1},2^{2n-1}+2^{2n}-1]\times [2^{2n-1}+1,2^{2n-1}+2^{2n}].
    \end{equation*}
    Note that this sequence is pairwise disjoint.
    
    Let $\{\psi_n\}_{n=0}^\infty$ be a sequence in $L^\infty(\Circ)$, yet to be
    defined with $\|\psi_n\|_{\infty} \leq 1$. Now we
    define the matrix $B = \{B_{j,k}\}_{j,k\geq 0}$ by
    \begin{equation*}
        B_{j,k} = \begin{cases}
            \widehat{\psi}_n(j+k),(j,k) \in S_n,n\geq 1,\\
            0,(j,k) \notin \bigcup_{n\geq 1} S_n.
        \end{cases}
    \end{equation*}
    We wish to prove that $B$ is bounded, and in fact $\|B\|\leq 1$. 
    Let $\{e_n\}_{n\geq 0}$ be the standard basis for $\ell^2(\Ntrl)$,
    with $e_n(m) = \delta_{n,m}$. Define the subspaces
    \begin{align*}
        \Hilb_n &= \span\{e_j\;:\;2^{2n-1}\leq j \leq 2^{2n-1}+2^{2n}-1\},\\
        \Hilb_n' &= \span\{e_j\;:\;2^{2n-1}+1\leq j\leq 2^{2n-1}+2^{2n}\}.
    \end{align*}
    Let $P_n$ and $P_n'$ be the orthogonal projection onto $\Hilb_n$ and $\Hilb_n'$
    respectively.
    So that
    \begin{equation*}
        B = \sum_{n\geq 1} P_n'\Gamma_{\psi_n}P_n,
    \end{equation*}
    where $P_n$ and $P_n'$ are the orthogonal projections onto $\Hilb_n$
    and $\Hilb_n'$ respectively.
    
    Now since the spaces $\{\Hilb_n\}_{n\geq 1}$ are pairwise orthogonal,
    as are the spaces $\{\Hilb_n'\}_{n\geq 1}$, we have
    \begin{align*}
        \|B\| &\leq \sup_{n\geq 1} \|P_n'\Gamma_{\psi_n}P_n\|\\
         &\leq \sup_{n} \|\Gamma_{\psi_n}\|\\
         &\leq \sup_{n} \|\psi_n\|_\infty \\
         &\leq 1.
    \end{align*}
    
    Now, we compute
    \begin{align*}
        \langle \Gamma_\varphi,B \rangle &= \sum_{n\geq 1} \langle \Gamma_\varphi,P_n'\Gamma_{\psi_n}P_n\rangle\\
        &= \sum_{n\geq 1} \sum_{j=2^{2n}}^{2^{2n}+2^{2n+1}}(2^{2n}-|j-2^{2n+1}|\overline{\widehat{\varphi}}(j)\widehat{\psi_n}(j)\\
        &= \sum_{n\geq 1} 2^{2n} (Q_{2n+1}*\varphi,\psi_n).
    \end{align*}
    Now, using the sharpness of H\"older's inequality, we can choose
    a sequence $\{\psi_n\}_{n=0}^\infty$ so that $\langle Q_{2n+1}*\varphi,\psi_n\rangle$
    is arbitrarily close to $\|Q_{2n+1}*\varphi\|_1$. Hence,
    \begin{equation*}
        \sum_{n\geq 1}2^{2n+1}\|Q_{2n+1}*\varphi\|_1 = 2\langle \Gamma_\varphi,B\rangle \leq 2\|\Gamma_\varphi\|_1.
    \end{equation*}
    In exactly the same way, we may prove that
    \begin{equation*}
        \sum_{n\geq 1}2^{2n}\|Q_{2n}*\varphi\|_1 < \infty
    \end{equation*} 
    that
    \begin{equation*}
        \sum_{n\geq 0} 2^{2n+1} \|R_{2n+1}*\varphi\|_1 < \infty
    \end{equation*}
    and
    \begin{equation*}
        \sum_{n\geq 1} 2^{2n}\|R_{2n}*\varphi\|_1 < \infty
    \end{equation*}
    and therefore that $\varphi \in B_1^1$.
\end{proof} 

\begin{remark}
    We thus conclude that
    \begin{equation*}
        \frac{1}{6}\sum_{n\geq 1}2^n \|W_n*\varphi\|_1 \leq \|\Gamma_\varphi\|_1 \leq 2\sum_{n\geq 0} 2^n\|W_n*\varphi\|_1.
    \end{equation*}
\end{remark}

\section{Interpolation}
\begin{lemma}
    Let $K$ be the real interpolation functor, described in \cite{interp}. Then
    \begin{equation*}
        K(\mathcal{L}^1,\mathcal{K})_{\theta,q} = \mathcal{L}_{p,q}.
    \end{equation*} 
    for $p = (1-\theta)^{-1}$.
\end{lemma}


\begin{corollary}
    Hence, we have that $\Gamma_\varphi \in \mathcal{L}_{p,q}$
    if and only if $\varphi \in (B_1^1,\VMO)_{\theta,q}$, where $p = (1-\theta)^{-1}$.
\end{corollary}


% Chapter 3

\chapter{Abstract Differential Algebra and Spectral Triples} % Main chapter title

\label{Chapter4} % For referencing the chapter elsewhere, use \ref{Chapter1} 

\lhead{Chapter 4. \emph{Abstract Differential Algebra and Spectral Triples}} % This is for the header on each page - perhaps a shortened title

%----------------------------------------------------------------------------------------


\section{Introduction}
For several decades now, mathematicians have been attempting to find analogues
of theorems of differential and algebraic geometry in noncommutative 
algebra.
The biggest obstacle to learning this topic is that many of the definitions
were arrived at through many years of hard work, and may seem unmotivated at first.
The purpose of these notes is to explain the algebra of Connes Differentials,
which were invented by Alain Connes as a noncommutative generalisation
of the exterior algebra bundle on a manifold.

\section{Classical Differential Algebra}
Let $M$ be an $n$ dimensional manifold. The cotangent bundle $\Omega^1(M)$
is a rank $n$ vector bundle on $M$. We build higher bundles by wedge products,
\begin{equation*}
    \Omega^p(M) := \bigwedge_p \Omega^1(M).
\end{equation*}
and define $\Omega^0(M) := C^\infty(M)$. See that $\dim_\Rl(\Omega^p(M)) = \binom{n}{p}$.


The exterior algebra bundle is the direct sum of all the $\Omega^p(M)$,
\begin{equation*}
    \Omega(M) := \bigoplus_{p=0}^\infty \Omega^p(M).
\end{equation*}


$\Omega(M)$ is a \emph{graded algebra}.


In general, if $A$ is an algebra over a ring $R$, we say that $A$ is $\Ntrl$-graded
if there exists a decomposition into submodules $A^{(p)}$,
\begin{equation*}
    A = \bigoplus_{p=0}^\infty A^{(p)}
\end{equation*}
such that $A^{(n)}A^{(m)} \subseteq A^{(n+m)}$.

In the case $A = \Omega(M)$, we have $R = \Rl$, and $A^{(p)} = \Omega^p(M)$.

The exterior derivative, $d:\Omega(M)\rightarrow \Omega(M)$ acts on the grading by,
\begin{equation*}
    d:\Omega^p(M) \rightarrow \Omega^{p+1}(M).
\end{equation*}
and $d^2 = 0$. Hence we have a sequence,

\begin{equation*}
    0\rightarrow \Omega^{0}(M) \xrightarrow{d} \Omega^1(M) \xrightarrow{d} \cdots \xrightarrow{d} \Omega^n(M) \xrightarrow{d} 0.
\end{equation*}
Denote $d_p$ as the restriction of $d$ to $\Omega^p(M)$. Then we have the \emph{de Rham cohomology}
spaces,
\begin{equation*}
    H^p_{dR}(M) = \frac{\ker(d_{p})}{\im(d_{p-1})}
\end{equation*}
This is a sequence of real vector spaces, and their dimensions are topological
invariants of $M$.

The maps $d_p$ satisfy a graded version of Leibniz's rule, for $a \in \Omega^n(M)$
and $b \in \Omega^m(M)$, we have:
\begin{equation*}
    d_{n+m}(ab) = d_n(a)b+(-1)^nad_m(b)
\end{equation*}

\section{Abstract Differential Algebra}
\subsection{Graded Differential Algebras}
We now take the ideas of the previous section and move them to a more abstract setting.
Let $R$ be a commutative ring, and let $A$ be an $\Ntrl$-graded algebra over $R$, with decomposition
\begin{equation*}
    A = \bigoplus_{p=0}^\infty A^{(p)}
\end{equation*}

There is also an $R$-linear map $d:A\rightarrow A$ such that,
\begin{equation*}
    d:A^{(p)}\rightarrow A^{(p+1)}.
\end{equation*}
and $d^2 = 0$.
If we denote the restriction of $d$ to $A^{(p)}$ as $d_p$, we
require that the maps $d_p$ satisfy a graded Leibniz rule, for $a \in A^{(n)}$
and $b \in A^{(m)}$,
\begin{equation*}
    d_{n+m}(ab) = d_n(a)b+(-1)^nad_m(b).
\end{equation*}

A pair $(A,d)$ satisfying these conditions is called a \emph{differential graded algebra}.

Thus we have a sequence,
\begin{equation*}
    0\rightarrow A^{(0)} \xrightarrow{d} A^{(1)} \xrightarrow{d} \cdots \xrightarrow{d} A^{(n)} \xrightarrow{d} \cdots
\end{equation*}
The quotient $R$-modules,
\begin{equation*}
    H^p_{dR}(M) := \frac{\ker(d_{p})}{\im(d_{p-1})}.
\end{equation*}
are the de Rham cohomology modules for the graded differential algebra $(A,d)$.
\subsection{K\"ahler Differentials}
Given an $R$-algebra $A$, we would like to be able to build an algebra
of differential forms over $A$, in a manner analogous to how $\Omega^1(M)$
is constructed from $C^\infty(M)$. It turns out that there is a good way of doing
this, called the algebra of \emph{K\"ahler differentials}. This is simplest in the commutative case,
which we briefly outline here.

Let $R$ be a commutative ring, and let $A$ be a unital commutative $R$-algebra. The module
$\Omega^1_{\com}(A)$
of K\"ahler differentials is defined as
\begin{equation*}
    \Omega^1_{\com}(A) := \frac{A\otimes_R A}{\langle c\otimes(ab)-(ca)\otimes b-(bc)\otimes a\rangle}
\end{equation*}
The idea here is that $\Omega^1_{\com}(A)$ is the left $A$-module spanned by all
symbols of the form $adb$, where $d(ab) = adb+bda$. We think of $a\otimes b$
as $adb$.

More precisely, we let $d:A\rightarrow \Omega^1_{\com}(A)$ be given by
\begin{equation*}
    da := 1_A\otimes a
\end{equation*}
Where $1_A$ is the unit in $A$.

The utility of $\Omega^1_{\com}(A)$ is that it allows us to study all
derivations on $A$. 

In full abstraction, a derivation on $A$ is a map $\theta:A\rightarrow M$, where
$M$ is some left $A$-module, such that $\theta$ satisfies the leibniz rule,
\begin{equation*}
    \theta(ab) = a\theta(b)+b\theta(a).
\end{equation*}

We see that $d$ is a derivation on $A$ to the $A$-module $\Omega^1_{\com}(A)$.
It is in fact universal with this property,
\begin{theorem}
    Let $A$ be a unital commutative $R$-algebra, and let $\theta:A\rightarrow M$
    be a derivation to some left $A$-module $M$. There exists a unique $R$-linear 
    map $\Omega(\theta)$ such that the following diagram commutes,\\
    \begin{displaymath}
    \xymatrix{
        A \ar[r]^d \ar[rd]^\theta & 
        \Omega^1_{\com}(A) \ar@{.>}[d]^{\Omega(\theta)} &\\
         &
        M
    } 
  \end{displaymath}
  In other words, there is an isomorphism of $R$-modules,
  \begin{equation*}
    D(A,M) \cong \Hom_R(\Omega^1_{\com}(A),M).
  \end{equation*}
  Where $D(A,M)$ is the set of derivations from $A$ to $M$.
  Note that this universal property defines $\Omega^1_{\com}(A)$ up
  to unique isomorphism.
\end{theorem}
\begin{proof}
    Basically, $\Omega(\theta)$ maps $adb$ to $a\theta(b)$. Checking the universal property
    is routine.
\end{proof}

We would now like to create a similar algebra of differentials for a non-commutative
associative algebra $A$ over $R$. In the noncommutative case, we must restrict
attention to derivations that take values in $A$-bimodules, rather than left
$A$ modules.
\begin{definition}
    Let $A$ be an associative
    unital algebra over a commutative ring $R$. 
    Let $m:A\otimes A\rightarrow A$ be the multiplication map. We define
    \begin{equation*}
        \Omega^1(A) = \ker(m)
    \end{equation*}
    This is an $A$ bimodule.
\end{definition}
This the motivation behind this definition is not at all clear. However, this
does agree with the commutative case and this provides the appropriate definition
for noncommutative K\"ahler differentials. To see this, we define the map
$d:A\rightarrow \Omega^1(A)$, by
\begin{equation*}
    d(a) = 1_A\otimes a-a\otimes 1_A.
\end{equation*}
We see that $d$ is a derivation. In fact, $\Omega^1(A)$
should be thought of as the space of all linear combinations
of terms of the form $ad(b)$.


$\Omega^1(A)$ satisfies the 
same universal property as $\Omega^1_{\com}$. Namely, if $M$ is an $A$-bimodule,
and $\theta:A\rightarrow M$ is a derivation, then there exists a unique $R$-linear
map $\Omega(\theta)$ such that the following diagram commutes,

    \begin{displaymath}
    \xymatrix{
        A \ar[r]^d \ar[rd]^\theta & 
        \Omega^1(A) \ar@{.>}[d]^{\Omega(\theta)} &\\
         &
        M
  } 
  \end{displaymath}
  
\subsection{Universal Differential Algebra}
Given an associative unital algebra $R$ over a commutative ring $R$,
we define
\begin{equation*}
    \Omega^p(A) := \bigotimes_{A,p} \Omega^1(A).
\end{equation*} 
And the algebra,
\begin{equation*}
    \Omega A = \bigoplus_{p} \Omega^p(A).
\end{equation*}
We extend the function $d:A\rightarrow \Omega^1(A)$ to $\Omega A$
by
\begin{equation*}
    d(ada_1da_2\cdots da_n) = dada_1da_2\cdots da_n.
\end{equation*}
 
\begin{theorem}
    $\Omega A$ is the ``largest" graded differential algebra generated by $A$. 
    
    If $(\Gamma,\Delta)$ is a graded differential algebra, with grading $\Gamma = \bigoplus_n \Gamma^{(n)}$,
    and $\rho:A\rightarrow \Gamma^{(0)}$ is an algebra homomorphism, then $\rho$
    extends uniquely to a morphism $\Omega A\rightarrow \Gamma$ such that the following
    diagram commutes,
    \begin{displaymath}
    \xymatrix{
        \Omega^p(A) \ar[r]^\rho \ar[d]^{d} & 
        \Gamma^{(p)} \ar[d]^\Delta&\\
        \Omega^{p+1}(A) \ar[r]^\rho & 
        \Gamma^{(p+1)}&
    }
    \end{displaymath}
\end{theorem}

\section*{Non-commutative Geometry}
\subsection{Spectral Triples}
Non-commutative geometry is an attempt to generalise theorems of differential geometry
to non-commutative algebras. The basic object of study in non-commutative differential
geometry is the \emph{spectral triple}. A spectral triple plays the role of a 
``manifold". The definition is as follows.
\begin{definition}
    A spectral triple is a triple, $(\A,\Hilb,\D)$. Where,
    $\A$ is a $*$-algebra of bounded operators on a Hilbert space $\Hilb$, and
    $\D$ is a densely defined unbounded self adjoint operator on $\Hilb$, such
    that $[\D,a] \in \mathcal{B}(\Hilb)$ for all $a \in \A$ and $(\D-\lambda)^{-1}$
    is compact for all $\lambda \in \Cplx\setminus \Rl$.
\end{definition}

This is supposed to be an abstract version of a spin manifold. The commutative case
is when $M$ is a Riemannian manifold with a spin bundle $S$, then we have $\A = C^\infty(M)$,
acting by pointwise multiplication on the bundle $\Hilb = L^2(S)$, and $\D$
is the dirac operator.

It can be shown, under fairly general conditions, that when $\A$ is commutative, then 
such a spin manifold can be constructed. Hence a spectral triple deserves
the name of ``non-commutative manifold".
\subsection{Connes Differentials}
Given a spectral triple, $(\A,\Hilb,\D)$, we would like to construct an ``exterior algebra"
on $\A$. Connes does this by identifying the $1$-form $da$ with $[D,a]$.

Since $[D,a]$ is a derivation on $\A$, by the universal property we have a map, $\pi:\Omega\A\rightarrow \mathcal{B}(\Hilb)$
given by $\pi(ada_1da_2\cdots da_n) = a[D,a_1][D,a_2]\cdots[D,a_n]$.

One may then na\"ively define the algebra of differential forms as $\pi(\Omega \A)$,
but this does not work since there exists $a \in \Omega \A$ such that $\pi(a) = 0$
but $\pi(da) \neq 0$. These are called ``junk forms" and we must factor them out to get
a good differential algebra. Hence, define
\begin{theorem}
    Let $J_0$ be the graded ideal of $\Omega \A$ defined by 
    \begin{equation*}
        J_0^{(p)} = \{a \in \Omega^p(\A) \;:\; \pi(a) = 0\}
    \end{equation*}
    And define $J^{(p)} = J_0^{(p)} + dJ_0^{(p)}$. Then $J = \bigoplus_p J^{(p)}$.
\end{theorem}

Now we can define the algebra of Connes' forms,
\begin{equation*}
    \Omega_{\D}\A = \frac{\Omega \A}{J} \cong \frac{\pi(\Omega \A)}{\pi(dJ_0)}
\end{equation*}

$\Omega_\D \A$ is naturally graded by the gradings on $\Omega \A$ and $J$, with the 
space of $p$-forms being $\Omega^p_\D \A = \Omega^p(\A)/J^{(p)}$.

Since $J$ is a differential ideal, the operator $d$ on $\Omega \A$
extends to $\Omega_\D \A$.  



 
% Chapter 1

\chapter{Weak $\mathcal{L}^p$ membership of quantised differentials} % Main chapter title

\label{Chapter5} % For referencing the chapter elsewhere, use \ref{Chapter1} 

\lhead{Chapter 5. \emph{Weak $\mathcal{L}^p$ membership of quantised differentials}} % This is for the header on each page - perhaps a shortened title

%----------------------------------------------------------------------------------------


\section{Introduction}
We have already given a thorough description of the quantised differentials $\qd f$
when $f$ is a function on the circle. It so happens that we can find
analogous, although weaker, results in far greater generality.


\section{The setting}
We shall consider spectral triples $(\A,\Hilb,\D)$
such that $\A$ is contianed in a semi-finite Von Neumann
algebra $\N$ which is contained in $\mathcal{B}(\Hilb)$.

From now on, fix a semifinite Von Neumann algebra $\N$ with a faithful
normal trace $\tau$. 



Formally,
\begin{definition}[Spectral Triple]
    A (semifinite) spectral triple is a triple $(\A,\Hilb,\D)$, where
    $\Hilb$ is a Hilbert space with $\N \subseteq \mathcal{B}(\Hilb)$,
    and $\A \subseteq \N$ is a $*$-algebra.
    
    $\D$ is a densely defined unbounded operator on $\Hilb$ satisfying the follownig
    two properties:
    \begin{enumerate}
        \item{} $[\D,a]$ is densely defined and extends to a bounded operator in $\N$
        for all $a \in \A$.
        \item{} $(\lambda -\D)^{-1}$ is $\tau$-compact for all $\lambda \in \Cplx\setminus \Rl$.
    \end{enumerate}
    
    A spectral triple $(\A,\Hilb,\D)$ can be either \emph{even} or \emph{odd}:
    \begin{itemize}
        \item{} We say that $(\A,\Hilb,\D)$ is even if there exists a
        $\Itgr/2\Itgr$ grading on the linear operators on $\Hilb$ such that $\A$
        is even and $\D$ is odd. Equivalently, there is an operator $\Gamma$
        on $\Hilb$ with $\Gamma^2 = 1$ and $\Gamma^* = \Gamma$ such that $a\Gamma = \Gamma a$ for all $a \in \A$
        and $\D\Gamma=-\Gamma\D$. 
        \item{} If no such operator $\Gamma$ exists, then we say that $(\A,\Hilb,\D)$
        is odd.
    \end{itemize}
\end{definition}

To define the ideal $\mathcal{L}^{p,\infty}$ we need to first
define the generalised singular values,
\begin{definition}[Generalised Singular Values]
    Let $T \in \N$, and $t \in [0,\infty)$. We define
    \begin{equation}
        \mu(t,T) := \inf\{\|TE\|\;:\;E\text{ is a projection, with }\tau(1-E) \leq t\}.
    \end{equation}
\end{definition}

\begin{definition}[$\mathcal{L}^{p,\infty}$ ideal]
    Let $\Psi_p:[0,\infty)\rightarrow[0,\infty)$ for $p > 0$ be defined as
    \begin{equation}
        \Psi(t) := \begin{cases}
            t,\text{ for }t \in [0,1]\\
            t^{1-\frac{1}{p}}\text{ for }t \geq 1.
        \end{cases}
    \end{equation}
    Now define the subset $\mathcal{L}^{p,\infty} \subseteq \N$ as
    \begin{equation}
        \mathcal{L}^{p,\infty} = \left\{T \in \N\;:\;\|T\|_{p,\infty} := \sup_{t > 0} \frac{1}{\Psi_p(t)}\int_0^t \mu(t,T)\;dt < \infty\right\}.
    \end{equation}
\end{definition}

\begin{definition}[Quantum Differentiability]
    A spectral triple $(\A,\Hilb,\D)$ is called $QC^k$ for $k \geq 0$
    if $\A$ is contained in the domain of the operator $\delta^k$, where $\delta(a) = [|\D|,a]$.
\end{definition}

\begin{definition}[Summability]
    A spectral triple $(\A,\Hilb,\D)$ is called $(p,\infty)$ summable
    for $p > 0$ if $(1+\D^2)^{-1/2} \in \mathcal{L}^{p,\infty}$.
\end{definition}

    
    
\section{The Theorem}

\begin{theorem}
    Let $k\geq 0$ and $p \geq 1$, and
    let $(\A,\Hilb,\D)$ be a $(p,\infty)$ summable $QC^k$ spectral triple such that $\D$
    is invertible. Let $F = \sgn(\D)$ such that $\D = F|\D|$. Then for all $a \in \A$,
    \begin{equation}
        [F,a] \in \mathcal{L}^{p,\infty}.
    \end{equation}
\end{theorem}
\begin{proof}
    Since $\D = |\D|F$, we can compute
    \begin{equation}
        [\D,a] = |\D|[F,a] + [|\D|,a]F.
    \end{equation}
    Or in other words,
    \begin{equation}
        da = |\D|\qd a + \delta(a)F.
    \end{equation}
    Since $\D$ is invertible, $|\D|$ is invertible. Thus,
    \begin{equation*}
        \qd a = |\D|^{-1}da + |\D|^{-1}\delta(a)F.
    \end{equation*}
    By assumption, $da,\delta(a) \in \mathcal{B}(\Hilb)$. Thus,
    since $(\A,\Hilb,\D)$ is $(p,\infty)$-summable, $|\D|^{-1} \in \mathcal{L}^{p,\infty}$.
    Hence, $\qd a \in \mathcal{L}^{p,\infty}$.

%    By the continuous functional calculus, we have the formula,
%    \begin{equation}
%        |\D|^{-1} = \frac{1}{\pi}\int_{0}^\infty \frac{(\lambda+\D^2)^{-1}}{\sqrt{\lambda}}\;d\lambda.
%    \end{equation}
%    
%    Let $a \in \A$,
%    Since by definition $F = \D|\D|^{-1}$,
%    \begin{align*}
%        [F,a] &= [\D|\D|^{-1},a]\\
%              &= [\D,a]|\D|^{-1}+\D[|\D|^{-1},a].
%    \end{align*}
%    Now we use the integral representation of $|\D|^{-1}$,
%    \begin{equation}
%        [F,a] = \frac{1}{\pi}\int_0^\infty ([\D,a](\lambda+\D^2)^{-1}+\D[(\lambda+\D^2)^{-1},a])\frac{d\lambda}{\sqrt{\lambda}}.
%    \end{equation}
%    Using the fact that $\D^2(\lambda+\D^2)^{-1} = 1-\lambda(\lambda+\D^2)^{-1}$,
%    we get
%    \begin{equation}
%        [F,a] = \frac{1}{\pi}\int_0^\infty (\lambda(\lambda+\D^2)^{-1}[\D,a](\lambda+\D^2)^{-1}-\D(\lambda+\D^2)^{-1}[\D,a]\D(\lambda+\D^2)^{-1})\frac{d\lambda}%{\sqrt{\lambda}}.
%    \end{equation}
%    
%    Now assume that $a^* = -a$, so that $[\D,a]^* = [\D,a]$ and $[F,a]^* = [F,a]$.
%    
%    Now for any $T \in \N$, we have
%    \begin{equation}
%        -\|[\D,a]\|T^*T \leq T^*[\D,a]T \leq \|[\D,a]\|T^*T.
%    \end{equation}
%    Hence,
%    \begin{align*}
%        [F,a] \leq \frac{\|[\D,a]\|}{\pi} \int_0^\infty (\lambda+\D^2)^{-1} \frac{d\lambda}{\sqrt{\lambda}}
%        &= \|[\D,a]\||\D|^{-1}.
%    \end{align*}
%    Similarly,
%    \begin{equation}
%        [F,a] \geq -\|[\D,a]\||\D|^{-1}.
%    \end{equation} 
%    
%    Thus since $|\D|^{-1} \in \mathcal{L}^{p,\infty}$, 
%    we have $[F,a] \in \mathcal{L}^{p,\infty}$ for $a^* = -a$,
%    which we extend to all $a \in \A$ by linearity.
\end{proof} 
 
% Chapter 1

\chapter{The Chain Rule in Quantised Calculus} % Main chapter title

\label{Chapter6} % For referencing the chapter elsewhere, use \ref{Chapter1} 

\lhead{Chapter 6. \emph{The Chain Rule in Quantised Calculus}} % This is for the header on each page - perhaps a shortened title

%----------------------------------------------------------------------------------------

\section{Introduction}
In classical analysis, the chain rule states that if $M$,$N$ and $P$ are manifolds, 
 and the maps $f:M\rightarrow N$ and $g:N\rightarrow P$ are smooth, then
\begin{equation*}
    d(f\circ g) = df \circ dg.
\end{equation*}
Where both sides of the equation are maps $TM\rightarrow TP$.

\begin{remark}
    The chain rule is simply a consequence of the functoriality of the tangent
    bundle construction, $M\mapsto TM$.
\end{remark}

 It is desirable
to find a noncommutative generalisation of this identity.


\section{The setting}
We let $\mathcal{L}^{p,\infty}_0$ be the closure of the finite rank
operators in the $\mathcal{L}^{p,\infty}$-metric topology.

Let $(\A,\Hilb,\D)$ be a spectral triple, satisfying the following properties:
\begin{enumerate}
    \item{} For all $a \in \A$, $\qd a \in \mathcal{L}^{p,\infty}$.
    \item{} $\A$ is closed under the holomorphic functional calculus.
    \item{} Let $\A_0 \subseteq \mathcal{B}(\Hilb)$ be the collection of all
    $T$ such that $[\sgn(\D),T]$ is finite rank. Then $\A$ is contained
    within the norm-closure of $\A_0$.
\end{enumerate}

\section{The Commutator Lemma}
\begin{lemma}
    Let $(\A,\Hilb,\D)$ be as above, and $a \in \A$. Then
    \begin{equation*}
        [\qd a,a] \in \mathcal{L}^{p,\infty}_0.
    \end{equation*}
\end{lemma}
\begin{proof}
    By our assumptions on $(\A,\Hilb,\D)$ there exists a sequence $\{a_n\}_{n=0}^\infty$
    with $\|a_n-a\| \rightarrow 0$, and 
    $\qd a_n$ is finite rank. Hence $[\qd a_n,a]$ is finite rank, and
    \begin{equation*}
        [\qd a,a_n] = \qd([a_n,a]) + [\qd a,a_n].
    \end{equation*}
\end{proof}
 
% Chapter 1

\chapter{Higher Dimensions} % Main chapter title

\label{Chapter7} % For referencing the chapter elsewhere, use \ref{Chapter1} 

\lhead{Chapter 7. \emph{Higher Dimensions}} % This is for the header on each page - perhaps a shortened title

%----------------------------------------------------------------------------------------
\section{Introduction}

Let $d > 0$ be a positive integer. 
Let $\A_{\lambda}$ be the $d$-dimensional non-commutative torus, that is the universal $C^*$-algebra
generated by unitaries $\{U_j\}_{j=1}^d$ subject to the commutation
relation $U_j U_k = \lambda_{jk} U_k U_j$ for a unimodular
scalar constant $\lambda_{jk}$, where we assume that $\lambda_{jj} = 1$
for all $j$, and $\lambda_{jk} = \lambda_{kj}^{-1}$.

$\A_\lambda$ has a unique normalised trace $\tau$.
The Hilbert space $L^2(\A_\lambda,\tau)$ 
has orthonormal basis $\{u(n)\}_{n \in \Itgr^d}$ where $u(n) = U_1^{n_1}U_2^{n_2}\cdots U_d^{n_d}$.

Define $\omega_{j,k}$ for $j,k \in \Itgr^d$ to be such
that $u(j)u(k) = \omega_{j,k}u(k+j)$.

Given $f \in L^2(\A_\lambda,\tau)$, we define
\begin{equation}
    \hat{f}(n) := \tau(fu(n)^*).
\end{equation}
Then we have
\begin{equation}
    f = \sum_{k \in \Itgr^d} \hat{f}(k)u(k),
\end{equation}
which converges in the $L^2$-sense. Note that $\tau(f) = \hat{f}(0)$.

For a function $F:\Itgr^d\rightarrow \Rl$, define $F(D) u(k) = F(k)u(k)$
as a densely defined linear operator on $L^2(\A_\lambda,\tau)$. 

It is easy to see that $F(D)$ is bounded on $L^2(\A_\lambda,\tau)$
if and only if $F \in \ell^\infty(\Itgr^d)$, with operator norm
$\|F\|_\infty$.

We shall identify $u(n) \in \A_\lambda$ with the multiplication operator
$M_{u(n)}f = u(n)f$ for $f \in L^2(\A_\lambda,\tau)$

\section{Commutators}
Let $n \in \Itgr^d$. We are interested in $[F(D),u(n)]$. 

Let $f \in L^2(\A_\lambda,\tau)$. 

Then,
\begin{align}
    u(n)f &= \sum_{k \in \Itgr^d} \hat{f}(k)u(n)u(k)\\
          &= \sum_{k \in \Itgr^d} \hat{f}(k)\omega_{n,k} u(n+k).
\end{align}
Hence,
\begin{equation}
    F(D) u(n)f = \sum_{k \in \Itgr^d} \hat{f}(k) \omega_{n,k} F(n+k) u(n+k).
\end{equation}

Now we compute,
\begin{equation}
    u(n)F(D)f = \sum_{k \in \Itgr^d} \hat{f}(k) F(k) \omega_{n,k} u(n+k).
\end{equation}

Thus,
\begin{equation}
    [F(D),u(n)]f = \sum_{k \in \Itgr^d} \hat{f}(k) (F(n+k)-F(k))\omega_{n,k}u(n+k).
\end{equation}

We can simplify this further,
\begin{align}
    [F(D),u(n)]f &= \sum_{k \in \Itgr^d} \hat{f}(k) (F(n+k)-F(k))u(n)u(k)\\
                 &= u(n)\sum_{k \in \Itgr^d}  \hat{f}(k) F(n+k)u(k) - u(n) \sum_{k \in \Itgr^d} \hat{f}(k)F(k)u(k)\\
                 &= u(n) F(n+D)f - u(n)F(D)f\\
                 \label{commutatorFormula}
                 &= u(n)F(n+D)f - u(n)F(D)f
\end{align}
So $[F(D),u(n)] = u(n)(F(n+D)-F(D))$. 

\section{Hilbert-Schmidt Commutators}
It is of interest to ask when $[F(D),u(n)] \in \mathcal{L}^2$. This is equivalent to,
\begin{equation}
    \|[F(D),u(n)]\|_{2}^2 := \sum_{j,k \in \Itgr^d} |\tau(u(j)^* [F(D),u(n)]u(k))|^2 < \infty.
\end{equation}

So we compute,
\begin{align}
    [F(D),u(n)]u(k) &= u(n)F(n+D)u(k) - u(n)F(D)u(k)\\
                    &= u(n)F(n+k)u(k) - u(n)F(k)u(k)\\
                    &= (F(n+k)-F(k))u(n)u(k).
\end{align}
Hence,
\begin{align}
    |\tau(u(j)^*[F(D),u(n)]u(k))| &= |F(n+k)-F(k)||\tau(u(j)^*u(n)u(k))|\\
                                  &= |F(n+k)-F(k)|\delta_{j,n+k}\\
\end{align}

So therefore
\begin{align}
    \|[F(D),u(n)]\|_{2}^2 &= \sum_{j,k \in \Itgr^d} |F(n+k)-F(k)|^2 \delta_{j,n+k}\\
                          &= \sum_{k \in \Itgr^d} |F(n+k)-F(k)|^2.
\end{align}

So we have necessary and sufficient conditions for $[F(D),u(n)] \in \mathcal{L}^2$. 

\section{Membership of $\mathcal{L}^p$}
We can see that since $F(D)$ and $F(n+D)$ are self adjoint, 
\begin{equation}
    [F(D),u(n)]^* = (F(n+D)-F(D))u(n)^*
\end{equation}
Hence,
\begin{equation}
    [F(D),u(n)]^*[F(D),u(n)] = (F(n+D)-F(D))^2.
\end{equation}
Hence,
\begin{equation}
    |[F(D),u(n)]| = |F(n+D)-F(D)|.
\end{equation}

Now,
\begin{align}
    \operatorname{Tr}(|[F(D),u(n)]|^p) &= \sum_{k \in \Itgr^d} \tau(u(k)^*|F(D+n)-F(D)|^pu(k))\\
    &= \sum_{k \in \Itgr^d} \tau(u(k)^*|F(k+n)-F(k)|^pu(k))\\
    &= \sum_{k \in \Itgr^d} |F(k+n)-F(k)|^p.
\end{align}

Hence, for any $p > 0$, we have $[F(D),u(n)] \in \mathcal{L}^p$ if and only if
$\{F(k+n)-F(k)\} \in \ell^p(\Itgr^d)$ with an equality of (quasi-)norms.


\section{Double commutators}

We are also interested in the properties of the double commutators,.
\begin{equation*}
    [[F(D),u(n)],u(n)]
\end{equation*}
for $n \in \Itgr^d$. By equation \ref{commutatorFormula}, this is
\begin{align}
    [[F(D),u(n)],u(n)] &= [F(D),u(n)]u(n) - u(n)[F(D),u(n)] \\
                       &= u(n)(F(n+D)-F(D))u(n)-u(n)u(n)(F(D+n)-F(D))\\
                       &= (F(D)-F(D-n))u(n)^2-(F(D-n)-F(D-2n))u(n)^2\\
                       &= (F(D-2n)-2F(D-n)+F(D))u(n)^2\\
                       &= u(n)(F(D+n)-2F(D)+F(D-n))u(n).
\end{align}
Hence,
\begin{equation*}
    |[[F(D),u(n)],u(n)]| = |F(D+2n)-2F(D+n)+F(D)|.
\end{equation*}

So we have $[[F(D),u(n)],u(n)] \in \mathcal{L}^p$
if and only if $\{F(k+n)-2F(k)+F(k-n)\}_{k\in \Itgr^d} \in \ell^p(\Itgr^d)$.



 

%----------------------------------------------------------------------------------------
%	THESIS CONTENT - APPENDICES
%----------------------------------------------------------------------------------------

\addtocontents{toc}{\vspace{2em}} % Add a gap in the Contents, for aesthetics

\appendix % Cue to tell LaTeX that the following 'chapters' are Appendices

% Include the appendices of the thesis as separate files from the Appendices folder
% Uncomment the lines as you write the Appendices

% Appendix A

\chapter{Classical Harmonic Analysis} % Main appendix title

\label{AppendixA} % For referencing this appendix elsewhere, use \ref{AppendixA}

\lhead{Appendix A. \emph{Classical Harmonic Analysis}} % This is for the header on each page - perhaps a shortened title


\section{Introduction}
Given a function $f:\Circ\rightarrow\Cplx$, we have an associated fourier series, 
\begin{equation}
\label{series}
    f \sim \sum_{n \in \Itgr} \hat{f}(n)z^n
\end{equation}
where $z:\Circ\rightarrow\Circ$ is the identity function, and
\begin{equation}
\label{coefficient}
    \hat{f}(n) = \int_\Circ z^{-n} f\;d\ha
\end{equation}
where $\ha$ is the normalised Haar (or arc length) measure on $\Circ$. Implicitly
$f$ is sufficiently regular so that these fourier coefficients exist.

I have used the symbol ``$\sim$" rather than an equals sign, since in general
we do not have equality. In general, for $f \in L^1(\Circ,\ha)$, the fourier
series might diverge almost everywhere if it is interpreted as a sum.

However, if one turns to alternative methods of summation, it is possible to 
interpret $\sim$ as an equality. 

We have already explored in a previous set of notes the concept
of Ces\`aro summation. Here we consider Abel summation.

\section{Abel summation}
Abel summation is inspired by Abel's theorem, which we prove now.
\begin{proposition}
    Suppose that
    \begin{equation*}
        f(z) = \sum_{n=0}^\infty a_n z^n
    \end{equation*}
    is a power series converging in $\{z \in \Cplx\;:\;|z| < 1\}$
    such that the coefficients come from a Banach space $X$. Suppose further that the sum
    \begin{equation*}
        \sum_{n=0}^\infty a_n
    \end{equation*}
    converges. Then,
    \begin{equation*}
        \lim_{z\rightarrow 1^-} f(z) = \sum_{n=0}^\infty a_n.
    \end{equation*}
    where by $z\rightarrow 1^-$, we mean that $z$ is restricted
    to the subset of the unit disc where $|1-z| \leq M(1-|z|)$
    for some constant $M$.
\end{proposition}
\begin{proof}
    Assume without loss of generality that 
    \begin{equation*}
        \sum_{n=0}^\infty a_n = 0.
    \end{equation*}
    Now define
    \begin{equation*}
        s_k = \sum_{n=0}^k a_n
    \end{equation*}
    and $s_{-1} = 0$.
    Then we have
    \begin{equation*}
        f(z) = \sum_{n=0}^\infty (s_n-s_{n-1})z^n.
    \end{equation*}
    so
    \begin{equation*}
        f(z) = (1-z)\sum_{k=0}^\infty s_k z^k.
    \end{equation*}
    
    Let $\varepsilon > 0$, and choose $n$ large enough such that $\|s_k\| < \varepsilon$
    for $k > n$. Then we have
    \begin{equation*}
        \left\|(1-z)\sum_{k=n}^\infty s_k z^k\right\| \leq \varepsilon|1-z|\sum_{k=n}^\infty |z|^k = \varepsilon|1-z| \frac{|z|^n}{1-|z|} \leq M \varepsilon.
    \end{equation*}
    
    When $z$ is suffciently close to $1$, we have
    \begin{equation*}
        \left\|(1-z)\sum_{k=0}^{n-1} s_k z^k\right\| < \varepsilon.
    \end{equation*}
    Hence, for $z$ sufficiently close to $1$, we have
    \begin{equation*}
        \|f(z)\| < (M+1)\varepsilon.
    \end{equation*}
\end{proof}
With this in mind, we define the Abel summation method.
\begin{definition}
    Let $\{a_k\}_{k=0}^\infty \subset X$ be a sequence in a Banach space $X$. Suppose that for 
    all $r \in (1-\varepsilon,1)$, for some $\varepsilon > 0$ we have
    \begin{equation*}
        \sum_{k=0}^\infty a_k r^k
    \end{equation*}
    exists.
    Then we define the Abel sum, 
    denoted by,
    \begin{equation*}
        A-\sum_{k=0}^\infty a_k
    \end{equation*}
    as
    \begin{equation*}
        A-\sum_{k=0}^\infty a_k := \lim_{r\rightarrow 1^-} \sum_{k=0}^\infty a_k r^k.
    \end{equation*}
    if this limit exists.
\end{definition}
Abel's theorem automatically implies that the Abel sum of a series
agrees with the usual sum, however there are series which are summable
in the Abel sense but not the classical sense. It is easy to see that,
\begin{equation*}
    A-\sum_{n=0}^\infty (-1)^n = \lim_{r\rightarrow 1^-} \sum_{n=0}^\infty (-r)^n = \lim_{r\rightarrow 1^-} \frac{1}{1+r} = \frac{1}{2}.
\end{equation*}

\section{The Poisson Kernel}
To sum a Fourier series in the classical sense, we use the Dirichlet kernel,
and we use the Fej\'er kernel for Ces\`aro summation. For Abel summation,
we use the Poisson kernel.

Given $f \in L^1(\Circ,\ha)$, define
\begin{equation*}
    A_r f := \sum_{n \in \Itgr} r^{|n|} \hat{f}(n) z^n.
\end{equation*}
with $r \in (0,1)$.
$A_r f$ exists since the fourier coefficients of $f$ are bounded. By definition,
\begin{equation*}
    A-\sum_{n\in \Itgr} \hat{f}(n) z^n = \lim_{r\rightarrow 1^-} A_r f.
\end{equation*}
where the limit is taken in an appropriate Banach space.

Like with classical and Ces\`aro sums, Abel sums can be constructed with a convolution.
\begin{proposition}
    We can write,
    \begin{equation*}
        A_r f = P_r * f.
    \end{equation*}
    where
    \begin{equation*}
        P_r = 1+\frac{rz}{1-rz} + \frac{r}{z-r}
    \end{equation*}
\end{proposition}
\begin{proof}
    The Fourier coefficients of $A_r f$ are the Fourier coefficients of
    $f$ multiplied by the coefficients of
    \begin{equation*}
        \sum_{n \in \Itgr} r^{|n|} z^n.
    \end{equation*}
    Hence define
    \begin{equation*}
        P_r = \sum_{n \in \Itgr} r^{|n|} z^n.
    \end{equation*}
    The result follows from summing this geometric series.
\end{proof}

The reason for the superiority of Abel summation
over classical summation is that the Poisson kernels form an approximate identity.
\begin{proposition}
    The Poisson kernels $\{P_r\}_{r \in (0,1)}$ form an approximate
    identity.
\end{proposition}
\begin{proof}
    By the formula,
    \begin{equation*}
        P_r = \sum_{n \in \Itgr} r^{|n|} z^n,
    \end{equation*}
    we have
    \begin{equation*}
        \int_\Circ P_r\;d\ha = 1.
    \end{equation*}
    
    To complete the proof, we convert to coordinates. Let $z = \exp(2\pi i\theta)$,
    for $\theta \in (-1/2,1/2)$, and we can regard $P_r$ as a function of $\theta$.
    
    Then we have
    \begin{align*}
        P_r(\theta) &= 1 + \frac{re^{2\pi i \theta}}{1-re^{2\pi i \theta}}+\frac{re^{-2\pi i \theta}}{1-re^{-2\pi i \theta}}\\
        &= 1 + \frac{re^{2\pi i \theta}(1-re^{-2\pi i \theta})+re^{-2\pi i\theta}(1-re^{2\pi i \theta})}{(1-re^{2\pi i \theta})(1-re^{-2\pi i \theta})}\\
        &= 1 + \frac{2r\cos(2\pi \theta)-2r^2}{1+r^2-2r\cos(2\pi \theta)}\\
        &= \frac{1-r^2}{1-2r\cos(2\pi\theta)+r^2}
    \end{align*}
    Let $\delta > 0$. 
    Now we estimate,
    \begin{equation*}
        \int_{\delta}^{1/2} P_r(\theta)\; d\theta \leq \int_\delta^{1/2} \frac{1-r^2}{1-2r\cos(2\pi \delta)+r^2}\;d\theta \leq \frac{1-r^2}{1-2r\cos(2\pi \delta)+r^2}
    \end{equation*}
    Hence this integral vanishes as $r\rightarrow\infty$.
    
    Moreover, we have
    \begin{equation*}
        1-2r\cos(2\pi \theta) + r^2 \geq (1-r)^2.
    \end{equation*}
    Hence $P_r \geq 0$. Therefore, $\|P_r\|_1 = 1$.
    
    Thus the Poisson kernels form an approximate identity.
\end{proof}
As a consequence of this, we have
\begin{enumerate}
    \item{} If $f \in C(\Circ)$, then $A_r f\rightarrow f$ uniformly.
    \item{} If $f \in L^p(\Circ,\ha)$, for $1\leq p < \infty$, we have $A_rf\rightarrow f$
    in the $L^p$ sense.
\end{enumerate}
Hence, if $f \in L^1(\Circ,\ha)$, we have
\begin{equation*}
    f(\zeta) = A-\sum_{n \in \Itgr} \hat{f}(n)\zeta^n
\end{equation*}
for almost all $\zeta \in \Circ$.

\section{Fourier Series of Measures}
Suppose $\mu$ is a complex Borel regular measure on $\Circ$. Associated
to $\mu$ is a fourier series,
\begin{equation*}
    \mu \sim \sum_{n \in \Itgr} \hat{\mu}(n) z^n
\end{equation*}
where
\begin{equation*}
    \hat{\mu}(n) = \int_\Circ z^{-n} \;d\mu.
\end{equation*}
It was proved in ``Ces\`aro convergence of Fourier series" that this series
can be interpreted as a sequence of measures converging in the Ces\`aro sense
and the weak* topology to $\mu$. Here we wish to analyse the Abel sums of the 
Fourier series of $\mu$. 
Define for $r \in (0,1)$,
\begin{equation*}
    A_r \mu = \sum_{n \in \Itgr} r^{|n|} \hat{\mu}(n) z^n.
\end{equation*}
This series converges because $\mu$ is bounded, so the Fourier coefficients 
are bounded.

Recall that if $f \in C(\Circ)$, we define $f*\mu$ as the function
\begin{equation*}
    (f*\mu)(\zeta) := \int_\Circ f(\zeta \tau^{-1})d\mu(\tau).
\end{equation*}
This allows us to compute $A_r\mu$ for a measure $\mu$:
\begin{proposition}
    If $\mu$ is a complex Borel regular measure on $\Circ$, then
    \begin{equation*}
        A_r \mu = P_r * \mu.
    \end{equation*}
\end{proposition}
\begin{proof}
    We compute,
    \begin{align*}
        (P_r * \mu)(\zeta) &= \int_{\Circ} \sum_{n \in \Itgr} r^{|n|} \zeta^n \tau^{-n} d\mu(\tau)\\
        &= \sum_{n \in \Itgr} \zeta^nr^{|n|}\int_\Circ \tau^{-n} d\mu(\tau).
    \end{align*}
    The interchange of the integral and summation is justified by the uniform convergence
    of the sum defining $P_r$. Hence we have $A_r \mu = P_r * \mu$. 
\end{proof} 

Recall that a sequence of measures $\{\mu_n\}_{n=0}^\infty$ converges
in the weak* sense if for any $f \in C(\Circ)$, 
we have $\int_\Circ f\;d\mu_n \rightarrow \int_\Circ f\;d\mu$.
Any $f \in L^1(\Circ)$ can be interpreted as a measure $\mu_f$, by $d\mu_f = fd\ha$. 

\begin{proposition}
    Let $\mu$ be a complex Borel regular measure on $\Circ$. The functions $\{A_r \mu\}_{r \in (0,1)}$,
    interpreted as measures, converge in the weak* sense to $\mu$ as $r\rightarrow 1^-$.
\end{proposition}
\begin{proof}
    Let $f \in C(\Circ)$. We compute,
    \begin{align*}
        \int_\Circ f\;d\mu_{A_r\mu} &= \int_\Circ fP_r*\mu\;d\ha\\
        &= \int_\Circ\int_{\Circ} f(\zeta) P_r(\zeta\tau^{-1}) d\mu(\tau)d\ha(\zeta).
    \end{align*}
    However note that $P_r(\tau) = P_r(\tau^{-1})$, so by Fubini's theorem,
    \begin{align*}
        \int_\Circ f\;d\mu_{A_r \mu} &= \int_\Circ \int_\Circ P_r(\tau\zeta^{-1})f(\zeta)d\ha(\zeta)d\mu(\tau)\\
        &= \int_{\Circ} P_r*f \;d\mu.
    \end{align*}
    However, since $f$ is continuous, we have that $P_r*f\rightarrow f$ uniformly
    as $r\rightarrow 1^-$. Hence
    \begin{equation*}
        \lim_{r\rightarrow 1^-} \int_\Circ f\;d\mu_{A_rf} = \int_\Circ f\;d\mu
    \end{equation*}
    So $A_r\mu$ converges in the weak* sense to $\mu$.
\end{proof}

\section{Harmonic functions on $\Disc$}
Abel summation is much more heavily studied than Ces\`aro summation. The reason
for this is that there is a close connection between the Abel sums $A_r f$
for a function $f$ on $\Circ$, and functions on $\Disc := \{ z \in \Cplx\;:\; |z| < 1\}$
which are \emph{harmonic}.

Given $f \in L^1(\Circ)$, define $\tilde{f}:\Disc\rightarrow\Cplx$
by
\begin{equation*}
    \tilde{f}(r\zeta) = A_rf(\zeta) = (P_r*f)(\zeta)
\end{equation*}
for $r \in [0,1]$ and $\zeta \in \Circ$.

Note that, by Young's convolution inequality,
\begin{equation*}
    \left(\int_\Circ |\tilde{f}(r\zeta)|^p\;d\ha(\zeta)\right)^{1/p} \leq \|P_r\|_1 \|f\|_{p}.
\end{equation*}
Hence,
\begin{equation*}
    \sup_{r \in [0,1)} \left(\int_\Circ |\tilde{f}(r\zeta)|^p\;d\ha(\zeta)\right)^{1/p} \leq \|f\|_p.
\end{equation*} 




% Appendix A

\chapter{Lorentz Spaces} % Main appendix title

\label{AppendixB} % For referencing this appendix elsewhere, use \ref{AppendixA}

\lhead{Appendix B. \emph{Lorentz Spaces}} % This is for the header on each page - perhaps a shortened title


\section{Introduction}
Lorentz spaces form a broad generalisation of $L^p$ spaces
and weak $L^p$ spaces. These notes cover the basic definitions.

\section{Weak $L^p$-spaces}
Let $(X,\mathcal{A},\mu)$ be a $\sigma$-finite measure space. 
Given a measurable funciton $f:X\rightarrow \Cplx$, we may define the 
distribution function 
\begin{equation*}
    d_f(\alpha) = \mu\{x \in X\;:\;|f(x)| \geq \alpha\}.
\end{equation*}
for $\alpha \geq 0$. 

By Markov's inequality, for $p > 0$,
\begin{equation*}
    d_f(\alpha) \leq \frac{\|f\|_p^p}{\alpha^p}.
\end{equation*}

The linear span of the class of functions $f$ for which
\begin{equation*}
    d_f(\alpha) \leq \frac{C^p}{\alpha^p}
\end{equation*}
for some constant $C$ is called the weak $L^p$ space
or $L^{p,w}(X)$. Given $f \in L^{p,w}(X)$, define
\begin{equation*}
    \|f\|_{p,w} := \sup_{\alpha > 0} \alpha d_f(\alpha)^{1/p}.
\end{equation*}

\section{Non-increasing Rearrangements}
Again let $(X,\mathcal{A},\mu)$ be a $\sigma$-finite
measure space, and $f:X\rightarrow \Cplx$ is measurable. Then define,
for $t \geq 0$,
\begin{equation*}
    f^*(t) = \inf\{s \geq 0\;:\;d_f(s) \leq t\}.
\end{equation*}
$f^*$ is called the non-increasing rearrangement of $f$.

\begin{lemma}
    \begin{equation*}
        \left(\int_X |f|^p\;d\mu\right)^{1/p} = \left(\int_0^\infty (f^*)^p\;d\mu\right)^{1/p}
    \end{equation*}
\end{lemma}
\begin{lemma}
    \begin{equation*}
        \sup_{t>0} t^qf^*(t) = \sup_{\alpha > 0} \alpha d_f(\alpha)^{1/p}.
    \end{equation*}
\end{lemma}

Inspired by the above two results is the following definition,
\begin{definition}[Lorentz spaces]
    Let $p,q > 0$. For a $\sigma$-finite measure space $(X,\mathcal{A},\mu)$, 
    and a measurable function $f:X\rightarrow \Cplx$, define
    \begin{equation*}
        \|f\|_{p,q} = \left(\int_0^\infty (t^{1/p}f^*(t))^q\frac{dt}{t}\right)^{1/q}
    \end{equation*}
    and
    \begin{equation*}
        \|f\|_{p,\infty} = \sup_{t\geq 0} t^{1/p} f^*(t).
    \end{equation*}
\end{definition}

\section{Discrete spaces}
When $X$ is a set with counting measure, the $L^{p,q}$ spaces
are denotes $\ell^{p,q}(X)$. 


% Appendix A

\chapter{Interpolation} % Main appendix title

\label{AppendixC} % For referencing this appendix elsewhere, use \ref{AppendixA}

\lhead{Appendix C. \emph{Interpolation}} % This is for the header on each page - perhaps a shortened title



\section{Introduction}
The method of interpolation is a very powerful one in analysis,
and it allows many results to be obtained from ``edge" cases.

Historically, interpolation is motivated by the Riesz-Thorin theorem,
which we state here.
\begin{theorem}
    Let $p_0,q_0 \in [1,\infty]$ and $p_1,q_1 \in [1,\infty]$
    with $p_0 \neq p_1$ and $q_0 \neq q_1$. Suppose that $(U,\mu)$
    and $(V,\nu)$ are measure spaces. Let $T$ be an operator such that
    \begin{equation*}
        T:L^{p_0}(U)\rightarrow L^{q_0}(V)
    \end{equation*}
    with norm $M_0$ and
    \begin{equation*}
        T:L^{p_1}(U) \rightarrow L^{q_1}(V)
    \end{equation*}
    with norm $M_1$.
    
    Let $\theta \in (0,1)$, and $p_\theta^{-1} = \theta p_0^{-1}+(1-\theta)p_1^{-1}$,
    and $q_\theta^{-1} = \theta q_0^{-1}+(1-\theta)q_1^{-1}$. 
    
    Then,
    \begin{equation*}
        T:L^{p_\theta}(U)\rightarrow L^{q_\theta}(V)
    \end{equation*}
    with norm $M \leq M_0^{\theta} M_1^{1-\theta}$.
\end{theorem}
This theorem can be proved directly, however it is more insightful to prove it using
abstract interpolation theory. The purpose of these notes is to introduce this theory.


\section{Abstract Interpolation Theory}
Let $\NLS$ be the category of normed linear spaces with morphisms
given by bounded linear maps.

\begin{definition}
    A pair $X,Y \in \NLS$ is called a compatible pair if $X$ and $Y$
    are both subspaces of a topological vector space $U$.
\end{definition}

\begin{definition}
    The category $\NLS_1$ is the category of compatible pairs $(X,Y)$ of
    normed spaces, where a morphism $T:(X_1,Y_1)\rightarrow (X_2,Y_2)$
    is a linear map $T:X_1+Y_1\rightarrow X_2+Y_2$ such that $T:X_1\rightarrow X_2$
    and $T:Y_1\rightarrow Y_2$ is bounded.
\end{definition}

\begin{proposition}
    Let $\Delta:\NLS_1\rightarrow\NLS$ be the function that maps $(X,Y)$
    to $X\cap Y$, where $X \cap Y$ is given the norm,
    \begin{equation*}
        \|x\|_{X \cap Y} = \max\{\|x\|_X,\|x\|_Y\}.
    \end{equation*}
        
    Let $\Sigma:\NLS_1\rightarrow\NLS$ be given by $\Sigma((X,Y)) = X+Y$,
    where $X+Y$
    is given the norm,
    \begin{equation*}
        \|x\|_{X+Y} = \inf\{ \|x_1\|_X+\|x_2\|_Y \;:\;x = x_1+x_2,x_1 \in X,x_2 \in Y\}.
    \end{equation*}
    Then $\Delta$ and $\Sigma$ are functors. 
\end{proposition}

\begin{definition}
    An interpolation functor is a functor $\mathcal{F}:\NLS_1\rightarrow\NLS$
    such that for $(X,Y) \in \NLS_1$, we have 
    \begin{equation*}
        \Delta((X,Y)) \subseteq \mathcal{F}((X,Y)) \subseteq \Sigma((X,Y))
    \end{equation*}


We say that an interpolation functor $\mathcal{F}$
is \emph{uniform} if for any morphism $T:(X_1,Y_1)\rightarrow (X_2,Y_2)$ in $\NLS_1$,
we have
\begin{equation*}
    \|\mathcal{F}(T)\| \leq C\max\{\|T\|_{X_1\rightarrow X_2},\|T\|_{Y_1\rightarrow Y_2}\}.
\end{equation*}
for some constant $C > 0$. If $C = 1$, we say that $\mathcal{F}$ is \emph{exact}.

We say that an interpolation functor $\mathcal{F}$ is of exponent $\theta \in (0,1)$
if
\begin{equation*}
    \|\mathcal{F}(T)\| \leq C\|T\|_{X_1\rightarrow X_2}^\theta \|T\|_{Y_1\rightarrow Y_2}^{1-\theta}.
\end{equation*}
for some constant $C > 0$. If $C = 1$, we say that $\mathcal{F}$ is \emph{exact
of exponent $\theta$}.

\end{definition}

\section{Real Interpolation: The $K$ Method}
\begin{definition}
    Let $(X_0,X_1) \in \NLS_1$. For $x \in X_0+X_1$ and $t > 0$, define
    \begin{equation*}
        K(x,t;X_0,X_1) = \inf\{ \|x_0\|+t\|x_1\|\;:\;x_0 \in X_0,x_1 \in X_1,x = x_0+x_1\}.
    \end{equation*}
    Then for $\theta \in (0,1)$ and $q \in [1,\infty)$, define
    \begin{equation*}
        \|x\|_{\theta,q;K} := \left(\int_0^\infty (t^{-\theta}K(x,t;X_0,X_1))^q\;\frac{dt}{t}\right)^{1/q}
    \end{equation*}
    and for $\theta \in [0,1]$,
    \begin{equation*}
        \|x\|_{\theta,\infty;K} := \sup_{t > 0} t^{-\theta} K(x,t;X_0,X_1).
    \end{equation*}
    
    The space $K_{\theta,q}(X_0,X_1)$ is the set of $x \in X_0+X_1$
    such that $\|x\|_{\theta,q;K} < \infty$.
    
\end{definition}

\begin{proposition}
    The function $(X,Y) \mapsto K_{\theta,q}(X,Y)$ is an interpolation functor.
\end{proposition}

\section{Complex Interpolation}
\begin{definition}
    Let $(X_0,X_1)$ be a compatible pair of Banach spaces. Let $\mathcal{S} := \{z \in \Cplx\;:\;\Re(z) \in (0,1)\}$. 
    
    Define the set $\mathcal{F}(X_0,X_1)$ to be the space of functions $f:\overline{\mathcal{S}}\rightarrow X_0+X_1$ which are complex differentiabile in $\mathcal{S}$, continuous on $\overline{\mathcal{S}}$
    and bounded on $\partial \mathcal{S}$.
    
    It is true that $\mathcal{F}(X_0,X_1)$ is a Banach space under the norm,
    \begin{equation*}
        \|f\|_{\mathcal{F}(X_0,X_1)} = \max\{\sup_{t \in \Rl} \|f(it)\|,\sup_{t \in \Rl}\|f(1+it)\|\}.
    \end{equation*}
    
    Let $\theta \in (0,1)$. Then, define
    \begin{equation*}
        (X_0,X_1)_\theta = \{ f(\theta)\;:\;f \in \mathcal{F}(X_0,X_1)\}.
    \end{equation*}
    We define the norm,
    \begin{equation*}
        \|x\|_{(X_0,X_1)_\theta} = \inf\{\|f\|_{\mathcal{F}(X_0,X_1)} \;:\;x = f(\theta)\}.
    \end{equation*}
\end{definition}

\begin{proposition}
    The mapping $(X_0,X_1)\rightarrow (X_0,X_1)_\theta$ is an interpolation
    functor, which is exact of exponent $\theta$.
\end{proposition}

\section{Using interpolation}
Once we know how to describe certain interpolation spaces, results
such as the Riesz-Thorin theorem become immediate.
\begin{proposition}
    Let $X$ be a measure space.
    Let $p_0,p_1 \in [1,\infty]$, and $p_\theta^{-1} = \theta p_0^{-1}+(1-\theta)p_1^{-1}$
    for $\theta \in (0,1)$.
    Then
    \begin{equation*}
        (L^{p_0}(X),L^{p_1}(X))_\theta = L^{p_\theta}(X).
    \end{equation*}
\end{proposition}

Another immediate corollary:
\begin{proposition}
    Let $G$ be a locally compact group abelian group equipped
    with bivariant Haar measure $\mu$, and $p \in [1,\infty]$. For $\varphi \in L^1(G,\mu)$, the mpa
    $Tf = \varphi * f$ is bounded on $L^p(G,\mu)$, with norm less than or equal
    to $\|\varphi\|_1$.
\end{proposition}
\begin{proof}
    To prove the case $p = 1$, we let $f \in L^p(G)$, and compute,
    \begin{align*}
        \|\varphi * f\|_1 &= \int_G \left|\int_G\varphi(x-y)f(y)\;d\mu(y)\right|\;d\mu(x)\\
        &\leq \int_G \int_G |\varphi(x-y)||f(y)|\;d\mu(y)\;d\mu(x)\\
        &= \|\varphi\|_1\|f\|_1
    \end{align*}
    where the interchange of integrals is justified by Tonelli's theorem.
    
    Now for $p = \infty$, we compute,
    \begin{equation*}
        \|\varphi*f\|_\infty = \operatorname{ess-sup}_{x \in G} \left|\int_G \varphi(x-y)f(y)\;d\mu(y)\right|
    \end{equation*}
    By H\"older's inequality, this can be bounded by $\|\varphi\|_1\|f\|_\infty$.
    
    The rest of the cases
    follow from complex interpolation of the pair $(L^1(G,\mu),L^\infty(G,\mu)$.
\end{proof} 


% Appendix A

\chapter{Group Actions on Von Neumann Algebras} % Main appendix title

\label{AppendixD} % For referencing this appendix elsewhere, use \ref{AppendixA}

\lhead{Appendix D. \emph{Group Actions on Von Neumann Algebras}} % This is for the header on each page - perhaps a shortened title


\section{Introduction}
There is a beautiful and extraordinary generalisation of the Fourier
transform in the theory of Von Neumann algebras relating
to the action of compact groups. The purpose
of these notes is to give a basic exposition of this idea.

\section{Compact Group actions}
Let $G$ be a compact abelian group equipped with normalised Haar
measure $\mu$, and let $\M$ be a Von Neumann algebra.

Recall that a group action on a Von Neumann algebra
is a group action $\alpha:G\times \M\rightarrow \M$ such that
for all $g \in G$ the function $a\mapsto \alpha(g,a)$ is an algebra
homomorphism.

Suppose that $G$ acts on $M$ in a way that is
\begin{enumerate}
    \item{} \emph{ergodic}: the only projections in $\M$
    fixed by $G$ are $0$ and $1$ and
    \item{} \emph{free}: there is no nontrivial projection $p \in \M$
    such that some $g \in G$ not equal to the identity fixes all of $p\M p$.
\end{enumerate}

We then have the following result:
\begin{proposition}
    Suppose that $G$ is a compact abelian group that acts freely and ergodically
    on the Von Neumann algebra $\M$, by the action $\alpha:G\times\M\rightarrow\M$. Then there is a set 
    \begin{equation*}
        \{ u(p) \;:\; p \in \widehat{G}\}
    \end{equation*}
    of unitary eigenoperators for the action indexed by the dual group $\widehat{G}$,
    and the map $p\mapsto u(p)$ is a representation of $\widehat{G}$.
\end{proposition}
\begin{proof}
    Let $x \in \M$ and $p \in \widehat{G}$. Define
    \begin{equation*}
        \hat{x}(p) := \int_G \alpha(s,x) p(s)^{-1} \;d\mu(s).
    \end{equation*}
    
    Let $g \in G$, then we can compute,
    \begin{align*}
        \alpha(g,\hat{x}(p)) &= \int_G \alpha(sg,x)p(s)^{-1} \;d\mu(s)\\
        &= p(g)\hat{x}(p).
    \end{align*}
    Hence we have that $\hat{x}(p)$ is an eigenoperator for the action of $G$.
    Hence, for $x,y \in \M$, we have that $\hat{x}(p)\hat{y}(p)^*$ is a fixed
    point of $G$, hence a scalar multiple of $1$. 
    
    Then define
    \begin{equation*}
        u(p) = \frac{\hat{x}(p)}{\|\hat{x}(p)\|}.
    \end{equation*}
    Hence $u$ is unitary, and for each $y \in \M$, we have $\hat{y}(p)$ is a scalar
    multiple of $u(p)$.
\end{proof} 
\begin{example}
    The prototypical example of this decomposition is for $\M = L^\infty(\Circ)$,
    and $G = \Circ$ with Haar measure  acting on $\M$ by translation. Then $\widehat{G} = \Itgr$,
    and for $f \in L^\infty(\Circ)$ and $n \in \Itgr$, we have
    \begin{equation*}
        \hat{f}(n)(\zeta) = \int_\Circ f(\tau\zeta) \tau^{-n} d\ha(\tau)
    \end{equation*}
    This is simply the $n$th Fourier coefficient times $\zeta^n$. Hence we have
    \begin{equation*}
        u(n)(\zeta) = \zeta^n.
    \end{equation*}
    So the system of unitaries is the set of monomials on $\Circ$. 
\end{example}
The important feature of this system of unitaries is that it spans
$\M$. First we recall the definition of the $\sigma$-weak topology.

\begin{definition}
    Let $\Hilb$ be a Hilbert space, and let $\{\xi_j\}_{j = 0}^\infty$
    and $\{\eta_j\}_{j=0}^\infty$ be sequences of elements of $\Hilb$
    such that 
    \begin{align*}
        \sum_{j=0}^\infty \|\xi_j\|^2 &< \infty\\
        \sum_{j=0}^\infty \|\eta_j\|^2 &< \infty.
    \end{align*}
    Then for $a \in \mathcal{B}(\Hilb)$, define the semi-norm,
    \begin{equation*}
        \left| \sum_{j=0}^\infty \langle \xi_j,a\eta_j\rangle\right|.
    \end{equation*}
    This system of semi-norms defines the $\sigma$-weak topology
    on $\mathcal{B}(\Hilb)$.
\end{definition}



\begin{proposition}
    Let $G$ be a compact abelian group acting freely and ergodically 
    on a Von Neumann algebra $\M$. Let
    \begin{equation*}
        \mathcal{P} = \{ u(p)\;:\; p \in \widehat{G}\}
    \end{equation*}
    be the corresponding unitary eigenoperators of $G$. Then the span
    of $\mathcal{P}$ is dense in the $\sigma$-weak topology on $\M$.
\end{proposition}
\begin{proof}
    (sketch: See \cite{pedersen} 8.1.6 for details) This follows from the claim that the \emph{Arveson spectrum}
    $\operatorname{Sp}(\alpha)$
    of $\alpha$ is $\widehat{G}$. We claim that since
    \begin{equation*}
        \operatorname{Sp}^{\perp}(\alpha) = \{ s \in G\;:\; \alpha(s,x) = x\;\text{ for all }x \in \M\}
    \end{equation*} 
    we have that $\operatorname{Sp}^\perp(\alpha) = \{0\}$ because $\alpha$
    is free,
    from which it follows that $\operatorname{Sp}(\alpha) = \widehat{G}$.
    
    Consequently, the system $\{u(p)\;:\; p \in \widehat{G}\}$ is $\sigma$-weakly
    dense in $\M$.
\end{proof}

\section{The space $\mathcal{L}^2(\M,\tau)$}
Suppose that $\tau$ is a faithful trace on $\M$ such that $\tau(1) = 1$,
and $\tau$ is invariant under the action of $G$, that is for
all $g \in G$ and $x \in \M$ we have $\tau(\alpha(g,x)) = \tau(x)$. 
Then the map $(x,y)\mapsto \tau(x^*y)$ is an inner product on $\M$,
and the completion of $\M$ in this inner product is denoted $\mathcal{L}^2(\M,\tau)$.

We require the following lemma from Pedersen \cite{pedersen},
Theorem 3.6.5,
\begin{lemma}
    A state $\varphi$ on a Von Neumann algebra is normal if and only if it is
    $\sigma$-weakly continuous.
\end{lemma}


Since $\M$ carries a group action, such a trace exists.
\begin{lemma}
    Let $\M$ be a Von Neumann algebra, and let $G$ act on $\M$ freely and ergodically.
    Then there is a $G$-invariant state on $\M$, and it is a faithful normal
    trace.
\end{lemma}
\begin{proof}
    Define for $x \in \M$, 
    \begin{equation*}
        \tau(x) = \int_G \alpha(s,x)\;d\mu(s) = \hat{x}(0).
    \end{equation*}
    See that $\alpha(g,\tau(x)) = \tau(x)$, so $\tau(x)$
    is $G$-invariant so by ergodicity must be in $\Cplx 1$. If we identify
    $\Cplx 1$ with $\Cplx$, we can think of $\tau(x)$ as a scalar, so $\tau(x)$
    is a state. By the continuity of the group action, we have that $\tau$ is normal,
    and if $\tau(x^*x) = 0$, we must have $\alpha(s,x)^*\alpha(s,x) = 0$
    for all $s \in G$, so $x = 0$. Hence $\tau$ is faithful.
    
    To prove that prove that $\tau$ is a trace it is sufficient to prove that
    \begin{equation*}
        \tau(u(p)u(q)) = \tau(u(q)u(p))
    \end{equation*}
    for all $p,q \in \widehat{G}$. 
    
    Let $g \in G$, then we have
    \begin{align*}
        \tau(u(p)u(q)) &= \tau(\alpha(g,u(p)u(q)))\\
        &= p(g)q(g)\tau(u(p)u(q)).
    \end{align*}
    If $p \neq q^{-1}$, we can find $g$ such that $p(g)q(g) \neq 1$. Thus,
    $\tau(u(p)u(q)) = 0$.
    
    Otherwise, if $p = q^{-1}$ there is a scalar $\lambda_p$
    with $|\lambda_p| = 1$ such that
    \begin{align*}
        \tau(u(p)u(q)) = \lambda_p\tau(u(p)u(p)^*) = \lambda_p.
    \end{align*}
    Hence $\tau(u(p)u(q)) = \tau(u(q)u(p))$.
    
    Hence, since $\tau$ is normal, it is $\sigma$-weakly continuous.
    Since the set $\{u(p)\;:\;p \in \widehat{G}\}$ is $\sigma$-weakly
    dense in $\M$, we have that $\tau(xy) = \tau(yx)$ for all $x,y \in \M$.
    
\end{proof} 


\begin{proposition}
    Let $\mathcal{P} := \{u(p)\;:\;p \in \widehat{G}\}$ be the set of unitary eigenoperators
    corresponding to the action of $G$. Then $\{u(p)\;:\; p \in \widehat{G}\}$
    is an orthonormal basis for $\mathcal{L}^2(\M,\tau)$.
\end{proposition}
\begin{proof}
    By standard Hilbert space theory, it is sufficient to prove that $\mathcal{P}$
    is orthonormal and has dense span.
    
    First we prove ortho-normality. Let $p,q \in \widehat{G}$. Then for all $g \in G$,
    \begin{align*}
        \tau(u(p)^*u(q)) &= \tau(p(g)^{-1}q(g)u(p)^*u(q))\\
        &= p(g)^{-1}q(g)\tau(u(p)^*u(q)).
    \end{align*}
    If $p \neq q$, there is some $g \in G$ such that $p(g)^{-1}q(p) \neq 1$,
    so we conclude that $\tau(u(p)^*u(q)) = 0$.
    If $p = q$, then by unitarity we have
    \begin{equation*}
        \tau(u(p)^*u(q)) = \tau(1) = 1.
    \end{equation*}
    Hence $\mathcal{P}$ is orthonormal.
    
    To prove that the span of $\mathcal{P}$ is dense in $\mathcal{L}^2(\M,\tau)$,
    it is sufficient to prove that it is dense in $\M$ in the norm $\|x\|_2 := \sqrt{\tau(x^*x)}$.
    This follows from the $\sigma$-weak continuity of $\tau$.
\end{proof}

\begin{remark}
    In fact, for any $p \geq 1$, we have that the norm $\|x\|_p = (\tau(|x|^p))^{1/p}$
    is $\sigma$-weakly continuous, so $\{u(p)\;:\;p \in \widehat{G}\}$
    is dense in $\mathcal{L}^p(M,\tau)$, defined as the completion of $\M$
    in the norm $\|\cdot\|_p$. 
\end{remark}


\section{Non-commutative Harmonic Analysis}
(The following is entirely original and therefore suspect)

We now fix $G = \Circ^d$, so that $\widehat{G} = \Itgr^d$. It is of interest
to prove certain results that are analogous to classical results of harmonic
analysis. In particular, one known classical result is that if $f \in L^1(\Circ^d,\ha)$,
then the Fourier coefficients $\hat{f}(n)$ vanish as $\|n\|\rightarrow \infty$. We now
prove an analogy of this result. 

Let $\ha$ denote the normalised Haar measure on $\Circ^d$.

From now on, denote the action of $t \in \Circ^d$ on $x \in \M$ as $\alpha_t(x)$.

Fix $\M$ a Von Neumann Algebra, and let $\Circ^d$ act on $\M$ freely
and ergodically. Let $\tau$ be the unique normalised faithful $\Circ^d$-invariant trace
on $\M$. Let
\begin{equation*}
    \mathcal{P} := \{ u(n) \;:\; n \in \Itgr^d\}
\end{equation*}
be the spanning system of unitary eigenoperators of the group action. 

For $a \in \mathcal{L}^1(\M,\tau)$ and $n \in \Itgr^d$, define
\begin{equation*}
    \hat{a}(n) := \tau(au(n)^*)
\end{equation*}

For any $a \in \mathcal{L}^2(\M,\tau)$, we have that
\begin{equation*}
    a = \sum_{n \in \Itgr^d} \hat{a}(n)u(n)
\end{equation*}
where the convergence is in the $\mathcal{L}^2$ norm. This is
an isomorphism with $\ell^2(\Itgr^d)$.
In general, for $N = (N_1,\ldots,N_d) \in \Itgr^d$, define
\begin{equation*}
    S_N a := \sum_{n = -N}^N \hat{a}(n)u(n)
\end{equation*}
where the summation runs over all multi-indices $n = (n_1,\ldots,n_d)$
such that for each $1\leq j \leq d$, we have $-N_j \leq n_j \leq N_J$.
and
\begin{equation*}
    \sigma_N a := \frac{1}{N_1N_2\cdots N_d}\sum_{n = (1,\ldots,1)}^N S_n a.
\end{equation*}
and the summation runs over multi-indices $n = (n_1,\ldots,n_d)$
such that for each $j$, $1 \leq n_j \leq N_j$.

Now we define the subspaces of ``\emph{continuous}" and ``\emph{uniformly continuous}"
functions in $\M$. 
\begin{definition}
    Define $\mathcal{C}(\M)$ to be the closure of the span of $\mathcal{P}$
    in the norm topology of $\M$.
    
    Define $\mathcal{U}(\M)$ to be the set of elements of $\M$
    such that for any $\varepsilon > 0$ there exists $\delta > 0$ such that
    \begin{equation*}
        \sup_{|t| < \delta} \|\alpha_t(x)-x\| < \varepsilon.    
    \end{equation*}
    Where, for $t \in \Circ^d$, If $t = (e^{2\pi i\varphi_1},\ldots,e^{2\pi i\varphi_d})$, we 
    denote
    \begin{equation*}
        |t| = \max\{|\varphi_1|,\ldots,|\varphi_d|\}.
    \end{equation*}
\end{definition}

\begin{definition}
    Let $n = (n_1,\ldots,n_d) \in \Itgr^d$, and $t = (t_1,\ldots,t_d) \in \Circ^d$.
    Then we introduce the notation
    \begin{equation*}
        t^d := t_1^{n_1}\cdots t_d^{n_d}.
    \end{equation*}    
\end{definition}

\begin{lemma}
    We have the inclusion,
    \begin{equation*}
        \mathcal{C}(\M) \subseteq \mathcal{U}(\M).
    \end{equation*}
\end{lemma}
\begin{proof}
    First let $u(n) \in \mathcal{P}$. Then for $t \in \Circ^d$,
    we have
    \begin{equation*}
        \alpha_t(u(n)) = t^n u(n)
    \end{equation*}
    Hence
    \begin{equation*}
        \|\alpha_t(u(n))-u(n)\| = |t^n-1|\|u(n)\|.
    \end{equation*}
    So since $\lim_{t\rightarrow 1} |t^n - 1| = 0$, we have 
    that for any $\varepsilon > 0$ there exists $\delta > 0$ such that
    \begin{equation*}
        \sup_{|t| < \delta} \|\alpha_t(u(n))-u(n)\| < \varepsilon. 
    \end{equation*} 
    
    Now let $a$ be in the linear span of $\mathcal{P}$. Since $a$
    is only a finite linear combination of terms of the form $u(n)$, $n \in \Itgr$, we have
    that for any $\varepsilon > 0$ there exists a $\delta > 0$ such that
    \begin{equation*}
        \sup_{|t| <  \delta} \|\alpha_t(a) - a\| < \varepsilon.
    \end{equation*}
    Hence the linear span of $\mathcal{P}$ is contained in $\mathcal{U}(\M)$.
    We will be finished if we can show that $\mathcal{U}(\M)$ is closed
    in the norm topology.
    
    Suppose that $\{x_n\}_{n=0}^\infty$ is a sequence in $\mathcal{U}(\M)$ converging
    in the norm topology to $x \in \M$. Then let $t \in \Circ^d$, by the triangle inequality,
    \begin{equation*}
        \|\alpha_t(x)-x\| < \|\alpha_t(x)-\alpha_t(x_n)\| + \|\alpha_t(x_n)-x_n\| + \|x_n - x\|
    \end{equation*}
    By assumption, the group action is continuous in the norm topology,
    so there is a constant $C$ such that $\|\alpha_t(x)-\alpha_t(x_n)\| < C\|x-x_n\|$. 
    Hence the result follows. 
\end{proof}

\begin{definition}
    Suppose that $\varphi \in L^1(\Circ^d,\ha)$, and $x \in \M$. Define the \emph{convolution}
    \begin{equation*}
        \varphi * x = \int_{\Circ^d} \varphi(t)\alpha_t(x)\;d\ha(t).
    \end{equation*}
\end{definition}

It can be proved that
\begin{equation*}
    S_N x = D_N * x
\end{equation*}
where 
\begin{equation*}
    D_N(t) = \prod_{j=1}^d\frac{t_j^{N_j+1/2}-t_j^{-N_j-1/2}}{t_j^{1/2}-t_j^{-1/2}}
\end{equation*}
and
\begin{equation*}
    \sigma_N x = F_N*x
\end{equation*}
where
\begin{equation*}
    F_N(t) = \prod_{j=1}^d\frac{t_j^{N_j}-2+t_j^{-N_j}}{N(t_j^{1/2}-t_j^{-1/2})^2}
\end{equation*}
where the fractional powers in the formulae for $D_N$ and $F_N$ are defined
in a principal value sense. See \cite{me} for proofs.

Now we define approximate identities,
\begin{definition}
    A net $\{\Phi_\lambda\}_{\lambda \in \Lambda} \subset L^1(\Circ^d,\ha)$ is called
    an approximation to the identity if,
    \begin{enumerate}
        \item{} For all $\lambda$, we have $\int_{\Circ^d} \Phi_\lambda \;d\ha = 1$.
        \item{} We have $\sup_{\lambda} \int_{\Circ^d} |\Phi_\lambda|\;d\ha < \infty$
        \item{} For any $\delta > 0$, we have $\lim_{\lambda \in \Lambda} \int_{|t| > \delta} |\Phi_\lambda|\;d\ha = 0$.
    \end{enumerate}
\end{definition}

It can be proved that the sequence $\{F_n\}_{n\in\Ntrl^d}$ is an approximate
identity (see \cite{me}).

Approximate identities are so named because of the following two results:
\begin{proposition}
    Let $x \in \mathcal{U}(\M)$, and let $\{\Phi_\lambda\}_{\lambda \in \Lambda}$ be an 
    approximate identity. Then we have
    \begin{equation*}
        \lim_{\lambda \in \Lambda} \|\Phi_\lambda*x - x\| = 0.
    \end{equation*}
\end{proposition}
\begin{proof}
    Using the fact that $\int_\Circ \Phi_\lambda\;d\mu = 1$, we compute,
    \begin{equation*}
        \Phi_n*x-x = \int_\Circ \Phi_\lambda(t)(\alpha_t(x)-x)\;d\mu(t).
    \end{equation*}
    Let $\varepsilon > 0$. Choose $\delta$ small enough such that
    \begin{equation*}
        \sup_{|t| < \delta} \|\alpha_t(x)-x\| < \varepsilon.
    \end{equation*}
    Now estimate,
    \begin{equation*}
        \|\Phi_\lambda*x - x\| \leq \int_{|t| < \delta} |\Phi_\lambda(t)|\|\alpha_t(x)-x\|\;d\mu(t) + (C+1)\int_{|t| \geq \delta} |\Phi_\lambda(t)|\|x\| d\mu(t)
    \end{equation*}
    where $C$ is a constant such that $\|\alpha_t(x)\| < C\|x\|$.
    So taking the limit over $\lambda$, we have
    \begin{equation*}
        \lim_{\lambda \in \Lambda} \|\Phi_\lambda*x-x\| < \varepsilon \sup_{\lambda \in \Lambda} \int_\Circ |\Phi_n|\;d\ha.
    \end{equation*}
    But $\varepsilon$ is arbitrary, so the result follows.
\end{proof}

\begin{proposition}
    We have that
    \begin{equation*}
        \mathcal{C}(\M) = \mathcal{U}(\M).
    \end{equation*}
\end{proposition}
\begin{proof}
    Let $a \in \mathcal{U}(\M)$. Since $\{F_n\}_{n\in\Ntrl^d}$ is an approximate identity, we have
    that $F_n*a\rightarrow a$ in the norm topology. But $F_n*a \in \Span(\mathcal{P})$.
    Hence $a \in \mathcal{C}(\M)$.
\end{proof}


As usual, define $\|x\|_p = \tau(|x|^p)^{1/p}$. We require the following lemma:
\begin{lemma}
    Let $p \geq 1$. Suppose $\varphi \in L^1(\Circ^d,\ha)$, and $x \in \mathcal{L}^p(\M,\tau)$. Then
    \begin{equation*}
        \|\varphi*x\|_p \leq \|\varphi\|_1\|x\|_p.
    \end{equation*}
\end{lemma} 
\begin{proof}
    First we establish the case $p  =  1$. This is a computation,
    \begin{equation*}
        \|\varphi * x\|_1 \leq \int_{\Circ^d} |\varphi(t)|\|\alpha_{t}(a)\|_1\;d\ha(t).
    \end{equation*}
    Similarly, we define $\|x\|_\infty = \|x\|$. Thus the $p = \infty$ case,
    \begin{equation*}
        \|\varphi*x\|_\infty \leq \|\varphi\|_1\|x\|.
    \end{equation*}
    
    Hence by interpolation, the result follows.
\end{proof} 


\begin{proposition}
     If $\{\Phi_\lambda\}_{\lambda \in \Lambda}$
    is an approximate identity, and $x \in \mathcal{L}^p(\M,\tau)$
    and $p \geq 1$
    then
    \begin{equation*}
        \lim_{\lambda \in \Lambda}\|\Phi_\lambda*x-x\|_p = 0.
    \end{equation*}    
\end{proposition}
\begin{proof}
    Let $\varepsilon > 0$.
    Since $\mathcal{U}(\M)$ contains the linear span of $\mathcal{P}$,
    and the linear span of $\mathcal{P}$ is dense in $\mathcal{L}^p(\M,\tau)$
    in the $\|.\|_p$ norm, we can find $y \in \mathcal{U}(\M)$ such
    that $\|x-y\|_p < \varepsilon$. Hence,
    \begin{align*}
        \|\Phi_\lambda*x-x\|_p &\leq \|\Phi_\lambda*x-\Phi_\lambda*y\|_p + \|\Phi_\lambda*y-y\|_p + \|y-x\|_p\\
        &\leq \sup_{\mu \in \Lambda} \|\Phi_\mu\|_1\varepsilon + \|\Phi_\lambda*y-y\|_p + \varepsilon.
    \end{align*}
    Now take the limit over $\lambda$, and thus we obtain the result.
\end{proof}

At last we can prove the following result:
\begin{proposition}
    Let $x \in \mathcal{L}^1(\M,\tau)$. Then we have $\hat{x}(n) \rightarrow 0$
    as $\|n\|\rightarrow\infty$.
\end{proposition}
\begin{proof}
    For $n = (n_1,\ldots,n_d) \in \Itgr^d$, denote $|n| := (n_1,\ldots,n_d)$.
    We have that $\tau(\sigma_{|N|-1}xu(N)^*) = 0$. Hence,
    \begin{align*}
        |\hat{x}(n)| &= |\tau(xu(n)^*)|\\
&\leq |\tau((x-\sigma_{|n|-1}x)u(n)^*)|\\
&\leq \|u(n)^*\|\|x-\sigma_{|n|-1}x\|_1\\
&= \|x-F_{|n|-1}*x\|_1.
    \end{align*}
    But the right hand side vanishes as $\|n\|\rightarrow\infty$.
\end{proof}

\section{The Operators $\delta_j$}
Let $\M$ be a Von Neumann Algebra, and let $\Circ^d$
act freely and ergodically on $\M$. 

The generalised differentiation operator $\delta_j$ is defined
as the \emph{infinitesimal generator} of the action of $\Circ^d$, as follows,
\begin{definition}
    Let $j \in \{1,\ldots,d\}$. 
    
    For $t \in \Circ$, we have an action $\alpha^j_t$ on $\M$
    which is the action of the $j$th coordinate of $\Circ^d$ on $\M$
    
    For $x \in \mathcal{M}$, we define 
    \begin{equation*}
        \delta_j(x) = \lim_{t\rightarrow 1} \frac{\alpha^j_t(x)-x}{|t|}
    \end{equation*}
    where $|t|$ is the minimal normalised arc length between $t$ and $1$.
    
    The limit is in the sense of the norm topology on $\M$.
    
    This limit may not exist for all $x$. Let $\Dom(\delta_j)$
    be the set of all $x$ such that $\delta_j(x)$ exists.
\end{definition}
Note that $\Dom(\delta_j)$ is automatically a vector space.



\begin{lemma}
    We have $\mathcal{P} \subset \Dom(\delta_j)$.
\end{lemma}
\begin{proof}
    For $t \in \Circ$, and $u(n) \in \mathcal{P}$ with $n = (n_1,\ldots,n_d)$, we have $\alpha^j_tu(n) = t^n_ju(n)$.
    
    Hence,
    \begin{equation*}
        \frac{\alpha^j_t(u(n))-u(n)}{|t|} = \frac{t^{n_j}-1}{|t|}u(n).
    \end{equation*}
    
    We parametrise $\Circ$ by $t\mapsto \exp( i \theta)$, for $\theta \in [0,2\pi)$. Then we
    have
    \begin{equation*}
        \frac{t^{n_j}-1}{|t|} = \frac{\exp(in_j\theta)-1}{\theta}.
    \end{equation*}
    Hence,
    \begin{align*}
        \delta_j(u(n)) &= \lim_{t\rightarrow 1} \frac{t^{n_j}-1}{|t|}u(n)\\
                &= \lim_{\theta\rightarrow 0} \frac{\exp(in_j\theta)-1}{\theta}u(n)\\
                &= in_ju(n).
    \end{align*}
    Hence, $u(n) \in \Dom(\delta_j)$.
\end{proof}


\begin{proposition}
    Suppose that $x \in \Dom(\delta_j)$. Then 
    \begin{equation*}
        \widehat{\delta_j(x)}(n) = in_j\widehat{x}(n).
    \end{equation*}
\end{proposition}
\begin{proof}
    By definition, $\Dom(\delta_j) \subseteq \M \subseteq \mathcal{L}^1(\M,\tau)$. 
    
    Let $x \in \Dom(\delta_j)$.
    
    Hence, we have $F_n*x\rightarrow x$ in the $\mathcal{L}^1$ sense, and since $\delta_j(x) \in \mathcal{L}^1(\Circ)$, we have $F_n*\D(x)\rightarrow \D(x)$ in the $\mathcal{L}^1$ sense. 
    
    Note that since $F_n*x$ is in the linear span of $\mathcal{P}$, we
    have $F_n*x \in \Dom(\delta_j)$. See that
    \begin{equation*}
        \delta_j(F_n *x) = \frac{1}{n_1n_2\cdots n_d}\sum_{k=0}^n \delta_j(D_k* x).
    \end{equation*}
    where the sum is over multi-indices $k = (k_1,\ldots,k_d)$ with each $1\leq k_m \leq n_m$.
    We also have,
    \begin{align*}
        \delta_j(D_k* x) &= \sum_{j=-k}^k \hat{x}(j)\D(u(j))\\
        &= \sum_{j=-k}^k ij\hat{x}(j)u(j).        
    \end{align*}
    
    Now we have,
    \begin{align*}
        \widehat{\delta_j(x)}(n) &= \lim_{k\rightarrow\infty} \tau(F_k*xu(-n))\\
                           &= \lim_{k\rightarrow\infty} in_j\hat{x}(n).
    \end{align*} 
    
    
\end{proof}


Now define 
\begin{equation*}
    \D_j := \frac{1}{i} \delta_j.
\end{equation*}
We define $\Dom(\D_j) = \Dom(\delta_j)$.

Hence we have the formula,
\begin{equation*}
    \widehat{\D_j x}(n) = n_j\hat{x}(n).
\end{equation*}


% Appendix A

\chapter{Dirac Operators} % Main appendix title

\label{AppendixE} % For referencing this appendix elsewhere, use \ref{AppendixA}

\lhead{Appendix E. \emph{Dirac Operators}} % This is for the header on each page - perhaps a shortened title



\section{Introduction}
This chapter is intended to give an introduction to the relationship between
the Dirac operator and the exterior algebra. The general philosophy
is that any expressions involving coordinates are to be avoided.

Throughout these notes, $(M,g)$ is a Riemannian manifold.

\section{Music, Clifford bundles and Modules}
A metric $g$ on a manifold $M$ gives us a canonical isomorphism between $T^*M$
and $TM$, called $\sharp$, pronounced ``sharp". For $x \in M$, given a linear functional
$\omega \in T^*_xM$ we define $\sharp\omega$ to be the unique
vector such that $\omega(v) = g(\sharp\omega,v)$ for all $v \in T_xM$.
This is called the ``musical isomorphism".

The Clifford bundle of $(M,g)$ is a vector bundle on $M$ defined as follows.
\begin{definition}
    Let $x \in M$. The clifford algebra at $x$, $\Cliff_x(M,g)$ is defined
    as the free associative unital algebra generated by $T_xM$ modulo the relation
    \begin{equation*}
        uv+vu = -2g(u,v)1
    \end{equation*}
    where $u,v \in \Cliff_x(M,g)$, and $1$ is the identity in $\Cliff_x(M,g)$.
    
    
    The clifford bundle $\Cliff(M,g)$ is the vector bundle on $M$ whose fibres are 
    $\Cliff_x(M,g)$.
\end{definition}
Let's not care about the topology on $\Cliff(M,g)$ at the moment.

There is clearly an embedding $TM \hookrightarrow \Cliff(M,g)$. 

Now for a vector bundle $V$ on $M$, we say that $V$ is a clifford module
if there is a right multiplication map $\gamma:\Cliff(M,g) \otimes V\rightarrow V$. 

A \emph{connection} on $V$ is a linear map
\begin{equation*}
    \nabla:V\rightarrow T^*M\otimes V.
\end{equation*}
satisfying the Leibniz rule, for $f \in C^\infty(M)$ and $v \in V$,
\begin{equation*}
    \nabla(fv) = df \otimes v +f\nabla(v).
\end{equation*}

Now we may define a \emph{Dirac operator}. Suppose $V$
is a clifford bundle with connection $\nabla$. Then the composition of linear maps,
\begin{equation*}
    V \xrightarrow{\nabla} T^*M \otimes V \xrightarrow{\sharp\otimes I} TM\otimes V \xrightarrow{\gamma} V
\end{equation*}
is called the Dirac operator associated with $V$ and $\nabla$. 

\section{Relationship with differentials}
Suppose we have a clifford bundle $V$ with connection $\nabla$. 

Via the musical isomorphism, we may regard any differential form $\omega \in \Gamma(T^*M)$
as an operator on $\Gamma(V)$, since $\sharp(\omega)$ is an element of $\Gamma(\Cliff(M,g))$,
it may act on $V$.

Similarly, by pointwise multiplication, any $f \in C^\infty(M)$ is an operator
on $\Gamma(V)$. 
\begin{theorem}
    We have an equality of operators on $\Gamma(V)$,
    \begin{equation*}
        [D,f] = df.
    \end{equation*}
    for any $f \in C^\infty(M)$.
\end{theorem}
\begin{proof}
    Let $f \in C^\infty(M)$ and $v \in \Gamma(v)$. Let 
    us compute $D(fv)$.
    
    By definition,
    \begin{equation*}
        D(fv) = (\gamma\circ(\sharp\otimes I)\circ\nabla)(fv).
    \end{equation*}
    
    By the Leibniz rule,
    \begin{equation*}
        (\sharp\otimes I)\nabla(fv) = \sharp(df)\otimes v + (I\otimes f)(\sharp \otimes 1)\nabla(v).
    \end{equation*}
    Hence,
    \begin{equation*}
        D(fv) = \gamma(\sharp(df))v + fD(v).
    \end{equation*}
    Therefore, $[D,f]v = df(v)$.
\end{proof}


\addtocontents{toc}{\vspace{2em}} % Add a gap in the Contents, for aesthetics

\backmatter
 
%----------------------------------------------------------------------------------------
%	BIBLIOGRAPHY
%----------------------------------------------------------------------------------------

\label{Bibliography}

\lhead{\emph{Bibliography}} % Change the page header to say "Bibliography"

\bibliographystyle{unsrtnat} % Use the "unsrtnat" BibTeX style for formatting the Bibliography

\bibliography{Bibliography} % The references (bibliography) information are stored in the file named "Bibliography.bib"

\end{document}  
